\documentclass{beamer}
\usepackage[utf8]{inputenc}
\usepackage{graphicx}
\title{Handwritten Letters Presentation}
\date{}
\begin{document}
\frame{\titlepage}
\begin{frame}[fragile]{German Letter}
\begin{verbatim}
Hofheim, den 22.11.
Liebes Mech.!
Herzlichen Dank für Deinen
Brief. Wann schickst Du mir
endlich die Bilder vom
Säntis? Ich warte schon ein
halbes Jahr darauf. Kannst Du
mir Eure Weihnachtswünsche
schreiben? Möglichst vor dem
1. Advent. Da mache ich in
Frankfurt nämlich große
Einkäufe. Was schenkt Ihr denn
den verschiedenen Verwand-
ten? Sind Giesel. und Herr
Geser schon weg? wohin ge-

hen sie eigentlich? Ist der
meine Betriebsleiter schon hier?
Was macht Ulla? Bis wann
bleibt sie noch? Hat sie schon
eine Stelle?
War es am 11. 11. im Semi-
nar schön? Hast Du schon
eine Adresse aus England?
Heute ist Feiertag. Ich habe
den ganzen Vormittag Platten
gespielt. Das wär' so was
für Dich. Ich stricke zur Zeit
einen Pullover für mich,
weiß. Mit meinem Mantel
bin ich sehr zufrieden. Gestern
war ich zum 1. Male in
einer englischen Diskussions-

er
Aien?
schon
uni =
d?
abe
ffen
as
zeit
ich,
lautet
gestern
im
ussions.
Stunden im Amerikahaus.
Es war schön. Ich habe aber
gemerkt, daß ich viele Wör-
ter verlernt habe. Morgen
abend gehe ich in einen
Singkreis, auch zum 1. Mal.
Herzlichen Gruß!
Dorle

Liebe Mutter!
Wie geht es Dir? Ich will Dir
mal kurz die Apfelpreise
auf dem Großmarkt in
Frankfurt schreiben. Coxe haben
wir für 48 und 50 Pf pro Pfund
an Kleinhändler verkauft, da-
bei sind noch würmige drin.
Die große ist allerdings sehr
schön. Goldparmänen gehen
schleppend, 25 Pf, aber zum
Teil schlechter als A. Es gibt auch
für 32 Pf. Boskoops kosten nur
20 bis 25 Pf, sind allerdings nicht
so besonders schön. Wir verkaufen
jetzt noch keine Boskoops. In Frank-
furt im Laden werden die Coxe
mit 70 Pf verkauft.

Liebe Mutter, vorhin bekam
ich Deinen Brief. Und dein
Urlaub ist es sehr bestimmt,
denn das Babettche, unsere Haus-
gehilfin geht am 1.12. weg. Da
ist für's Haus wieder wohl
niemand hier. Mal sehen, ob
es sich doch noch einrichten
läßt.
Hier rät mir jeder ab, Geld auf
die Sparkasse zu legen, weil
es sehr wahrscheinlich Inflation
wird.
Jetzt am 3.12. in Frankfurt
groß einkaufen. Ich brauche Unter-
wäsche, Schlafanzüge. Dann will
ich mir ein Kleid kaufen für
Sonntags. Ich denke, das kommt
billiger als machen lassen.
Dann brauche ich noch 1 bis 2
Arbeitshosen, eine Kittelschürze,
und eine gute Stoffjacke. Mit
dem Mantel bin ich sehr zufrie-
den. Ich stricke mir zur Zeit einen
Pullover, will mir aber doch einen

stricken. Die Sachen bekommt
ich in Frankfurt aber alle
viel billiger als dort bei Euch.
Daß ich an Büchern immer
Spaß habe, weißt Du ja. Das
beige Wollkleid und dein
handgewebten Gürtel ist
mir zu kurz, und auch ein
wenig zu eng. Ob es der Chris-
tel paßt? Und eine weiße
Polobluse, die noch beinahe
neu ist.

Was soll ich denn der
Großmama schenken?

Herzlichen Gruß,
Dorli.

Die Apfelpreise ziehen an!
und allgemein rechnet man da=
von, daß die Apfel im Früh=
jahr teuer werden. Es sind
auch keine Auslandsäpfel auf
dem Markt. Minderwertige
Ware steht viel herum. Boden=
seeobst sticht sehr vorteilhaft
ab, hier auf dem Markt und
wird gerne gekauft.
Hier in der Zeitung werden
für Frankreich ein Dauermäd=
Mädchen für Haushalt und
Landwirtschaft gesucht. Ich
habe nun mal um nähere
Bedingungen geschrieben. Auch
habe ich neulich durch den
Radio eine Adresse aufge=

schreiben. Die Stelle in Baden-
Baden gibt in einer kanadi-
schen Zeitung Anzeigen auf.
Dort sind etliche Arbeiter,
scheinbar sehr gesucht. Meinst
Du, ich sollte mich mal dahin
wenden? Oder soll ich es lieber
bei Dir lassen? Am liebsten
ginge ich zuerst einmal
in die Schweiz. Hast Du schon
einen geeigneten Betriebsleiter
gefunden? Wie ist die UI?
Hast Du eigentlich vor, das
Pachtland oder der Groß-Ursel
zu kaufen?
Herzlichen Gruß. Darli
Fortsetzung
nächstes Blatt!
Dante Hoffmann
Hofheim/Taunus
\end{verbatim}
\end{frame}
\begin{frame}[fragile]{English Letter}
\begin{verbatim}
Hofheim, Nov. 22nd
Dear Mech.!
Many thanks for your
letter. When are you sending me
finally the pictures from the
Säntis? I've already been waiting a
half year for them. Can you
write me your Christmas wishes?
Preferably before the
1st Advent. Because then I'm doing
in Frankfurt big
shopping. What are you (pl.) giving
the various rela-
tives? Are Giesel. and Mr.
Geser already gone? Where did they go-

he
Aien?
already
uni =
d?
abe
ffen
as
time
I,
goes
yesterday
in the
ussions.
Hours in the Amerikahaus.
It was nice. But I
noticed that I many wor-
ds have forgotten. Tomorrow
evening I am going to a
singing circle, also for the 1st time.
Warm regards!
Dorle

Dear Mother!
How are you? I want to
quickly write you the apple prices
from the wholesale market in
Frankfurt. Cox apples
we have sold for 48 and 50 Pf per pound
to small retailers, and among
them are still wormy ones.
The large ones are, however, very
nice. Golden Pearmains are selling
slowly, 25 Pf, but some
are worse than A. There are also some
for 32 Pf. Boskoops only cost
20 to 25 Pf, but they are not
particularly nice. We are not selling
any Boskoops yet. In Frank-
furt in the shops, Cox apples are
sold for 70 Pf.

Liebe Mutter, vorhin bekam
ich Deinen Brief. Und dein
Urlaub ist es sehr bestimmt,
denn das Babettche, unsere Haus-
gehilfin geht am 1.12. weg. Da
ist für's Haus wieder wohl
niemand hier. Mal sehen, ob
es sich doch noch einrichten
läßt.
Hier rät mir jeder ab, Geld auf
die Sparkasse zu legen, weil
es sehr wahrscheinlich Inflation
wird.
Jetzt am 3.12. in Frankfurt
groß einkaufen. Ich brauche Unter-
wäsche, Schlafanzüge. Dann will
ich mir ein Kleid kaufen für
Sonntags. Ich denke, das kommt
billiger als machen lassen.
Dann brauche ich noch 1 bis 2
Arbeitshosen, eine Kittelschürze,
und eine gute Stoffjacke. Mit
dem Mantel bin ich sehr zufrie-
den. Ich stricke mir zur Zeit einen
Pullover, will mir aber doch einen

English Translation:

Dear Mother, just now I received
your letter. And your
vacation is very definite,
because Babettche, our house-
helper is leaving on Dec. 1st. So
then there will probably be no one
here for the house again. We'll see if
it can still be arranged
after all.
Here everyone advises me against putting money
in the savings bank, because
it's very likely inflation
will happen.
Now on Dec. 3rd in Frankfurt
big shopping. I need under-
wear, pajamas. Then I want
to buy myself a dress for
Sundays. I think that will be
cheaper than having it made.
Then I also need 1 to 2
work trousers, a smock apron,
and a good fabric jacket. With
the coat I am very satisfie-
d. I am currently knitting myself a
pullover, but I do want to get one

knitting. The things I get
in Frankfurt, however, are all
much cheaper than there with you.
That I always have
fun with books, you know. The
beige wool dress and your
handwoven belt is
too short for me, and also a
little too eng. Whether it fits Chris-
tel? And a white
polo shirt, which is still almost
new.

What should I then
give Grandma?

Warm regards,
Dorli.

Apple prices are rising!
and it is generally expected tha-
t, that apples in sprin-
g will become expensive. There are
also no foreign apples on
the market. Inferior
goods are lying around a lot. Lake Constance-
fruit stands out very favorably
, here on the market and
is eagerly bought.
Here in the newspaper are
for France a permanent gir-
l for household and
farm work sought. I
have now written for more detailed
conditions. Also
I recently through the
radio picked up an address

write. The office in Baden-
Baden places advertisements in a Canadi-
an newspaper.
There are several workers,
apparently highly sought after. Do you think
I should maybe there
apply? Or should I rather it
leave with you? Preferably
I would go first of all
to Switzerland. Have you already
a suitable operations manager
found? How is the UI?
Do you actually intend, the
leased land or the Groß-Ursel
to buy?
Warm regards. Darli
Continuation
next page!
Dante Hoffmann
Hofheim/Taunus
\end{verbatim}
\end{frame}
\begin{frame}{IMG_3762.jpeg}
\centering
\includegraphics[width=0.9\linewidth]{input\IMG_3762.jpeg}
\end{frame}
\begin{frame}{IMG_3763.jpeg}
\centering
\includegraphics[width=0.9\linewidth]{input\IMG_3763.jpeg}
\end{frame}
\begin{frame}{IMG_3764.jpeg}
\centering
\includegraphics[width=0.9\linewidth]{input\IMG_3764.jpeg}
\end{frame}
\begin{frame}{IMG_3765.jpeg}
\centering
\includegraphics[width=0.9\linewidth]{input\IMG_3765.jpeg}
\end{frame}
\begin{frame}{IMG_3766.jpeg}
\centering
\includegraphics[width=0.9\linewidth]{input\IMG_3766.jpeg}
\end{frame}
\begin{frame}{IMG_3767.jpeg}
\centering
\includegraphics[width=0.9\linewidth]{input\IMG_3767.jpeg}
\end{frame}
\begin{frame}{IMG_3768.jpeg}
\centering
\includegraphics[width=0.9\linewidth]{input\IMG_3768.jpeg}
\end{frame}
\begin{frame}{IMG_3769.jpeg}
\centering
\includegraphics[width=0.9\linewidth]{input\IMG_3769.jpeg}
\end{frame}
\end{document}