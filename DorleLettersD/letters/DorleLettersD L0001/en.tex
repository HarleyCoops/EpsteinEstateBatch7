\documentclass[12pt]{article}
\usepackage[utf8]{inputenc}
\usepackage[T1]{fontenc}
\usepackage[english]{babel}
\usepackage{geometry}
\geometry{margin=1in}
\usepackage{parskip}
\begin{document}
\section*{DorleLettersD L0001 (English)}
\noindent
Toronto, March 24th

Dear Mother,
I wish you all a very happy Easter and hope that you have warm weather. Will you be hiding eggs in the woods again? Or is everyone too grown-up for that now? It snowed quite a bit here again yesterday, everything was white again. I’m curious to see what it will be like next Sunday in any case, when our big Easter steamboat trip is. Mechi and I won't be leaving until very early Saturday morning, as Mechi has to work on Good Friday. Yesterday we celebrated Mechi's birthday with cake (mocha) and fruit salad. She will tell you herself what all she got. By the way, I found the key a few days later on Monday in the kitchen; Mother had apparently put the statement in a package, with the nice excuse that she wanted to look for the album. (Haha!)
On Thursday, school is over, thank God. No wonder you don't get anything done and can't make much sense of things and so on, with school every evening and then homework and typing at home. At Mezgerer's, I managed the practicing with ease, because with all the practicing and working, I gave my best for the group. Now we're singing in multiple parts again for birthdays or next Christmas. Now, here's what I'd like to wish for: some books, then the First Aid book that Mechi has, the book with the 2000 questions. And is there still that question-and-answer game we have? Or another one? Not one with "Do you love me?" "not in the least," but something geographical or historical or like that? And then I'd like the third volume of the Christmas stories that Liesl has, the one with Peter in it. That just occurred to me, you can perhaps save that for Christmas.
On Thursday I have a meeting with the Scout leaders, or rather, I'm supposed to definitely help out with the Girl Guides. Besides that, we also have to start up a group with 14-18 year-olds, which is difficult, because at that age many have left and aren't that interested anymore.
The day after tomorrow is our last performance at school. I will really miss the classes with Eggeling, our gym teacher. Her training is really such that those who have taken a year off, because of babies, are just as elastic and flexible as if they had always participated. What she accomplishes very well is good posture. We saw a before and after, and after half a year, most people were much taller, which also makes them look slimmer, even if they don't lose any weight at all. But the heavy ones really do lose weight; and the ones who are too thin put some on, sounds strange but it's actually true. I haven't lost or gained weight in the two years, although I sometimes go 3 times a week.
Horst wouldn't get the Remut-Ihr from Bökelern, who is in Germany right now, is bringing me a light meter, because between what he says he can do, it's the best one there is. Mr. Kochendörfer got it for me; he gets a 25\% discount, so it cost me 10.92 Swiss francs. I would now like to try developing the color films myself, which will be much cheaper. A lot of people here do it all themselves. Well, when Wechs is back home, you'll probably get to see a few pictures, because there weren't very many in August of last year. I only have the one of the vase, Wechs pasted the others in. I think if I go to the West in the summer, I'd rather go to Vancouver than Calgary, simply because there's more to do there. Firstly, evening classes, then concerts, and university, lectures, and theater. I wouldn't want to have to miss out on that, especially since you can get into many concerts and lectures for free here. And in the cities here there are so many parks, many more natural beauties and meadows where you can lie in the sun in shorts with a book after work. And on the weekend you get out of the city anyway. It's not like in Germany, where you have to feel sorry for the poor children in the city who have never seen a tree or a cow.
Recht expressed her concerns yesterday that you might find her very spoiled. What is still the same is that she regularly falls asleep during lectures or concerts, which is why she doesn't even go to school or to concerts anymore. I just can't get over it; she has a ticket for 12 symphony concerts and only goes once! What's the story with the coffee machine? Should I send the money back with someone? Or should I rather buy something with it that one of you needs? Could you write to me as soon as possible? I gave the machine to Inge for her birthday, it only takes 2 minutes for the coffee to be ready, and it tastes great!
Hopefully the situation with the trees isn't too bad. Here, in a case like that, the government would provide subsidies.
Once again, warm Easter greetings to everyone else,
Dork.
\end{document}
