\documentclass[12pt]{article}
\usepackage[utf8]{inputenc}
\usepackage[T1]{fontenc}
\usepackage[english]{babel}
\usepackage{geometry}
\geometry{margin=1in}
\usepackage{parskip}
\begin{document}
\section*{DorleLettersA L0005 (English)}
\noindent
Toronto, Nov. 6, '59

Dear Mother,

Now I finally have time to write to you. I was just about to leave for Toronto for my vacation when I received your letter. Since I moved 8 weeks ago, I got your letter a bit late, but that made me all the happier to receive it. I must also apologize for not having written for so long. I really should find enough time for it. However, for the last two to three months, I've been working a lot of overtime at the bank. I'm now doing the work of an assistant accountant there, which is probably something like a bookkeeper in Germany, I'm not exactly sure. I handle all the loans and mortgages and year-end financial statements. It's very interesting. Tomorrow, the bank is moving into a brand-new building, just two blocks away. There we'll also have a wonderful break room with nice furniture and a built-in hot plate, sink, and cabinets.

I have three weeks of vacation, and tomorrow morning is little Ursel Schnell's christening. She is such a darling child, but she was just vaccinated for smallpox, so she will very likely cry in church. Oh well, that will pass. She isn't shy with strangers at all, laughs a lot, and lets anyone hold her, which is typical for here, where people take their babies with them wherever they go—to the cottage, to parties and celebrations, or to friends' houses. The Schnells always bring their little bassinet, which has a good mattress in it. Ursel is 8 months old now and looks very healthy, although Franki Grenzen in Vancouver was much bigger and heavier at that age.

We just got back from a party. We had the whole old youth group together. It was at Sigmund and Ellen's, who bought a terrific house three weeks ago. The two of them have a darling little girl, Cornelia, 10 months old, and the next one is due in May. Then there was Karin and Christian von Rosen Thiel with their son Frank, 2 years old; Vera and Winrich Weller with their daughter Ruth, 9 months old; Werner Kjör and his wife Christel Wernuescher; then Joe von Schwerin with his fiancée Brigitte Achatz; Eckhart von Schwerin with his fiancée Dagi Damm, who are getting married in January; and Gisela Assenmacher. It was quite a commotion with the babies, and whenever one cried, all the moms would jump up and prick up their ears to figure out which one it was, and three would sit back down in their armchairs, satisfied and proud, while the fourth would dejectedly announce: "That's mine." But soon the adults were making more noise than the children; old memories were revived, old guest books and poems were read aloud—it was great. Well, you would have had fun too, I'm sure.

You asked what I would like for Christmas. I hope this letter arrives in time, and even if the package arrives after Christmas, I'll know it's my own fault.

I would really like a mountain sleeping bag, because I was going to buy one for myself now anyway. But there are two conditions attached. First, it must be able to be rolled up very small, and it can't weigh more than 4-5 pounds, and it would have to be warm enough for temperatures below zero and for sleeping in the snow. I would gladly pay some of the cost myself, but you know, when you have to carry the sleeping bag in addition to a tent, food, and skis and then burrow into the snow somewhere, these things are very important, especially the size. If that's not available, then I have another wish, namely Rieker climbing boots with the rubber climbing sole, not the felt-lined ones, but sturdy leather boots. Or else crampons, or an ice axe. Did I just dream this recently, or was I once given an ice axe, you know, the one I had hanging over my bed in the bedroom with the "Egyptian" walls? It was such a nice little one. If it wasn't branded and bent for a gift, or maybe Father's crampons would do. Does anyone use them? For smaller things, I'd like a Plischeke silhouette calendar, or the fabric gaiters that you pull over ski boots and pants, or any kind of craft booklets or instructions, or the new songbook "Der Turm." Unfortunately, I can no longer tell what you all want, as I've already sent the package.

By the way, linocutting is a lot of fun, and did I ever write to you how much everyone envies me for the small pair of binoculars? I take them with me on every ski tour or climbing trip, and they have often helped us find the way or the best route on a rock face, or spot ski tracks and examine glacial crevasses. It's so light, and everyone is amazed at how powerful the magnification is. The others from here are all ten times as big and heavy.

I wonder what Liesel is doing now? What is her job? I have no idea and no address. How is Ursel? Do you have any plans, including for the business? You can't possibly keep going like this without help. I would like to hear from you, quite honestly, if you fundamentally think or feel that we, who are gallivanting around here in America, should by rights be there helping. I often have a guilty conscience, and yet I do nothing about it. The whole family is probably upset about it, perhaps justifiably so. What is actually planned for the Bernbach house?

I would also like to take a few guitar lessons now to get past a certain point. The Canadian folk song group is a lot of fun. I now know many beautiful English, Canadian, and Australian songs.

Many warm greetings
and all the best,
Your Dala

P.S. If you haven't sent the sleeping bag, could you please write and let me know, because then I would go ahead and buy one quickly.
\end{document}
