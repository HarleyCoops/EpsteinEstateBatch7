\documentclass[12pt]{article}
\usepackage[utf8]{inputenc}
\usepackage[T1]{fontenc}
\usepackage[english]{babel}
\usepackage{geometry}
\geometry{margin=1in}
\usepackage{parskip}
\begin{document}
\section*{DorleLettersA L0002 (English)}
\noindent
Mt. Baker, Washington
Vancouver May 20, '59

Dear Mother and siblings!
I wish you a very happy Easter!
Do you also have spring weather like we have
here in Vancouver? Everything is already in bloom, although
today we are on Mt. Baker. Here we have 3-4
meters of snow. We are sitting in our cottage now,
reading and writing. Wilma, a nice Austrian woman,
is just telling us about Australia, about
sharks and kangaroos. Because of the hot summer
(summer there is in December, of course), the
snakes have multiplied so much that they are quite a plague.
In some areas of Australia, you always have
to check your bed in the evening to see if there are any
snakes in it. It can happen that they
crawl up to you at night; you find them
often. Well, I don't know if we would complain about that.
The entire ocean around Australia is full of
sharks. The beach is usually separated by a
fence underneath, which the sharks
cannot get through. Sometimes, however, during
a storm, one breaks through. In the surf,
you can see the sharks when they come,
and then a siren blows, and everyone runs
out of the water. Wilma has turned her sights from Australia to Germany and
inquired there whether she could work as a domestic
help. And of all places, she wrote to Konstanz.
She will go there next year. I will
give her your address then, and she will
surely visit you sometime. Or maybe you'll
need a domestic help then. In Australia,
in Melbourne, she has many German friends, also speaks
some German, and would like to get to know Germany and
its people sooner by living with a family.
Three other Australian women are already going to Europe this summer;
they might visit you too, hopefully you
don't mind. But I know how nice it is when
you go somewhere foreign and have an address
in your pocket. If I ever go to Australia,
I'll also have many addresses there where I
can go, and they will know me from pictures.
In England, too, I now have so many people I can
visit.
At the end of April, a few of us want to climb to the peak
of Mt. Baker. It is 10,800 feet high. We'll have to
spend one night up there below the summit in an ice
cave. The person leading it is a German; he
gave ski instruction for 12 years in the Wehrmacht.
Dear Mutti, when I received your letter,
I knew right away that this was it, without even
opening it. I was very happy about it.
I am always very happy when a letter comes from you.
I am sending you a few pictures. I hope they
are good. The one of the factory is not so good.
The factory is located in a residential area,
surrounded by a large park. The work is very
factory-like and mechanized.
One from our group is a
good amateur photographer; he
has now given me a lot of pictures.
Now I am making paper prints from
slides and, conversely, slides from
negatives. If only one had more
time.
We finally climbed Mt.
Baker last Sunday. The weather was
glorious and also very warm in the sun.
The last stretch below the
summit is not so easy, a rather
steep ice wall. At the top, we had to
cut steps with an ice axe. The view
from the summit (4000 meters high) was
one of a kind. On the descent,
we then slid down with the rope. I acted
as the guinea pig and let myself
fall, and the others had to
catch me. One of our group
is one of the best climbers in B.C.
He has made many first ascents,
also in the Alps.
Very warm greetings,
Greetings to Christel from Doph.
\end{document}
