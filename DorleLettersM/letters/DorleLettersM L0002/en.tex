\documentclass[12pt]{article}
\usepackage[utf8]{inputenc}
\usepackage[T1]{fontenc}
\usepackage[english]{babel}
\usepackage{geometry}
\geometry{margin=1in}
\usepackage{parskip}
\begin{document}
\section*{DorleLettersM L0002 (English)}
\noindent
This is my account of how I came to emigrate to Canada and, by a roundabout way, ended up in the youth group:

In the summer of 1951, I felt the urge to do something new. At the time, I was working at a fruit-growing business in Hofheim near Frankfurt. I was 20 years old and wanted to experience something. So I wrote to the consulates of England, France, and Canada with the intention of working abroad for a few years and learning languages. By trade, I was a state-certified fruit grower, a profession I had not chosen voluntarily. Instead, I was forced by the death of my father to leave the one-year school I was attending to help at home, where I was then offered a position at a nursery.
A letter arrived from the Canadian consulate with a visa offer, and after a visit to the consulate in Karlsruhe, I had my ticket and visa in hand. The official asked me where in Canada I was headed. I hadn't even thought about it, so he suggested Niagara Falls. I had heard of it in geography class somehow, so that was something concrete. Yes, that's where I wanted to go. The passage cost \$165.00, and the Canadian government loaned me the sum without my having to commit to anything. I had three weeks to say goodbye at home, buy and pack two large suitcases, and write my name, the name of my ship, and the departure date on the outside of the suitcases in black oil paint: M.S. Fairsea, October 29, 1951.

The train to Bremerhaven was an emigrant train that picked up emigrants in large cities, starting in Munich. It was a motley crew. In Bremerhaven, everyone was sent to a barracks camp, where we had to stay for 8 days. I don't know why we were there, maybe for delousing? or some other quarantine. Finally, it was time to board the ship. The M.S. Fairsea was bringing Canadian troops who had been in Germany back to Canada. The ship was accordingly primitive. I was in a dormitory with 99 other women and their small children. There were no portholes, as we were in the stern, two decks down. The ship's propeller was right next to us, making a terrible racket, especially when it came out of the water in high seas. There were about 400 women and children and 800 men on board. For the 400 women, there were 4 toilets, all of which stopped working after the first day. Overboard! A group of young people were allowed to sleep on deck, where we had seats available. For the 1200 passengers, about 300 of us slept on the floor around the smokestack, where it was warm and in the fresh air. Before departure, we were informed that we were only allowed to land in Canada with \$10.00. All our remaining money was confiscated, although we were allowed to go ashore one last time to buy something at the kiosks with the money. There was nothing useful to be had, so I bought a beret, which I never wore afterwards. I believe this confiscation of money was not ordered by the Canadian government, but by the ship's crew, as it sailed under the Panama flag and had an Italian crew. During the crossing, we had a heavy storm, and 90\% of my 99 dorm mates were seasick, with screaming children beside them. We couldn't land in Halifax either, so we sailed on to Quebec City. There, another special train was waiting, and after a two-day train ride, we arrived in Niagara Falls. \$10.00 in my pocket and no acquaintances, except for the five others from the ship who also got off here. We stayed in a cheap hotel for three days, as it was Remembrance Day weekend. With \$2.00 in my pocket, I went to the employment office. Several employers were already waiting there for us immigrants, cheap labor. The others couldn't speak any English at all, so I had to translate for them. In doing so, I realized I didn't know many of the simplest everyday words myself. What we had learned in school, reading Caesar and little nightingales, wasn't much help here. A female doctor snapped me up right away to work as a housekeeper. I was glad to have found work and lodging. The salary was \$45.00 a month. The doctor had his practice in the house, so from the very beginning, I had to answer the telephone when his wife and the children were alone. The wife helped me a lot with the language and gave me books to read. It took less than two months before I noticed that I was no longer translating into German when I read.

In the spring of 1952, I was drawn to the outdoors and found work in Beamsville at a large horticultural business that exported rose bushes all over the world, yes, even to Holland! Many immigrants worked there, most of whom had committed to two years as repayment for their passage. The women earned 35 cents an hour, the men 50 cents. The owners knew why they were bringing workers from Europe: the craftsmen. And so, that spring, construction began in Niagara-on-the-Lake. Motels and a theater, where the Shaw Festival would later take place, were built by us. My debt for the crossing was now paid off, and so I bought a car, a 1940 Ford, for which I paid \$500.00. But first I had to get my driver's license. An acquaintance took me to practice: back and forth, forwards and backwards, in first, second, and reverse gear. After a week, I was ready to take the driving test. An acquaintance drove with me to a neighboring town where the test was held. I had arranged it so that it was shortly before closing time. I had never driven on a main road before. It went well, including the parking, and I got my license. Now, however, I had to drive home alone in the dark, and you could say that was a slow drive. In the summer, I changed jobs and moved from Beamsville to St. Catharines. First, I worked in a canning factory, then I harvested fruit, and ended up in the kitchen. There I quickly learned all about diets. In 1953, I moved bag and baggage to Toronto and worked at Toronto General Hospital as an "Assistant Dietitian." I lived with my friend on Sherbourne Street, just a few houses from Dr. Goegginger's church. So it wasn't long before we ended up at a church service. Afterwards, we stood on the street, wondering what to do on a Sunday. Since I had a car, it didn't take long for me to have passengers for an outing. That was the beginning of a happy time in the youth group of the Goegginger congregation. We were all single young people, and together we were a family. In 1956, I moved to Vancouver, but the memories and the few years with the friends from the youth group will remain forever.
\end{document}
