\documentclass[12pt]{article}
\usepackage[utf8]{inputenc}
\usepackage[T1]{fontenc}
\usepackage[english]{babel}
\usepackage{geometry}
\geometry{margin=1in}
\usepackage{parskip}
\begin{document}
\section*{DorleLettersM L0003 (English)}
\noindent
Hofheim, February 27, 1950.

Dear Folks!
How are you all doing? Do you have
someone now? You could send us
Käthy's address sometime. It seems
she didn't write to us first after all.
Have you actually finished rating the
diary? Are you done with
pruning? And have you already
sown? We still have to send you 3 weeks' worth.
We are finishing up with
threshing. For raspberries, we have
the "Preussen" variety. When the
helper, the son, and I are on it
alone, we prune currants.
There are a lot of black ones, but
red ones too. In my opinion, the people here in Oestlau are not as
progressive as we are on Lake Constance,
although Mr. Betzel's is one of the most exemplary
farms. Summer pruning is
also done. But the pears, for example,
we haven't scraped, and they have been
scraped every year! With the currants,
much more of the older wood is left,
especially a lot of older wood. All
the young shoots are removed; at most,
if a branch is withered, it's
taken out. When I told him before
that we cut out the old wood,
Mr. Betzel said he had seen bushes
with branches as thick as an arm.
He said those would really bear fruit. Well, who
knows best! There is one small
sweet cherry tree, but otherwise, for cherries, there is a
\end{document}
