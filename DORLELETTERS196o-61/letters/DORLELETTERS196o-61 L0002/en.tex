\documentclass[12pt]{article}
\usepackage[utf8]{inputenc}
\usepackage[T1]{fontenc}
\usepackage[english]{babel}
\usepackage{geometry}
\geometry{margin=1in}
\usepackage{parskip}
\begin{document}
\section*{DORLELETTERS196o-61 L0002 (English)}
\noindent
Holland-America Line

R.M.S. Statendam
October 2/61

Dear Mother and Siblings,
Today is already Monday, and it’s the first day that one can at least somewhat write. We had hardly left Southampton when it started to get stormy, just as I had wished for. But then it went on with full force until yesterday evening, 4 full days. That is very rare for this time of year. The sky was mostly blue and the sun was shining, and it looked magnificent when the storm whipped up the water and blew it over the entire ship. We were in the outer bands of a hurricane. The 220-meter-long, 10-story house rocked quite a bit and groaned and creaked like an old woman. The entire bow dipped into the water and the spray shot over the whole ship. At the stern of the ship, the waves were supposed to be gone, much higher than the deck. Naturally, very many people were seasick; I was alone at our table on the first day. Many of the programs also had to be canceled, since you couldn't stand up straight. There were movies two or three times every day, though. The ship is much nicer than the other one I came on. The cabins are very large, with lots of closet space and drawers, and all have their own toilet and shower. I have two old women from Ulm with me in my cabin. There are very many Germans on the ship, at our table there are 5 Germans as well. We are over a day behind schedule because of the storm; hopefully it will still work out for me to get to work on Monday. One of my acquaintances on the ship is a baritone from the Metropolitan in New York. He is giving a concert tonight, we'll see how he can sing. Another is a young doctor from Germany who has a scholarship for the States. A girl, also German, also works at a bank in Boston and she worked briefly in Germany as a substitute for someone. But by and large, our impressions of Germany are the same.
You should all visit the Betzels sometime, you could probably hear and see many interesting things. The new farmstead is really great. It was part of that settlement plan, so they also had to fight hard against regulations, but Mr. Betzel doesn't let himself be intimidated so easily! When the building authority stipulated that he couldn't have basement windows, everything was nevertheless built in such a way that now—after the inspection—he can simply take out the few bricks. When support beams were required in the basement, they were made right away so that they could be easily moved or removed completely. So, the farmstead looks like this in the floor plan: [drawing of a layout with labels: Laundry Room, Shed, Garage, Tool Shed, Farmhouse]

The dotted part hasn't been built yet; that will be a hall with a drive-through, partly open. The farmhouse has a complete apartment upstairs and downstairs. Hermann and his wife and son (6 months old) live upstairs. Downstairs is the kitchen, dining room, living room, office, bedroom, bath, and toilet. [drawing of a floor plan with labels: Office, Living Room, Bedroom, Hall, Bath, Stairs] Upstairs are a living room, kitchen, 3 bedrooms, and a bath. The whole house has automatic oil heating and running hot water. In the basement, where the furnace is, there is a room for changing and drying work clothes. You go in directly from the courtyard downstairs. There is also an ironing room in the basement, a storage room, and the water pump, like the one you have in Stettlen, only stronger. They have their own well in the courtyard, 54 meters deep, and always enough water, also for watering the grounds. The storage area has a full basement. You either drive with
[Diagram of a building with labels: Driveway, Helper's Room, Warehouse, Sliding Door]
the car into the warehouse or unload from the outer ramp. Then they have two low flatbeds with two wheels on one end, where you then hook in two wheels with a drawbar on the other end to move them. When you unhook the lever, it stands very firmly and doesn't roll away. These are placed on the freight elevator, which, by the way, the farmer built himself from an old Ford motor. The basement below is completely laid with red brick tiles. There is neon lighting everywhere because you can sort better with it. Air shafts and ventilators provide for air circulation. Part of the basement will later be built as a cold storage facility, one section colder, one warmer. Opposite the warehouse is a low, long building, divided into four. One is a tool shed and workshop, one a machine shed (Unimog, Ferguson tractor, Volkswagen bus, Holden, Diesel O' Take), one a laundry room with a washing machine and bathtub for the helpers, and a garage (Hirschi's sports car and another old passenger car) of fruit bushes, the
\end{document}
