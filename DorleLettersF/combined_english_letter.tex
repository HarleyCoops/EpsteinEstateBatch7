```latex
\documentclass{article}
\usepackage[utf8]{inputenc}
\usepackage{geometry}
\geometry{a4paper, margin=1in}
\begin{document}
\subsection*{Source: E2B20E4B-6B4D-4F6B-803A-1CED8BE8DA6D_1_105_c.jpg}
Rocks. One notices that the area here is sparsely populated; one hardly finds "Private Property" signs. In the small towns there are mostly lumberjacks or sawmills, for example, a single one in Smooth Rock Falls and Geraldton. The lakes are full of logs. One still has to be satisfied with the roads; except for a few places one can still drive 50-55 miles/h. The weather is great, wind, clouds, sun. --- Now it is 10:30 AM, and it's time for me to let him drive again, because he seems a bit sulky already. The area around Lake Nipigon is lovely, the lake with the setting sun (that was a bit earlier) and many wild-growing flowers. To the right and left, high, steep rock faces in which caves seem to have their nests, the road finally not straight, but pleasantly winding and it goes up and down. --- In Pat--- we sleep in a side street in the car. July 2nd, clock set back by 1 hour. We have driven 1125 km so far, already completely through Germany (in terms of distance). The weather is wonderful again; the forest smells so dark. Behind Pat---, at a main lake, we had breakfast in our usual way. Well, it's not for Friedrich (apl.) --- just oatmeal, water, and sugar. --- From Pat--- to Kenora I drive again; it's not bad at all here by the lakes now. In Kenora, the un-

\subsection*{Source: EFB1C23B-A2F7-495A-A4AC-0AB9466AD61D_1_105_c.jpg}
It was a nice spot, until suddenly a large, wet, brown bear trotted by. That was too much for us, and we looked for another picnic spot. We then found spots everywhere, but it was nicer. There are three covered spots with 6 tables and a beautiful viewpoint, from where you can go hiking, etc. It's a good place to shelter well from the rain. And in the evenings, it's quite cool outside, not like here, where it's still 38$^{\circ}$C in the evenings. --- That evening, we also had a barbecue, and that was good too. It smells as if beef lightly at sun car company, no more. Now it also realizes that it's good if we don't buy bananas for EUR45 / 100, etc. July 8. Now it's gotten quite late, because I couldn't get up earlier, because I had such stomach pains all night like never before, it was great. It's raining a little, we're driving to Jasper today. At the Columbia Glacier, we wanted to consider whether we should drive up, but the weather wasn't good enough for us. The mountains are quite high, up to 3800 m with a lot of snow and glaciers. Krug

\subsection*{Source: 7DCCCE8A-F60C-4255-AE23-B48FB25D00AC_1_105_c.jpg}
Toronto, June 21, 55
189 Robert Street.

Dear much-esteemed Mrs. Hofmann,

You are certainly a little surprised that I am writing to you. You probably know me at most from stories as Dorle's friend --- and it is as such that I am writing to you.

Lately, Dorle has been very distressed because she feels that she is becoming more and more estranged from home. And since I understand this very, very well, even though she would certainly never say anything, I wanted to write to you. Do not lose your trust in Dorle; although she is a very reserved and difficult person, she is sincere, selfless, clean, and very capable. I would absolutely vouch for her.

I am always afraid that Hecki, who is still very young and away from home for the very first time, judges things too quickly,

\subsection*{Source: A95DBDCC-1B78-4C0D-8B15-D92FFCF48D00_1_105_c.jpg}
III
Driving into the last larger city in old Ontario's territory, securing [something] in the car and making basic preparations. Kenora is a nice town on Lake of the Woods, among the Indians (but truly). --- Immediately after Kenora is the border to Manitoba, the smallest state in the West. Here we live in a Tom hotel in Winnipeg, which costs almost nothing, 40c. Winnipeg does not have many sights; it is quite old (but not very old), has very rarely thick trams and a flat, monotonous surroundings. Since it is not late yet, we still drive another passage of the Prairie, the Prairie day. Far, far in the Prairie, there is an uncanny weather, for the sheet lightning illuminates everything. (We would also notice that the next day.)

July 3rd. At 6 AM we set off. This morning, first we wanted to [go to] Lake, but a storm had already raged quite a bit. First we were in puddles, but soon the entire road was in the water. It was a splendid area, something similar to a bird sanctuary, but Reef. It took all effort that we either didn't get stuck or didn't roll completely into the lake on the left. When we finally arrived at the village that lay on the beautiful Lake, was

\subsection*{Source: E7DE1DD1-2214-4B6A-BF6A-2374DAB5E71C_1_105_c.jpg}
III
We drove into the last major city in Eastern Ontario, secured the car, and settled in for the night. Kenora is a nice town on Lake of the Woods with many Indians (really). Right after Kenora is the border to Manitoba, the smallest province in the west. Here we found ourselves living in a gas station in Winnipeg again, which certainly didn't cost just 40c. Winnipeg doesn't have many sights; it's quite old (but not beautifully old), with many visible streetcars and a flat, monotonous surrounding area. Since it wasn't late yet, we drove another stretch of the Prairie, the gate to the Prairie. There, within the Prairie, there's a lake behind it. The weather shines, it brightens everything. (We should also remember that the next day.)

July 3rd. We set off at 6 AM this morning. First, we were supposed to bathe in the lake, but the initial plan was already partly shifted. First we were in puddles, then soon the whole road was covered with water. It was indeed a muddy area, so it was similar to a bird sanctuary, but this took all our effort so that we either didn't get stuck, or didn't roll completely into the lake on the left. But we finally arrived at what lay by the beautiful lake, of

\subsection*{Source: 85E98C07-1B1F-4C67-8DEB-E7A4FCD4425C_1_105_c.jpg}
In Hauptbach, it's that we've now made it to Jasper, however with almost no fuel in the tank, no oil in the engine, and with a half-broken spring --- a pleasant situation.

July 9, it's 5:30 in the morning, glorious weather, wait, but glorious, we are driving back to Dampf today. Right after Jasper we saw another bear, man, you always see them most in proximity to people. It's not that the journey is a 'Bulgerriß', because all the luggage is on the move. We are also driving up the back river this time, it is a glorious feeling. --- In Lake Louise we want another campsite, because it's full now and [we] hiked to Banff. It's only 5 o'clock, we thought, [and got] into a canoe and paddled up the Bow River for a bit. It's glorious here. The river is quite narrow, wild trees hang into the water and one can let oneself drift so beautifully underneath. I was sitting in a canoe, with a view. It is when one sits in the back and it also tips slightly at times, but it's nicer than anything else.

July 10 Today we only got up at 8 o'clock, that is the latest so far. In the icy Bow River we saw off the Canadians, they wanted to go back to Calgary, we meet Siegfried Frank, who today

\subsection*{Source: 8BEF7F5E-D9A8-49D9-A992-4D4A9D2DA189_1_105_c.jpg}
II
Past the grove, I'm tired again. --- Krug from Hearst drives Siegfried off again, I wake up and everything around me is forest, sparse forest, which looked bare, having been so devastated by a forest fire last year. I saw signs everywhere along the road like: "Use the ash tray, prevent fire, save the one." At a very large spot where only the uppermost branches on the trees were still green, it said: "this happened through carelessness." Between Hearst and Kinglac (135 miles) there is nothing marked on the map that would indicate a lake, and indeed, Hearst itself seemed forgotten, even for diving. But then, fortunately, in the middle of the forest, there was a "Transcanade Rodge" with a petrol station. In my joy about it, I wrote a poem:

The dolge was salvation in need,
for what for both of us is bread,
is for the "Cleß" (Chevrolet) the barrel,
and to the next [place] not much further.

Since I resumed driving as intended and wished to gather rest, I stopped shortly thereafter next to a beautiful lake for cooking and bathing, so that Rest could truly be alone. The lakes here in Northern Ontario are magnificent, forest, old logs pulled into the water by beavers and

\subsection*{Source: 2E16A3BF-099E-4C93-8417-6AF3A4FA1DB4_1_105_c.jpg}
IX
drive to the springs. We spend the day with them in Banff. At 5 o'clock we drive back together. We still arrive in Calgary at 11 PM. It is simply a wonderful city. In a small restaurant, where we drink a milkshake (ice cream with milk), we heard the "Polka of the Fisher Girl" (Fisherwoman from Lake Constance).

July 10. Liesl, Ditta and I went to the Stampede parade this morning. It was a large, colorful parade, and I was amazed that real characters (cowboy, Indian, etc.) were there, even in the smallest details. We even looked at the Stampede fairground. In the evening, starting at 9 PM, we then went to the cinema in Calgary, as it was already starting to rain, and we still wanted to make Toronto our destination, and Liesl absolutely wanted to drive off immediately after the cinema, so at 12 o'clock we completely say goodbye to beloved Calgary. Imagine, Ditta even gave me photos to take with me for enlargement as a farewell gift, because there is no German photographer in Calgary, and that won't be an opportunity there?

July 12. All night it rained intermittently, then from

\subsection*{Source: 320E1790-BA6A-47E7-A43F-75B0C1EBBC55_1_105_c.jpg}
Toronto, July 20, '55.

Dear Mother!

You are surely all waiting for a detailed report of my trip, well, now it's finally coming. Briefly in advance, it was great, I saw so much of Canada that I now have a completely different impression of it. But I want to tell everything nicely in order. If I get carried away again, it's because I have too much to tell and want to write everything at once.
So on Tuesday, June 20, I bought a Vito B, oh, you already knew that, then groceries for the first few days. At Braunmühle I still had to pick up photos, bathe, pack, eat etc., so we started punctually at 10 o'clock. I still got help making sandwiches, by the way, she did end up leaving with Inge, but she will tell you about that herself. --- Now I'm writing to you from my diary, because that's my daily bread, I wrote every day, often while driving, after cooking, before going to bed, or elsewhere.
June 30. 10:15 PM. Think how we are 53

\subsection*{Source: 0321F7E4-1475-4798-AE44-DFDD34BE88FA_1_105_c.jpg}
Regarding previous letters
XI. A rope excites. The road, though unpaved, is moderately in full, noble, yellow, and swiftly filling its entirety, at least that's how it seems to me. But extreme rarity makes all this quieter. Everywhere the greenest haze and Cochrems everywhere so powerfully made of yellow and forest, in which agitation stirs. Forest fire, I have learned, (by the way, it would occur to me if suddenly there were 138 forest fires in Ontario), pines with stuff, spruces with trunks --- finally, lakes; the forest mostly goes directly to the shore, and the lakes are many Cochrem up to Cobailt, the place where we drove on Friday morning, through the night, fog, and rain. When I arrive, I am dead tired; accordingly, I also sleep.
July 16. The 8th day. It has dawned, so beautiful that I almost no longer experience it in Toronto, where you still have the club open. In beautiful weather, we arrive in Toronto at 5 PM. It is perhaps good for someone, should God ever deem it so, for something contagious to happen at the club sometime, then we, the people, would have had the time.

\subsection*{Source: B01024A2-A42A-4EF2-8C52-5A00FE10F503_1_105_c.jpg}
XI
did not function properly anymore, he thoroughly cursed me out, saying I had driven it to ruin, but he has now learned from professional expertise that it could be from all the water we drove through. It was also perfectly fine again the next day.

July 13th. We were still sitting at Regina's yesterday evening, however, in a bad mood. The Jeep's clutch was repaired, but the brakes and the poor road conditions were not resolved. But all that should then be fine, so that we would rather manage the 135 miles without a workshop, because with the rain, the road there will also have been worse. --- Today we have already driven for 6 hours again, a hellish schedule that makes everything feel overwhelming. In Winnipeg we had the car inspected, had nothing overhauled. And after that, the steering starts to wobble terribly when one operates the steering. And since it costs approx. 40 dollars, he thinks that it's not much. Frankly, people are very willing (to pay), but I'm not keen on it. We are staying overnight in Kenora.

\subsection*{Source: C2163ED0-B58D-4DD3-9589-1625CA76AFA6_1_105_c.jpg}
which they cannot judge at all yet,
that Dorte will thus be given a brave haven.
Actually, it's really none of my business to interfere here, but I'm doing it just this once; but I feel sorry for Dorte, because she is very attached to you, dear Mrs. Hofmann, and to her sisters --- and what if she can't show it!
I don't want to keep you any longer now and will therefore conclude. You can be proud of Dorte; I don't know a single girl who has conducted herself so bravely and properly abroad alone --- like Dorte.
Please excuse this letter; it is not meant to upset you. I am very fond of Dorte and want to help her.
Sincerely,
I am your Inge Kordewitsch.

\subsection*{Source: CD2E4F82-8E50-42D2-91A7-85AF2EC309B7_1_105_c.jpg}
The whole path was covered with water, so we could not get to the lake. That was a disappointment, especially since the prairie was now beginning and the lakes were ending. I found myself transported into a very strange temple; things there were made that way, but they look much sleeker there.
Just now we reached the border to Saskatchewan, the roads are even more beautiful, simply great, I always drive 60 M/h (96 km/h), which is easy to do with these cars, and especially because the roads are dead straight and therefore you have a long view. In Manitoba we were able to see a lot of oil. We are now beginning to see the most beautiful train cars of the prairie, which stand at every smallest station. The weather is again a bit, a little fresher than yesterday; due to the storm many parts here are already partially flooded.
And we drive through Regina; it didn't seem to be a remarkable city; by midday we will arrive in Saskatoon. Saskatoon on the South Saskatchewan River is the most beautiful Canadian city I have seen so far. (But it changes later). The city lies to the left and right of the river, over which five beautiful, large bridges lead, which feature a lovely park with illuminated water features. I am sleeping here in the hotel, it was dark. For a lot of...

\subsection*{Source: A2C7CB5C-860B-4E4C-BB53-A59B03277170_1_105_c.jpg}
It was a magnificent spectacle of colors and wind on the water. The boat struggled a few times, only dragging its anchor, and without any lightning, we covered 35 miles of water. Despite that, it was a warm 35 degrees at night; we had the car open and were just lying in our pajamas in sleeping bags. You can perhaps imagine that we felt the need to stretch our legs at 5 AM, even though we had both slept well. So I slipped out of the sleeping bag, Inge was still in the back, and we drove on like that until just before Montreal. There we bathed, freshened up, and ate the rest of the potato salad. Then we went to Montreal. Right from the border, one noticed a difference. All traffic signs were dense, we didn't follow the traffic rules as strictly, and the people speak French and are bilingual, just as German and English are school languages. The people are not as polite as in Ontario; they are more direct. Montreal itself is beautiful, plus, completely different from Toronto; it gives a much more European impression. Churches are on every corner, very nice ones with beautiful altars, and above all St. Joseph's Oratory is magnificently designed. It's in the middle of the city and on a mountain. So I'll send you pictures as soon as I have time. Have I already written that we have a bottle-can opener? Warm regards, Doris.

\subsection*{Source: 393B98F9-3F19-4CC3-B8F7-9EBB4296A5E6_1_105_c.jpg}
As I woke up, they were standing in the ditch; the road was torn up, and it's always raining there. No consideration is given when driving, and since the narrow roads at the outer edge (150 cm) aren't good, you can tell it's raining. It was 5 AM, so we managed to get a yellow tractor, a tractor that barely pulled us. The tractor driver quite properly spoke his mind, because the speed limit on this tree-lined stretch was 25 mph and he definitely drove 50. Now I continue driving, because I don't believe he isn't tired. A road with construction sites began, because there was a lot of water on the road, more than I had ever seen, and it was raining in torrents. After 3 hours of driving, we were told again that the road was impassable. He was annoyed that I had driven so long and on such bad roads, but in this area, I simply have more experience. But after only half an hour, the next (fourth) construction site appeared, and the tractor drove at full road speed with me, involving about 3 hours of waiting at each one until the excavator had done enough with the ramp towing. I'll be glad if we make it to Regensburg today, and he noticed that the coupling [of the vehicle], which had already endured many overnight stays,

\subsection*{Source: 39B99908-E0DA-4EAD-A384-A173C547B02E_1_105_c.jpg}
VII.
Yes, a bath was taken at eleven, at approx. 38$^{\circ}$C. They also bathed, it is wonderful, but at night one is so exhausted that one would like to sleep in the bath; it is, by the way, the opposite of a means to lose weight.
July 2nd. The weather is hardly worse than this morning, but we are driving away to Lake Louise. (80 miles) The drive is wonderful, now for once a few high peaks are stuck in the clouds. Then at Lake Louise we park our car, it is quite cold, then we walk along the lake, through the forest and always higher up, sometimes over cliffs, sometimes over stones up to the Bing Glacier. The sight even in the ice field, I like that so much, do you still remember Bishop Ping? It is just somehow different on foot, one is much more into the mountains up there. I believe it also made a tremendous impression on almost everyone these days, regarding which he also didn't seem to have much respect, for he merrily scrambles up a crumbling slate wall with sandals, not entirely free from vertigo. In the small tea house we drank a cup of tea. After approx. 2 hours we arrive back at our car, and we immediately look for a campsite, in a small forest clearing not far from the Hotel Chateau Louise.

\subsection*{Source: D48A1148-50D3-4941-8AFB-B28193D2ACEF_1_105_c.jpg}
At Biel, deep in Sellaunn. I'm now glad that you reminded Mother about the spade, because it was badly needed for the dogs. In my diary I wrote: Must
1. When digging a path in the mud, put wood underneath, as it was spinning freely in the rut with the tires. First scold Sellaunn a bit so nothing splashes, then put stones under the wheels; that provides a firm surface. Meaning: I didn't even have a tow rope with me; that almost left me speechless. If I had asked before the trip if he had a rope, he surely would have said I shouldn't interfere in everything --- Well, the main thing is that we get out again, though ready for a bath. Even here you can only swim at a designated sandy beach; in other places, it's too marshy. As good as the wash was, there were so many creatures, because when we came out of the water, about 5-6 leeches were hanging on our legs, which were hard to get off. That really sent me to Biel, and I had already been looking forward to the lakes in Ontario.
Now we're heading, somewhat resigned (me, us?), towards Polen-hosen to the estate, to grease up and have a smoke.

\subsection*{Source: A661ADBF-10FC-4BB6-A7F1-0C4481091238_1_105_c.jpg}
IV
Mosquitoes. We are staying overnight again.
On July 4th, when we left Saskatoon this morning, it was quite cool; we are also already quite far north. Until the North Saskatchewan River, where we bathe and have breakfast, the weather is beautiful again. At the spot where I washed, there lay a dead calf, hide and all, ugh, darn it! We then set off at 8 o'clock; then in the mornings, it was always aching legs from sleeping, because the roads are bad here again, and most of the time I hit the worst spots. At 12 o'clock in Lloydminster, this wheel was again dealt with for the third time. It still looks intact; now it should last a long time, even though it's good. --- We are soon in Edmonton, so it was said. A friend lives there, with whom he already worked together in Germany. Since we haven't properly washed ourselves with truly fresh water for days now, we are now looking for a lake. The lakes indicated on the map were all in swampy areas. So we drive 23 miles before Edmonton into Elk Island Park. We just wanted to eat at a picnic spot, which was completely empty, but since it hadn't rained, and there was no talk of an early start, we'll sleep soon.

\subsection*{Source: 9D90CE16-2CF0-4368-9C4C-E50902BEC4A4_1_105_c.jpg}
The new address: 247 Franklin Ave. Toronto / Ont.

the journey is such that I cannot grasp it,
I now have a completely different impression of Canada,
which is very valuable. Everything was very beautiful for me, from the
purest [landscapes] over the mountains. I hope you can also
imagine it to some extent, especially if you know B.C.,
which once had those 'Komelhaften' things that I didn't get to experience,
hey, but not by airmail. Of all the tourist spots
some have become very beautiful, but it's too good
for Edel to have come to an end, I also personally feel from
That was a very serious journey, because with that, it is

if I start, it's better. I don't get something.
I'm doing well, now I'm quite homesick, for the
frommtine, for Hedwig's, serious beer, chocolate, cinema, everyone
is very keen on it. In Toronto it is
it's always 36$^{\circ}$C, otherwise 38-39$^{\circ}$C,
one can hardly do anything in the evenings, when it all comes down. Nothing in a good sense.
we need good space, especially
the roof, the heat makes
I want to properly heal by getting sleep
not always writing down what I forget

\subsection*{Source: EAC7478D-6E86-4E1A-A843-A2D73636A450_1_105_c.jpg}
V.
Already a lot. But since all gas stations and shops were closed, it unfortunately fell through.
Evenings in Edmonton. The preparations and the futile effort for the money gift had been in vain, because Siggi's friend went to Flin-Flon (Honeymoon) yesterday. Siggi is disappointed and in a bad mood, which I don't blame him for. We are wondering if the money and time will suffice until Vancouver. Since both are almost completely used up, it's too risky for us; furthermore, we wouldn't be able to stay anywhere then, and that's worthless. It's better to really see the area of the National Park between Calgary, Banff, and Jasper. So, tomorrow morning we're driving to Calgary, where Siggi has another friend. --- Edmonton is also located on the North Saskatchewan River, and although the surrounding area is partly flat, the city is built uphill on both sides of the river, which always looks nice. Edmonton is the fastest-growing city in Canada, and is the gateway to Northern Canada, which is why it unfortunately also attracts all sorts of undesirable people. These are largely still gold panners from Alaska who cannot work in winter, or other adventurers. Of course, the entire city doesn't consist of these.

\subsection*{Source: 9549FA40-F5EF-4F2A-8677-A665685A57AC_1_105_c.jpg}
July 14
So I drive off again with the rickety steering, on long, continuous stretches. It's pouring down, and comforts are once again very sparse, which is partly quite visible, because I haven't washed since Calgary. After breakfast, Hugh always likes to let me drive, and so I'm now driving once again towards other, bleak land and countless lakes. In the endless forests, raspberries and blueberries grow in abundant quantities, and there's no getting rid of them. Krug was with old people in the house where Hugh also spent the night. Too early for a fresh start, now after Port Arthur on Lake Superior (Great Lake). The area here and around Port Arthur is inherently passable all the way to Lake Superior; I drove 8 hours today, so I slept quite well again.

July 15
The sky is still cloudy, but at least it's not raining, so I can finally go and take a proper bath in wonderful Lake Superior now. I find the simple landscape, which on the way looked significantly more beautiful when the sun was shining. I drive to Longlac, where the gas station attendant

\subsection*{Source: EBB29E74-2580-442C-8D49-564D6D57E02A_1_105_c.jpg}
Spadima Rol is driving. Everything is fine except for:
leaking gasoline to be dealt with. (Riffprisch swings) Nozzle without
a trigger, not quite perfect cooking. The rest will
become apparent with time. Since I've already been running around quite a bit today,
I'll soon fall asleep, my head
on Siegf's lap, and when I wake up, they are refueling for the
first time in grey art; the gasoline here already costs
44 \textcent/gallon. I immediately fall back asleep, because my stomach feels
like a stone.
July 1st. It's only 5:15 AM and I'm already getting up.
(So now we've had dinner; I had potato
and bean salad, both with a lot of onion, and I had
made it, and for dessert, 1 peach, which I was given by the
hospital. To drink, there's only water.
(today, tomorrow, and in the evenings.) Now back to the di-
ary.
We are in Cobalt on Lake Timiskaming; we will
have a great breakfast (Pumpernickel, sausage, tomatoes,
onions, and mustard; I eat a lot of the latter) and then we're
on our way. The lake closely resembles Lake Constance. --- Now I may
drive; Siegf. is already feasting (damn it!), but
it's great driving, otherwise I probably wouldn't have
slept so well.
After 6\textonehalf{} hours of driving (375 km) on bad roads,
which, however, do not take the highway, in the morning

\subsection*{Source: 44320CA6-C525-414C-A481-21AA124CEA33_1_105_c.jpg}
VIII

Then they live in the middle of the street in bear caps, playing with their young boys. Photographing feeding is strictly forbidden, because the bears don't get enough and then smash the windows. As soon as you open a door or window, they come here.

Now I am sitting on a stone on the bank of the Alaska streams and rivers, anxiously waiting for an eagle shed. They look practical but also more impractical. A smaller one, for convenience, as close as possible to the riverbank for eating, after driving. He certainly feels he has started on it (he says!), but as if from the ground up, he succeeded in getting a lot. Although he had no stone with him and in front, the first one sank deeper and deeper. Since he is too lazy to attract anyone, he went there for a while. When it didn't attract on its own, when a car pulled up in reverse with 6 men sitting and he didn't arrive, nothing else remained but to wait for an eagle shed.

Meanwhile, they have now come around, thank God. He needs to attract many things. If he has nothing on him, he should dare a lot. And above all, if you already make such an impression, you must not be too cowardly, yes, too cowardly, more than he allows.

\subsection*{Source: 607D5F84-451B-481D-8939-A1C5410FCA42_1_105_c.jpg}
like a forest, but thankfully, where you can canoe, and most importantly, that in 2 hours by car you are in the middle of the mountains. What applies to Germany and Northern Germany is, in Canada, Manitoba and Saskatchewan \& (rarely just driven through once) and Southern Germany corresponds to Alberta and B.C.

Right behind Calgary you can already see the snow-covered mountains, I think I know what that felt like for me. The lakes and the foothills begin immediately too. At the park entrance 5 miles before Banff you are already surrounded by magnificent mountains. --- In Banff, we first visited the Museum of the Rockies and got some brochures about the area. (Here you can also rent mountain guides, isn't that something?) --- We had our lunches at a very secluded spot at Lake Minnewanka, which is just as lonely and quiet as in the Alps. It usually lies around on the ground towards the end of July, which for me as an experienced Rad. concerning the Er. Mad. is, is innate to him. --- In the evening, we took a long walk through the forest, uphill, where all the tree trunks are. Here is also one of the 2 hot sulfur springs, with approx. 44$^{\circ}$C, it smells like a chemistry lab and the rocks around them are yellow. Already below Banff

\subsection*{Source: F87B7B22-A7B4-40E8-9687-05C1786D169E_1_105_c.jpg}
Pensions, but already on Main Street we saw them, that I wouldn't have liked to spend the night there. By the way, Hugh's friend tells us later, many people go in the summer to northern Canada or Alaska to sawmills or gold panning operations. One can even go up privately and try their luck with it, though then 2/3 goes to the government. In winter they then come to Edmonton and usually spend all their money again, because after half a year of this monotonous life they want to live well again. But some also go up because of the lovely landscape, which is said to be unique. ---
July 5th. In pouring rain and with a bad weather forecast, I woke up with a stomachache. The drive to Calgary is monotonous; I slept. Then we first looked for the friend's apartment. The streets are divided by numbers and cardinal directions, so that it's easy to find your way; for example, the friend lives at 2126 25A Street SW. We'll then grease the car, and do an oil change. Calgary immediately makes a good impression, especially clean. One sees many well-maintained houses everywhere, and nice cowboys. Someone brings the car freshly washed out of the garage. But the rest is all secondary, because he loves his Clef more than anything else.

\subsection*{Source: 46805168-0938-4D87-BB88-36D8C1F36C95_1_105_c.jpg}
VI

July 6. The weather is glorious; it was lovely yesterday too, and I'm in excellent spirits. Last night, Highs., his friend Hinz, and his friend Dieter and I were in the city. They showed us the city, the city garden (zoo), and the distribution of Rockie Stone. Then we drove to a park, where there are royal colors and we swam. It was a nice change; moreover, the two are so nice, they are also proper men, which one cannot claim about Highs. The two, one 26, the other 27, live together in a nice basement apartment with a bedroom, living room, and kitchen combined. Each has his own car, a Ford and a Volkswagen; they are a carpenter and an electrician, and live nicely, a bit outside Calgary. It was completely natural that I slept at their place. In the mornings, I was woken up by music, which automatically starts 10 minutes before the alarm---that's the right thing for me! I haven't written in my diary for a long time. If the Hofmann family ever moves to Canada, only Calgary comes into question, because in all of Germany, it's the most unique city. The city (185,000) with the many beautiful single-story bungalows, the many rivers Bow and Elbowriver, the

\subsection*{Source: 3409BD22-9140-4174-9839-128D92F7DE32_1_105_c.jpg}
Dear Mother and Siblings
Today is our second day of vacation. We sleep a lot and eat at our leisure. We are already in Montreal, and travel tomorrow and in the evenings. At the market is the St. Joseph's Oratory in Montreal, it is wonderful, similar to Gardin, has unfortunately already healed 23749 cripples. mech cards so fast was.

\subsection*{Source: 49310721-7114-403B-B4D9-049668301F1E_1_105_c.jpg}
Trois-Rivières, Quebec, 5.7.55

Dear folks,
Just a quick note, before it gets dark. Now our second holiday day is already over, which means Wiland had to drive on Tuesday. Inge's old friend and I wanted to be back at Robben's home right away. She, or technical drawing, has something going on in Toronto. By 4 PM she was in bed. Then for so long, and give me time, that such a brewery gets [it]. Otherwise no city [visible] over the water. We then again for a while from Quebec. That was already new territory then up to the eastern end of Lake Ontario, the Thousand Islands, and large beautiful cities, and along the St. Lawrence River which is narrow there. The small ones near Kingston were, of course. But I drove to the shore onto a mowed meadow and first went swimming again in the river, as plans were underway. For a long time, he still wanted Chicago. We still had potato salad and sandwiches, then we played until we were surprised by a thunderstorm. It was very beautiful.
\end{document}
```