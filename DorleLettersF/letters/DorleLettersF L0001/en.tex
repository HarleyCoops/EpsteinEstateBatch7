\documentclass[12pt]{article}
\usepackage[utf8]{inputenc}
\usepackage[T1]{fontenc}
\usepackage[english]{babel}
\usepackage{geometry}
\geometry{margin=1in}
\usepackage{parskip}
\begin{document}
\section*{DorleLettersF L0001 (English)}
\noindent
Toronto, July 20, '55.

Dear Mother!

I'm sure you've all been waiting for a detailed account of my trip, and now, here it finally is. To put it briefly, it was magnificent. I saw so much of Canada that I have a completely different impression of it now. But I want to tell you everything in order. The reason I haven't written again is that I have too much to tell and want to write it all down at once.
So, on Thursday, June 30th, we chartered a Vita 3 with a driver, because—as you already know—we then had to get groceries for the first few days, I still had to pick up pictures from Braunmüller, bathe, pack, eat, etc., so we started promptly at 10 p.m. Inge helped me make sandwiches; by the way, she did end up going away with Inge, but she'll tell you about that herself. - Now I'm writing to you from my diary, which is just perfect, as I wrote in it every day—often while driving, after cooking, before going to sleep, or somewhere else.
June 30th, 10:15 p.m.: We left from 53 Spadina Rd, packed as we are. Everything is in order except for: an increasing smell of gasoline in the suitcase (Siegfried's, awful), a pistol without a marksman, not-quite-perfect suitcases... the rest will reveal itself in time. Since I've already run around quite a bit today, I'll fall asleep soon; my head on Siegfried's lap, and when I wake up, we are diving into a gray haze for the first time. The gas here already costs 44 c/gallon. I go right back to sleep, because my stomach feels like a stone.

July 1st. It's only 5:15 a.m. and I'm already up. So, now we've had dinner; I made potato and bean salad with lots of onion, and served both with Spam | peaches for dessert, which I got as a gift from the hospital. To drink, there's only Johnnie Walker, morning and evening. But now, back to the diary.
We are in Cobalt on Lake Timiskaming, where we will have a wonderful breakfast (pumpernickel, sausage, tomatoes, onion, and mustard, the last of which only I eat) and are in the best of spirits. The lake glistens like Lake Constance—now I get to watch, Siegfried is already snorkeling (yuck!), but he drove wonderfully, otherwise I wouldn't have slept so well.
After 6 1/2 hours of driving (375 km) on bad roads that still call themselves 'Highways,' past forests, I'm tired again. - Shortly before Hearst, Siegfried pulls off the road again. When I wake up, we are surrounded by forest, a sparse forest that, as I saw later, was devastated by a forest fire last year. All along the road there are signs like: "Use the ashtray, prevent fire," etc. At one very large spot, where the topmost branches on the trees were still green, a sign read: "this happened through carelessness." Between Hearst and Longlac (135 miles) there is nothing on the map that would suggest a village, and Siegfried indeed seemed to have forgotten to get gas in Hearst. But then, by great luck, there was a "Transcanada Lodge" with a gas station in the middle of the woods. In my joy, I wrote a poem:

The lodge was our rescue in our need,
for what for us both is bread
is for the "Chev" (Chevrolet) the gasoline
and we wouldn't have made it much longer.

Since I drove again after Geraldton and Siegfried was sleeping, I pulled over shortly after next to a beautiful lake to cook and swim, so that Siegfried could get some proper sleep. The lakes here in Northern Ontario are magnificent—forest, old logs dragged into the water by beavers, and rocks. You can tell that the area is barely populated; you hardly ever see a "Private Property" sign. In the small towns, there are mostly wood factories or sawmills, for example a huge one in Smooth Rock Falls and Geraldton. The lakes are full of logs. You have to be content with the roads; apart from a few spots, you can still drive 50-55 miles/h.
The weather is magnificent—wind, clouds, sun. - It's 10:15 p.m. now, and it's a good thing I'm letting him drive again, because he seems to be a bit grumpy. The area here at Lake Nipigon is magnificent, the lake with the setting sun (that was a bit earlier) and the countless trees, with high, steep rock faces to the right and left in which caves seem to have their nests. The road is finally not straight, but wonderfully winding, and it goes up and down. In Port Arthur we sleep in the car on a side street.
July 2nd. Set the clock back by 1 hour. We have driven 1125 km so far, completely without incident and through it all. The weather is wonderful again; the forest smells different than at home. After Port Arthur, we had breakfast outdoors at a small lake. At first it was a bit stiff; for breakfast we had: oatmeal, water, and sugar. - From Port Arthur to Kenora I'm driving again, it's no trouble, the lakes just don't end. In Kenora, the last larger town in Northern Ontario, we stop to have the car greased and the oil changed. Kenora is a nice town on the Lake of the Woods with many Indians (but decent ones). Right after Kenora is the border to Manitoba, the smallest province in the west. Here, we stopped at a gas station in Winnipeg, where gas only costs 40c again. Winnipeg doesn't have many sights; it's quite old (but not beautifully old), has lots of awful streetcars, and flat, monotonous surroundings. Since it wasn't late yet, we drove on to Portage la Prairie, the gateway to the prairie. There, in the middle of the prairie, it is endlessly flat and green; the distant lightning says it all. (We were to find that out again the next day.)
July 3rd. We left at 6 o'clock this morning. At first, we wanted to swim in the lake, but the storm had raged quite a bit during the night. At first there were only puddles, but soon the whole road was covered with water. It was a magnificent area, like a bird sanctuary, but Siegfried had his work cut out for him to keep us from either getting stuck or sliding off into the lake on the right or left. But when we finally reached the village on the beautiful lake, the entire yard was underwater, so we couldn't get to the lake. That was a disappointment, since this was where the prairie began and the lakes ended. I was tempted to wash myself in a very muddy puddle, but I came out considerably dirtier. - We just got to the border of Saskatchewan. The roads are even better, just fantastic. I'm always driving 60 m/h (96 km/h), which is easy to do in these cars, and especially because the roads are straight as an arrow, giving you a long view. In Manitoba there were quite a few oil wells; here the grain elevators begin, the trademarks of the prairie, which stand at every tiny train station. The weather is very hot again, a bit cooler than yesterday; because of the storm, many fields here are also quite flooded. - We just drive through Regina, it doesn't seem to be a special city. In the afternoon we arrive in Saskatoon. Saskatoon, on the South Saskatchewan River, is the most beautiful Canadian city I've seen so far. (But that is yet to change.) The city lies on the left and right of the river, crossed by five beautiful, large bridges. In a magnificent park with illuminated fountains stands a castle-like hotel; it was glorious. (But tons of mosquitoes.) We spend the night in the car again.
July 4th. When we left Saskatoon this morning, it was quite cool; we are already pretty far north. By the time we reach the North Saskatchewan River, where we swim and have breakfast, the weather is lovely again. At the spot where I washed, there was a dead calf with its hide and everything, yuck! I started driving at 8 o'clock, because Siegfried always has rheumatism in his legs in the morning from sleeping. The roads are bad here again, and I usually get the worst stretches. At 12 o'clock in Lloydminster, Siegfried took the wheel again; he usually likes to take over when I drive for so long, even though it's going well. - We'll soon be in Edmonton, where Siegfried's friend lives, with whom he worked in Germany. Since we hadn't washed with really fresh water for days, or rather, without washing ourselves at all, we are now desperately looking for a lake. The lakes marked on the map were all in swampland. So we drive 25 miles from Edmonton into Elk Island Park. We were just about to eat at a picnic spot that was completely empty, but since it had rained heavily during the night, and Siegfried, being so clever from sleeping well, drove right into the mud, which was in some places quite deep. I'm just glad, Mother, that you reminded me about the spade, because we really needed it here. In my diary I wrote:
1. When getting stuck in the mud, never put wood underneath, as it's incredibly slippery and makes the tires run hot. First, shovel some of the mud away so nothing splashes up, then put stones under the wheels to provide a solid base. Siegfried didn't even have a tow rope with him, which left me speechless for a moment. If I had asked before the trip if he had a rope, he would have surely told me not to meddle in everything. - Well, the main thing is that we got out again, though we were ready for a bath. Here, too, you can only swim at a designated sandy beach; the other spots are too swampy. That also meant laundry duty. What bad luck it was, because when we came out of the water, there were 5-6 leeches on our legs, which are very hard to get off. That really gave me the shivers, and I had been so looking forward to the lakes in Ontario.
Now we're heading back (like us?) to Edmonton - we're going to rest, get the car greased, and buy a wedding present. But since all the gas stations were closed, both plans fell through.
Evening in the car in Edmonton. The preparations and the futile effort to find a wedding present were all for nothing, because Siggi's friend left on his honeymoon yesterday. Siggi is in a very bad mood, which I can't blame him for. Whether the money and time will last until Vancouver is too risky for us, down to the last penny. Besides, we wouldn't be able to stop anywhere, and that's not worth it. It's better this way, to really see the area of the national parks between Calgary, Banff, and Jasper properly. So tomorrow morning we'll drive to Calgary, where Siggi also has a friend. - Edmonton is also on the North Saskatchewan River, and although the surroundings are also quite flat, the city is built up the slopes on the left and right of the river, which always looks nice. Edmonton is the fastest-growing city in Canada; it's the gateway to Canada's north and therefore unfortunately also attracts all the worst sorts of people. These are largely old gold prospectors from Alaska who can't work in the winter, or other adventurers. Of course, the whole city doesn't consist of these people, but we saw them right on the main street, so I wouldn't have liked to spend the night there. By the way, Siegfried's friend told us later that many people go to northern Canada or Alaska in the summer to work in sawmills or gold mines. You can even go up there privately and try your luck, although 2/3 of it then goes to the government. In the winter, they come to Edmonton and usually spend all their money again, because after half a year of that monotonous life, they want to live well for a change. But some also go up there because of the magnificent landscape, which is said to be unique. -
July 5th. I woke up at half past 10 in the morning to pouring rain and a bad weather forecast. The drive to Calgary is monotonous; I slept. Then we first looked for the friend's apartment. The streets are divided by numbers and cardinal directions, so it's easy to find your way around. For example, the friend lives at 2126 25A St. SW. We then had the car greased, washed, and the oil changed. Calgary makes a good impression right away, above all it's clean. There are villa districts, well-bred cattle, and nice cowboys. - The car is just coming back from the garage, freshly washed. Siegfried is in a great mood because of it; he loves his Chev more than anything.

July 6th. The weather is glorious; by the way, it was nice yesterday too, and I'm in the best of spirits. Last night, Siegfried, his friend Riez, and his friend Dieter and I were in town. They showed us the city, the public gardens (zoo), and the petrified wood from the Rockies. Then we drove to a park where we went on the swings and a ride. It was a nice change of pace. Besides, the two of them are so nice; they are also well-rounded men, which you can't say about Siegfried. The two of them, one 26, the other 27, live together in a nice basement apartment with a bedroom, living room, and kitchen. Each has his own car, a Ford and a Volkswagen. They are a carpenter and an electrician, and live nicely, about half an hour from Calgary. It was a matter of course that we slept at their place. In the morning, I was woken up by music that automatically starts 10 minutes before the alarm, which is just the right thing for me! - What I wrote in my diary there: If the Hofmann family ever moves to Canada, only Calgary would be an option, because it most resembles Germany with the Alps. The city (185,000) with its many beautiful single-story bungalows, the two rivers, the Bow and Elbow, the forest-like park where you can camp, and most importantly, that you can be in the middle of the mountains in 2 hours by car. What Northern Germany is to Germany, Manitoba and Saskatchewan are to Canada (I've driven through them now), and Southern Germany corresponds to Alberta and B.C.
Right after Calgary you can already see the snow-covered mountains. I think I know what a feeling that was for me. The foothills and the mountains begin right away. At the park entrance, 5 miles before Banff, you are already surrounded by magnificent mountains. - In Banff, we first went to the mineral museum and the animal exhibit and picked up some brochures about the area. By the way, you can also hire mountain guides here (but that's not for us?). - We cooked our lunch at a very secluded spot on Lake Minnewanka; here it is just as lonely and quiet as in the Alps. Siegfried also wanted to fish in places that I thought were suitable, but since he is a city slicker, that's just not in his nature. - In the evening, we took a long walk through the forest, always uphill, away from all the tourists. One of the two hot sulfur springs is here, with a temperature of about 44°C. It stinks like rotten eggs and the rocks also have a yellow shimmer. Up above Banff a bath has been made from it with a temperature of about 38°C. We went for a dip too; it's wonderful, but afterwards you're so exhausted that you just want to sleep. By the way, it's the best way to lose weight.
July 7th. The weather looks a bit worse this morning, but we're heading to Lake Louise anyway (80 miles). The drive is magnificent, even if a few high peaks are hidden in the clouds. But then at Louise, we park our car. It's quite cold. Then we walk along the lake, through the forest, always higher up, then over again, across rocks, all the way to the Victoria Glacier. The whole thing is called the Plain of Six Glaciers. I love that sort of thing, remember from the Birkenkopf? It was still something different, though; you're much more directly up in the mountains. I think the air, too. The day made a powerful impression on him, even if he doesn't have much respect for its dangers, because he cheerfully scrambles up a brittle shale wall in sandals, not entirely free from vertigo. We had a cup of tea in the small teahouse.
After 6 to 7 hours, we get back to our car and immediately looked for a place to cook, in a small forest clearing not far from the Chateau Lake Louise hotel. It was a nice spot, until suddenly a large, brown bear trotted past at a very close distance. That was a bit too much for us, so we looked for another nice picnic spot. We had avoided these public places until now, but it's so nice there. There are open-air sites with 6 tables and a fireplace, where you can cook something, etc. Above all, you can stop there even when it's raining. To be honest, it's also quite cool outside there, not like here, where it's still 38°C in the evening. - They often commented on me in that dress, not that it doesn't fit, but that Siegfried's dress shirt would easily get ruined in a breakdown, no longer. Now he also sees that it's good if I don't buy bananas for 45c/lb, etc.
July 8th. Well, it's gotten quite late, but I couldn't get up earlier because I had the worst stomach ache all night, like never before, it was terrible. It's raining a little; we're driving to Jasper today. We would rather skip the Columbia Icefield, though we considered driving up, but the weather wasn't good enough for us. The mountains are quite high, up to 3800 m, with a lot of snow and glaciers. Shortly after, we see a mother bear playing with her little cub right in the middle of the road and take pictures. Feeding is strictly forbidden because then the bears don't get enough and smash the windows. As soon as you have a door or window open, they come in.
Now I'm sitting on a rock on the bank of the Athabasca River and have been waiting for a tow truck for a long time. The owner of the perfect but impractical eight-seater full of creature comforts drove as close as possible to the riverbank to eat. The main thing is that we're now heading into Jasper, albeit with almost no gas in the tank, no oil in the engine, and a half-broken spring. A comfortable situation.
July 9th. It's 5:30 in the morning, magnificent weather, cold but glorious. We're driving back to Banff today. Right after Jasper we saw another bear; they know they're most likely to find food near people. The drive is a delight, as all the peaks are clear. This time we also drive up the Icefield Parkway; it's a magnificent feeling. - In Lake Louise we stop again at the picnic area. Since it's only 5 o'clock, we make a pot of coffee and paddle a little way up the Bow River. It's wonderful here. The river is quite narrow, wild trees hang into the water, and you can just drift along so nicely under them. I do know what a canoe looks like; it's pointed at the front and back and therefore tips quite easily, but it's more beautiful than anything else.
July 10th. Today we didn't get up until 8 o'clock, the latest so far. We were supposed to meet Hans-Peter, whose parents gave him permission, but since they were heading straight back to Calgary, we met Siegfried and Frank instead, who were both driving to the springs. We spend the rest of the day with them in Banff, and at 5 o'clock we drive back together. We sleep in Calgary one last time and drive through the night; it's simply a magnificent night. In a small restaurant where we drink a milkshake (ice cream with milk), we heard the "Polka of the Fishermaid" (Fisherwoman from Lake Constance).
July 11th. Siegfried, Dieter, and I were at the Stampede parade this morning. It's a large, colorful parade—the same as all parades, except that here they have western cavalry (cowboys), Indians, etc. with them. Then we look at the Stampede fairgrounds. In the evening I meet a few nice girls and then we all went from Calgary to the cinema, because it was already starting to rain just as we wanted to get the tourists to bed. Siegfried absolutely wants to leave right after the cinema, so at 12 o'clock at night, with a heavy heart, we say goodbye to beloved Calgary. Quiet and wet. As a farewell gift, Dieter gave me negatives to enlarge, because they don't have German photo sizes in Calgary. Wasn't that sweet of him?
July 12th. It rained almost the whole night, because when I wake up, we're in a ditch. The road is soft, and it's still raining. Siegfried didn't even pull over, because the paved roads aren't asphalted on the outer edge (50 cm), you can't see it in the rain. It was 5 o'clock in the morning, so we had a good chance of getting a tow truck in the daylight to pull us out. The tow truck owner gave Siegfried a proper piece of his mind, because the speed limit on this construction stretch was 25, and he was definitely driving 50. Now we're driving behind a truck; I don't think he'll do that again either. A road with construction sites, ditches, gravel, and clay began, with lots of water on the roads because it was still standing in the ditches. Besides, it was pouring rain. After 3 hours, Siegfried is driving again. The road is good, but he's annoyed that I drove off the bad road for so long, but I just have much more experience in this area. But after just half an hour, the next construction site comes up, we'll see. The convoy goes as far as the next town. Because of it, every truck has to wait 3 hours until the construction vehicles are done with their work. I'll be glad if we even make it to the next town. When Siegfried took over for the morning, he noticed that the clutch wasn't working properly anymore. He yelled at me quite harshly that I had driven it into the ground, but as I've now learned from expert sources, it can happen from all the water we drove through. It was perfectly fine again the next day.
July 13th: We slept last night somewhere between Swift Current and Regina, though in a bad mood. The clutch had to be repaired; the cable couldn't handle the bad roads. We decided it would be better to make the 135 miles without a workshop, because with the rain, the road there would have gotten worse too. - Today we've already been driving for 6 hours again, in the heaviest prairie rain that makes everything dark. In Brandon, in the rain, we had the car greased, then an overhaul. Afterwards, the steering started to rattle very badly; you could hardly hold the steering wheel. Since it costs about 40 dollars, he thinks it will last until Toronto. He told me I should tell Mom that I'm not comfortable with it. We spend the night in Kenora.
July 14th. So I set off again with the broken steering and the completely worn-out speedometer. It's pouring rain and the morning sun is once again very sparse, which you can see quite clearly, because I haven't washed since Calgary. After breakfast, Siegfried always likes to let me drive, and so now I'm driving through Northern Ontario again, that land of countless lakes, where raspberries and blueberries grow in endless forests in vast quantities, and they couldn't taste any better. He went to pick up mail, which was being held for Siegfried at a farm for the time being, as he is going to Newmarket too early, because the post office is near the Great Lake. The area here and around Port Arthur is rocky right up to the huge lake. I drove for 8 hours today; it didn't feel like it went very well.

July 15th. The sky is still overcast, but at least it's not raining, so I finally get a proper bath in a wide, empty lake. The journey continues tomorrow. I'm driving through the simple landscape, which looked significantly more beautiful on the way here when the sun was shining. I drive to Longlac, where the road goes uphill a bit. The road isn't paved, but it's not as dusty as before. Everything goes much faster than during the day, at least it seems that way to me. The fresh air counters the old grogginess and the lethargy of my limbs and mind. Forests where the last great forest fire did its work (by the way, there were 138 forest fires at once in Ontario), rocks with mountains, lakes with logs and little islands, water lilies. The forest usually goes right up to the shore, after the water lilies. After Cochrane, we drove another 240 km until 1 a.m. (Standard Time) to Cobalt, the place we left from on Friday morning. The night was foggy and rainy. When I arrive, I am dead tired, and feel correspondingly bad.
July 16th. The last day has dawned. As nice as it is, I'm looking forward to Toronto again, the club, and the people. In beautiful weather, we arrive in Toronto at 5 p.m. It might be something for later, to see something more western, then we would have had the time - the people. My address: 247 Franklin Ave, Toronto / Ont.

The trip is over. I can't quite process it. I have a completely different impression of Canada now, which is worth a lot. It was all very special, but the mountains were the nicest. I hope you can all imagine it to some extent, especially when you see the slides or maybe the camel hair blanket (?) that I'm bringing you, though not by airmail. Some of the color photos also turned out nicely, but it's too late now for you to see them; I'll show them to you in person. We'll see about the trip, whether it works out. I'll write to you about that in the next letter. It's better if I start. You won't get something.
I'm doing well, only I'm quite homesick for the pious ones, for Johannes, the berry harvest, chocolate, going to the cinema, etc. - it's terribly hot here. In the shade it's great, as I said, it's always 36°C, sometimes for hours 38-39°C. It's awful, you can hardly do anything in the evening when you come into the room. In the whole house, by the way. September 1st, because we need the tennis court, especially with no roof below. But the heat makes you tired. Otherwise, I want to help diligently. I shouldn't always put it off, because I forget.
\end{document}
