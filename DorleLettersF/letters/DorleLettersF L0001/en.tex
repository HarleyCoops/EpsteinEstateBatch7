\documentclass[12pt]{article}
\usepackage[utf8]{inputenc}
\usepackage[T1]{fontenc}
\usepackage[english]{babel}
\usepackage{geometry}
\geometry{margin=1in}
\usepackage{parskip}
\begin{document}
\section*{DorleLettersF L0001 (English)}
\noindent
Toronto, July 20th, '55.

Dear Mother!

I'm sure you've all been waiting for a detailed account of my trip, and now, here it is at last. To put it briefly, it was magnificent. I saw so much of Canada that I have a completely different impression of it now. But I want to tell you everything in order. The reason I haven't written again is that I have too much to tell and want to write it all at once.
So, on Thursday, June 30th, we chartered a car, the Vita 3, because, as you already know, then groceries for the first few days, I still had to pick up pictures from Braunmüller, bathe, pack, eat, etc., so we started punctually at 10 p.m. Inge helped me make sandwiches; by the way, she did end up going away with Inge, but she'll tell you about that herself. - Now I'm writing to you from my diary, which is just great; I wrote every day, often while driving, after cooking, before going to sleep, or somewhere else.
June 30th, 10:15 p.m.: We left from 53 Spadina Rd, packed as we are. Everything is in order except for: an increasing smell of gasoline in the suitcase (Siegfried's, awful), a pistol without a shooter, not-quite-perfect suitcases... the rest will reveal itself over time. Since I've already run around quite a bit today, I'll fall asleep soon; my head on Siegf.'s lap, and when I wake up, we are diving into a gray haze for the first time. The gasoline here already costs 44 c/gallon. I go right back to sleep, because my stomach feels like a stone.

July 1st. It's only 5:15 a.m. and I'm already up. So now we've had dinner; I made potato and bean salad with lots of onion and served both with canned ham | peaches for dessert, which I got as a gift from the hospital. To drink, there's only Johnnie Walker, morning and evening. But now, back to the diary.
We are in Cobalt on Lake Timiskaming, where we will have a wonderful breakfast (pumpernickel, sausage, tomatoes, onion, and mustard, the latter of which only I eat) and are in high spirits. The lake glistens like Lake Constance - now I get to watch, Siegf. is already snorkeling (yuck!), but he drove very well, otherwise I wouldn't have slept so soundly.
After 6 1/2 hours of driving (375 km) on bad roads, which nevertheless call themselves 'Highways', past forests in the morning, I'm tired again. - Shortly before Hearst, Siegfried pulls off the road again. When I wake up, there is nothing but forest around us, a sparse forest that, as I saw later, was devastated by a forest fire last year. All along the road there are signs like: "Use the ashtray, prevent fire," etc. At one very large spot, where the topmost branches on the trees were still green, it said: "this happened through carelessness." Between Hearst and Longlac (135 miles) there's nothing on the map that would suggest a village, and Siegfried really seemed to have forgotten to fill up in Hearst. But then, by great luck, there was a "Transcanada Lodge" with a gas station in the middle of the woods. In my joy, I wrote a poem:

The Lodge was our salvation in our need,
for what for us both is bread
is for the "Chev" (Chevrolet) the gasoline
and we wouldn't have made it much longer.

Since I drove again after Geraldton and Siegfried was sleeping, I pulled over shortly after next to a beautiful lake to cook and swim, so Siegfried could get some proper sleep. The lakes here in Northern Ontario are magnificent: forest, old logs dragged into the water by beavers, and rocks. You can tell that the area is barely populated; you hardly ever see a "Private Property" sign. In the small towns, there are mostly wood factories or sawmills, for example a huge one in Smooth Rock Falls and Geraldton. The lakes are full of logs. You have to be content with the roads; apart from a few spots, you can still drive 50-55 miles/h.
The weather is magnificent: wind, clouds, sun. - Now it's 10:15 p.m., and it's a good thing I'm letting him drive again, because he seems to be a bit grumpy. The area here at Lake Nipigon is magnificent, the lake with the setting sun (that was a bit earlier) and the countless trees, with high, steep rock faces on the right and left in which caves seem to have their nests, the road finally not straight for once, but wonderfully winding, and it goes up and down. In Port Arthur, we sleep in the car on a side street.
July 2nd. Set the clock back 1 hour. We have driven 1125 km so far, completely without incident and through it all. The weather is wonderful again, the forest smells different than at home. After Port Arthur, at a small lake, we had breakfast in the open. It was a bit stiff at first. For breakfast we had: oatmeal, water, and sugar. - From Port Arthur to Kenora I'm driving again, it's no trouble, the lakes just don't stop. In Kenora, the last larger town in Northern Ontario, we had the car greased and the oil changed. Kenora is a nice town on the Lake of the Woods with many Indians (but decent ones). Right after Kenora is the border to Manitoba, the smallest province in the west. Here we stopped at a gas station in Winnipeg, where it only cost 40c again. Winnipeg doesn't have many sights, it's quite old (but not beautifully old), with lots of awful streetcars and flat, monotonous surroundings. Since it wasn't late yet, we also saw Portage la Prairie, the gateway to the prairie. There, in the middle of the prairie, it is an endlessly flat, green expanse, where the light tells the whole story. (We were to notice that again the next day.)
July 3rd. We left at 6 o'clock this morning. At first we wanted to swim in the lake, but a storm had raged quite a bit during the night. At first there were only puddles, but soon the whole road was covered with water. It was a magnificent area, like a bird sanctuary, but Siegfried had his work cut out for him to keep us from either getting stuck or sliding right or left completely into the lake. But when we finally reached the village on the beautiful lake, the entire yard was under water, so we couldn't get to the lake. That was a disappointment, since this was where the prairie began and the lakes ended. I was tempted to wash myself in a very muddy puddle, but came out considerably dirtier. - We just drove onto the highway into Saskatchewan. The roads are even nicer, just great. I'm always driving 60 m/h (96 km/h), which you can easily do in these cars, and especially because the roads are dead straight and you therefore have a long view. In Manitoba there were quite a lot of oil wells; here the grain elevators are starting, the landmarks of the prairie, which stand at every tiny train station. The weather is very hot again, a bit cooler than yesterday; because of the storm, many places here are also quite flooded. - We're just driving through Regina, it doesn't seem to be a special city. In the afternoon we arrive in Saskatoon. Saskatoon on the South Saskatchewan River is the most beautiful Canadian city I've seen so far. (But that's about to change.) The city lies on the left and right of the river, crossed by five beautiful, large bridges. In a magnificent park with illuminated fountains stands a castle-like hotel; it was wonderful. (But tons of mosquitoes.) We're spending the night here again.
July 4th. When we left Saskatoon this morning, it was quite cool; we are already pretty far north. By the time we reach the North Saskatchewan River, where we swim and have breakfast, the weather is lovely again. At the spot where I washed myself, there was a dead calf with its hide and everything, yuck! At 8 o'clock I started driving, because Siegfried always has rheumatism in his legs in the morning from sleeping. The roads are bad again here; I usually get the worst stretches. At 12 o'clock in Lloydminster, Siegfried took the wheel again. He usually likes to take it when I've been driving for so long, although it's going well. - We'll soon be in Edmonton, where Siegfried's friend lives, with whom he worked in Germany. Since we hadn't washed ourselves with really fresh water for days, we are now desperately looking for a lake. The lakes marked on the map were all in swampland. So we drive 25 miles from Edmonton to Elk Island Park. We were just about to eat at a picnic spot that was completely empty, but since it had rained heavily during the night, and Siegfried is so good at getting stuck, we were soon deep in the mud. I'm just glad, Mother, that you reminded me about the spade, because we really needed it here. I wrote in my diary:
1. When getting stuck in the mud, never put wood underneath, as it's incredibly slippery and the tires get hot. First, shovel some of the mud away so nothing splatters, then put stones under the wheels to provide a solid surface. Siegfried didn't even have a tow rope with him, which left me speechless for a moment. If I had asked before the trip if he had a rope, he would have surely told me not to meddle in everything. - Well, the main thing is that we got out again, although we were ready for a bath. Here, too, you can only swim at a designated sandy beach; in other places it's too swampy. That meant work with the laundry, too. It was such bad luck, because when we got out of the water, there were 5-6 leeches hanging on our legs, which are very hard to get off. That really gave me the shivers, and I was already not looking forward to the lakes in Ontario.
Now it's on to Edmonton again (what else?) - we're going there to rest, get the car greased, and buy a wedding gift. But since all the gas stations were closed, both plans fell through.
Evening in the car in Edmonton. The preparations and the futile effort to find a wedding gift were all for nothing, because Siggi's friend left on his honeymoon yesterday. Siggi is in a very good mood, which I don't hold against him. Whether the money and time will last until Vancouver is too risky for us, besides, we wouldn't be able to stop anywhere, and that's not worth it. It's nicer this way, to really see the area of the national parks between Calgary, Banff, and Jasper. So tomorrow morning we're driving to Calgary, where Siggi also has a friend. - Edmonton is also on the North Saskatchewan River, and although the surroundings are also quite flat, the city is built up the slopes on the left and right of the river, which always looks nice. Edmonton is the fastest-growing city in Canada; it's the gateway to Canada's north and therefore unfortunately also attracts all the worst kinds of people. These are largely old gold miners from Alaska who can't work in the winter, or other adventurers. Of course, the whole city doesn't consist of these people, but we saw them right on the main street, so that I wouldn't have liked to spend the night there. By the way, Siegfried's friend told us later that many people go to northern Canada or Alaska in the summer to work in sawmills or gold mines. You can even go up there privately and try your luck, though 2/3 of it then goes to the government. In the winter, they come back to Edmonton and usually spend all their money again, because after half a year of that monotonous life, they want to live well for a change. But some also go up there because of the magnificent landscape, which is said to be unique. -
July 5th. I woke up at half past 10 in the morning to pouring rain and a bad weather forecast. The drive to Calgary is monotonous; I slept. Then we first looked for the friend's apartment. The streets are divided by numbers and cardinal directions, so it's easy to find your way around; for example, the friend lives at 2126 25A St. SW. We then had the car greased, washed, and the oil changed. Calgary makes a good impression right away, above all, it's clean. There are villa districts, well-bred cattle, and nice cowboys. - The car is just coming back from the garage, freshly washed. Siegfried is therefore in a great mood, because he loves his Chev more than anything else.

July 6th. The weather is absolutely glorious, by the way yesterday too, it was beautiful, and I'm in the best of moods. Last night, Siegfried, his friend Riez, and his friend Dieter and I were in the city. They showed us the city, the public gardens (zoo), and the petrified wood from the Rockies. Then we drove to a park where we rode the carousel and swung on the swings. It was a nice change of pace. Besides, the two of them are so nice; they are also well-rounded men, which you can't say about Siegfried. The two of them, one 26, the other 27, live together in a nice basement apartment with a bedroom, living room, and kitchen. Each has his own car, a Ford and a Volkswagen. They are a carpenter and an electrician, and they live nicely, about half an hour from Calgary. It was a matter of course that we slept at their place. In the morning, I was woken up by music that automatically starts 10 minutes before the alarm, that's just the thing for me! - I also wrote this in my diary: If the Hofmann family ever moves to Canada, only Calgary would be an option, because it most resembles Germany with the Alps. The city (185,000) with its many beautiful single-story bungalows, the two rivers, Bow and Elbow, the forest-like park where you can camp, and most importantly, that you can be in the middle of the mountains in 2 hours by car. What Northern Germany is to Germany, Manitoba and Saskatchewan are to Canada (I've driven through them once), and Southern Germany corresponds to Alberta and B.C.
Right after Calgary you can already see the snow-capped mountains. I think I know what a feeling that was for me. The foothills and the mountains begin right away. At the park entrance, 5 miles before Banff, you are already surrounded by magnificent mountains. - In Banff, we first went to the mineral museum, the animal exhibit, and got some brochures about the area. By the way, you can also hire mountain guides here (but that's not for us?). - We cooked our lunch at a very secluded spot on Lake Minnewanka; here it is just as lonely and quiet as in the Alps. Siegfried also wanted to fish in places that seemed suitable to me, but since he is a member of the M.-d., that's out of the question for him. - In the evening, we took a long walk through the forest, always uphill, away from all the tourists. One of the 2 hot sulfur springs is also here, with a temperature of about 44°C. It stinks like rotten eggs and the rocks also have a yellow shimmer. Up above Banff, a bath has been made from it with a temperature of about 38°C. We went for a swim too, it's wonderful, but afterwards you're so exhausted that you just want to sleep. By the way, it's the best way to lose weight.
July 7th. The weather looks a bit worse this morning, but we're driving to Lake Louise anyway (80 miles). The drive is magnificent, even if a few high peaks are hidden in the clouds. At Lake Louise we park our car, it's quite cold, then we walk along the lake, through the forest, always higher up, then over again, over rocks to the glaciers. The whole thing is called a "plain of six glaciers," I like that so much, do you remember from the Birkenkopf thing? It was still a bit different, you are much more directly up in the mountains. I think the air too. The days have made a powerful impression on him, even if he doesn't have much respect for its dangers, because he cheerfully scrambles up a brittle slate wall in sandals, not entirely free from giddiness. We had a cup of tea in the small teahouse. After 6 to 7 hours we get back to our car, and we immediately looked for a place to cook, in a small forest clearing not far from the Chateau Lake Louise hotel. It was a nice spot, until suddenly a large, brown bear trotted past at a very low height. That was a bit too much for us and we looked for another nice picnic spot. We have avoided these public places so far, but it's so nice there. There are spots without roofs, with 6 tables and a fire pit, where you can cook something, etc. Above all, you can stop there even when it's raining. Honestly, it's also quite cool outside, not like here where it's still 38°C in the evening. - They often criticized me for my dress, not that it doesn't fit, but that Siegfried's dress pants easily get ruined in a breakdown. Now he also sees that it's good if I don't buy bananas for 45c/lb, etc.
July 8th. Well, it's gotten quite late, but I couldn't get up earlier because I had such a stomach ache all night like never before, it was terrible. It's raining a little, we're driving to Jasper today. We'd rather skip the Columbia Icefield, whether we should drive up, but the weather wasn't good enough for us. The mountains are quite high, up to 3800 m with a lot of snow and glaciers. Shortly after, right in the middle of the road, we see a mother bear playing with her little cub and take pictures. Feeding is strictly forbidden, because then the bears don't get enough and break the windows. As soon as you have a door or window open, they come in.
Now I'm sitting on a rock on the bank of the Athabasca River and have been waiting for a tow truck for a long time. Because Siegfried, the owner of the perfect but impractical eight-seater full-comfort vehicle, wanted to get as close as possible to the riverbank for dinner. The main thing is that we're heading into Jasper now, albeit with almost no gas in the tank, no oil in the engine, and with a half-broken spring, a cozy situation.
July 9th. It's 5:30 in the morning, wonderful weather, cold but wonderful, we're driving back to Banff today. Right after Jasper we saw another bear; they know that they are always most likely to be found near people. The drive is a pleasure, because all the peaks are clear. This time we also drive up the Icefield, it's a magnificent feeling. - In Lake Louise we stop again at the picnic spot, and since it's only 5 o'clock, we make ourselves a pot of coffee and paddle a little way up the Bow River, it's wonderful here. The river is quite narrow, wild trees hang into the water and you can drift so nicely under them. I know what a canoe looks like, it's pointed at the front and back and therefore tips quite easily, but it's more beautiful than anything else.
July 10th. Today we didn't get up until 8 o'clock, that's the latest so far. We get permission for Hans-Peter from his parents. Just as they want to go back to Calgary, we meet Siegfried and Franke, who are both driving to the springs. We spend the rest of the day with them in Banff and at 5 o'clock we drive back together. We sleep in Calgary one last time and drive through the night, it's simply a wonderful night. In a small restaurant, where we drink a milkshake (ice cream with milk), we heard the "Polka of the Fishermaid" (Fisherwoman from Lake Constance).
July 11th. Siegfried, Dieter, and I were at the Stampede parade this morning. It's a large, colorful parade, the same as all parades, except that here there's western cavalry (cowboys), Indians, etc. Then we look at the Stampede fairgrounds. In the evening I meet a few nice girls and then we all went from Calgary to the cinema, because it was already starting to rain when we wanted to get the tourists to bed. Siegfried absolutely wants to leave right after the cinema, so at 12 o'clock at night we say goodbye to our beloved Calgary with a heavy heart. Quiet and wet. As a farewell gift, Dieter gave me negatives to enlarge, because in Calgary they don't have German photo sizes. Wasn't that clever of him?
July 12th. It rained almost the whole night, because when I wake up, we are in the ditch, the road is softened, and it's still raining. Siegfried hadn't pulled over far enough, because the paved roads aren't asphalted on the outer edge (50 cm), you can see that in the rain. It was 5 o'clock in the morning, so we had daylight to get a tow truck to pull us out. The tow truck owner gave Siegfried a proper piece of his mind, because the speed limit on this construction stretch was 25, and he was definitely driving 50. Now we're driving behind a truck; I don't think he's going any faster either. A road with construction sites, ditches, gravel, and mud began, because there was a lot of water on the roads from what was still in the ditch, and besides, it was pouring rain. After 3 hours Siegfried is driving again, the road is good, only he's annoyed that I drove off the bad road for so long, but I just have much more practice in this area. But after only half an hour comes the next construction site, we'll see. The welcome sign goes all the way to Halle. Because of him, and every truck has to wait 3 hours until the construction vehicles are done with their work. I'll be glad if we even make it to Prague. When Siegfried took over the wheel for tomorrow, he noticed that the clutch wasn't working properly anymore. He yelled at me quite harshly that I had ruined it, but as I have now learned from expert circles, it can happen from all the water we drove through. It was perfectly fine again the next day.
July 13th: We slept in Medicine Hat on the way to Regina last night, though in a bad mood; the clutch had to be repaired, the cable couldn't handle the bad roads. We decided it would be better to make the 135 miles without a workshop, because with the rain the road there would have gotten worse too. - Today we've already been driving for 6 hours, in the heaviest prairie rain that makes everything dark. In Swift Current, in the rain, we had the car greased, then an overhaul. Afterwards, the steering started to rattle very badly, you could hardly hold the steering wheel. Since it costs about 40 dollars, he thinks it will last until Toronto. He said I should write to Mom that I'm not comfortable with it. We're spending the night in Brandon.
July 14th. So I'm driving off again with the broken steering and the completely worn-out speedometer. It's pouring rain and the morning sun is once again very sparse, which you can see quite clearly, because I haven't washed myself since Calgary. After breakfast Siegfried always likes to let me drive, and so I'm now driving through Northern Ontario again, that land of countless lakes, where raspberries and blueberries grow in endless forests in huge quantities, and they couldn't taste any better. King was there to pick up mail, staying at the farm for Siegfried for the time, since he's going to Newmarket too early, because the post office is also very Superior (Great Lake). The area here and around Port Arthur is rocky right up to the huge lake. I drove for 8 hours today, it didn't feel like it went very well.

July 15th. The sky is still overcast, but at least it's not raining, so I finally get a proper bath in a wide, empty lake. The journey continues tomorrow. I'm driving through the simple landscape, which looked significantly more beautiful on the way here when the sun was shining. I'm driving to Longlac, where the Tomahawkler Lake is. The road is not paved, but not as dusty as before, everything goes much faster than during the day, at least it seems so to me. The scent of valerian makes the old grumpiness and the weariness of the limbs and pigs less powerful. Forests where the last great forest fire did its work (by the way, like Julie said, there were 138 forest fires in Ontario at once), rocks with mountains, lakes with logs - little islands, water lilies, the forest usually goes right up to the shore, after the water lilies. After Cochrane we drove another 240 km until 1 o'clock in the morning (S.T.) to Cobalt, the place where we started on Friday morning. The night, fog and rain, when I arrive I'm dead tired, and feel correspondingly bad.
July 16th. The last day has dawned. As nice as it is, I'm looking forward to Toronto again, the club, and the people. In beautiful weather, we arrive in Toronto at 5 o'clock in the afternoon. It might be for later, maybe we can see something more western now, then we would have had the time - the people.
My address: 247 Franklin Ave, Toronto / Ont.

The trip is over, I can't quite assess it, I have a completely different impression of Canada now, which is worth a lot. It was all very much for itself, but the mountains were the nicest. I hope you can all imagine it somewhat, especially when you see the B.O. or the camel hair (?) I'm bringing you, hey, but not by airmail. Some of the slides turned out nicely too, but it's too late now for you to see them, I'll show them to you in person. We'll see about the drive, whether it works out. I'll write to you about that in the next letter. It's better if I start. You're not getting something.
I'm doing well, only I'm quite homesick for the devout, for Johannes, the berry harvest, chocolate, going to the movies, etc. - it's terribly stiff here, it's crazy in the area, as I said at night, it's always 36°C, sometimes even 38-39°C for hours, it's crazy, you can hardly do anything in the evening when you come into the room. By the way, in the whole house. Sept. at 1, because we need it for tennis, especially with no roof downstairs. But the heat is making me tired. Otherwise, I want to help diligently. I shouldn't always postpone it, because I forget.
\end{document}
