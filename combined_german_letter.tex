\documentclass{letter}
\usepackage[utf8]{inputenc} % For proper handling of various characters
\usepackage[german]{babel}    % For German hyphenation and typographic rules
\usepackage{ragged2e}     % For \RaggedRight

% Optional: Add sender and recipient details
% \name{Absender Name}
% \address{Absender Straße \\\\ Absender Stadt, PLZ Ort}
% \signature{Absender Name}

\begin{document}

% \begin{letter}{Empfänger Name \\\\ Empfänger Straße \\\\ Empfänger Stadt, PLZ Ort}
\begin{letter}{Empfänger Name/Adresse Platzhalter}

\opening{Sehr geehrte/r Empfänger/in,} % Or a more appropriate salutation in German, e.g., Liebe/r [Name],

\RaggedRight % To prevent overfull hboxes with long words or hyphenated text

% --- Content from german_output/IMG_3762_german.txt ---
Hofheim, den 22.11.
Liebes Mech.!
Herzlichen Dank für Deinen
Brief. Wann schickst Du mir
endlich die Bilder vom
Säntis? Ich warte schon ein
halbes Jahr darauf. Kannst Du
mir Eure Weihnachtswünsche
schreiben? Möglichst vor dem
1. Advent. Da mache ich in
Frankfurt nämlich große
Einkäufe. Was schenkt Ihr denn
den verschiedenen Verwand-
ten? Sind Giesel. und Herr
Geser schon weg? wohin ge-

% --- Content from german_output/IMG_3763_german.txt ---
hen sie eigentlich? Ist der
meine Betriebsleiter schon hier?
Was macht Ulla? Bis wann
bleibt sie noch? Hat sie schon
eine Stelle?
War es am 11. 11. im Semi-
nar schön? Hast Du schon
eine Adresse aus England?
Heute ist Feiertag. Ich habe
den ganzen Vormittag Platten
gespielt. Das wär\' so was
für Dich. Ich stricke zur Zeit
einen Pullover für mich,
weiß. Mit meinem Mantel
bin ich sehr zufrieden. Gestern
war ich zum 1. Male in
einer englischen Diskussions-

% --- Content from german_output/IMG_3764_german.txt ---
er
Aien?
schon
uni =
d?
abe
ffen
as
zeit
ich,
lautet
gestern
im
ussions.
Stunden im Amerikahaus.
Es war schön. Ich habe aber
gemerkt, daß ich viele Wör-
ter verlernt habe. Morgen
abend gehe ich in einen
Singkreis, auch zum 1. Mal.
Herzlichen Gruß!
Dorle

% --- Content from german_output/IMG_3765_german.txt ---
Liebe Mutter!
Wie geht es Dir? Ich will Dir
mal kurz die Apfelpreise
auf dem Großmarkt in
Frankfurt schreiben. Coxe haben
wir für 48 und 50 Pf pro Pfund
an Kleinhändler verkauft, da-
bei sind noch würmige drin.
Die große ist allerdings sehr
schön. Goldparmänen gehen
schleppend, 25 Pf, aber zum
Teil schlechter als A. Es gibt auch
für 32 Pf. Boskoops kosten nur
20 bis 25 Pf, sind allerdings nicht
so besonders schön. Wir verkaufen
jetzt noch keine Boskoops. In Frank-
furt im Laden werden die Coxe
mit 70 Pf verkauft.

% --- Content from german_output/IMG_3766_german.txt ---
Liebe Mutter, vorhin bekam
ich Deinen Brief. Und dein
Urlaub ist es sehr bestimmt,
denn das Babettche, unsere Haus-
gehilfin geht am 1.12. weg. Da
ist für\'s Haus wieder wohl
niemand hier. Mal sehen, ob
es sich doch noch einrichten
läßt.
Hier rät mir jeder ab, Geld auf
die Sparkasse zu legen, weil
es sehr wahrscheinlich Inflation
wird.
Jetzt am 3.12. in Frankfurt
groß einkaufen. Ich brauche Unter-
wäsche, Schlafanzüge. Dann will
ich mir ein Kleid kaufen für
Sonntags. Ich denke, das kommt
billiger als machen lassen.
Dann brauche ich noch 1 bis 2
Arbeitshosen, eine Kittelschürze,
und eine gute Stoffjacke. Mit
dem Mantel bin ich sehr zufrie-
den. Ich stricke mir zur Zeit einen
Pullover, will mir aber doch einen

% --- Content from german_output/IMG_3767_german.txt ---
stricken. Die Sachen bekommt
ich in Frankfurt aber alle
viel billiger als dort bei Euch.
Daß ich an Büchern immer
Spaß habe, weißt Du ja. Das
beige Wollkleid und dein
handgewebten Gürtel ist
mir zu kurz, und auch ein
wenig zu eng. Ob es der Chris-
tel paßt? Und eine weiße
Polobluse, die noch beinahe
neu ist.

Was soll ich denn der
Großmama schenken?

Herzlichen Gruß,
Dorli.

% --- Content from german_output/IMG_3768_german.txt ---
Die Apfelpreise ziehen an!
und allgemein rechnet man da=
von, daß die Apfel im Früh=
jahr teuer werden. Es sind
auch keine Auslandsäpfel auf
dem Markt. Minderwertige
Ware steht viel herum. Boden=
seeobst sticht sehr vorteilhaft
ab, hier auf dem Markt und
wird gerne gekauft.
Hier in der Zeitung werden
für Frankreich ein Dauermäd=
Mädchen für Haushalt und
Landwirtschaft gesucht. Ich
habe nun mal um nähere
Bedingungen geschrieben. Auch
habe ich neulich durch den
Radio eine Adresse aufge=

% --- Content from german_output/IMG_3769_german.txt ---
schreiben. Die Stelle in Baden-
Baden gibt in einer kanadi-
schen Zeitung Anzeigen auf.
Dort sind etliche Arbeiter,
scheinbar sehr gesucht. Meinst
Du, ich sollte mich mal dahin
wenden? Oder soll ich es lieber
bei Dir lassen? Am liebsten
ginge ich zuerst einmal
in die Schweiz. Hast Du schon
einen geeigneten Betriebsleiter
gefunden? Wie ist die UI?
Hast Du eigentlich vor, das
Pachtland oder der Groß-Ursel
zu kaufen?
Herzlichen Gruß. Darli
Fortsetzung
nächstes Blatt!
Dante Hoffmann
Hofheim/Taunus

\closing{Mit freundlichen Grüßen,} % Or a more appropriate closing in German

\end{letter}
\end{document} 