\documentclass[12pt]{article}
\usepackage[utf8]{inputenc}
\usepackage[T1]{fontenc}
\usepackage[english]{babel}
\usepackage{geometry}
\geometry{margin=1in}
\usepackage{parskip}
\begin{document}
\section*{DORLELETTERS1965 L0001 (English)}
\noindent
Dillmann

Banff, Jan. 4/65

Dear Mother! Your letter and a package from you arrived today, New Year's Eve. I thought the other package might have been sent on the same day, but I suppose it will have to wait. The package with the film: It's a lovely surprise that you had to get them a second time. Yes, it's not good, but since you have so many, it doesn't matter that you're sending it again. But it was so cold, about -25° Celsius, and windy too, so we weren't outside much. We didn't go to the torchlight run either, but we could see the torches from here. Your hands get cold, and especially your face. Eric's nose got frostbitten right away. It's the coldest we've had so far. We also had kindergarten at the lodge for the first time. The children were with complete strangers, and because Jan ran so excitedly to the toys, he wasn't so happy there when I later saw him playing with the other children. We can take him for a week for 10 weeks, as long as it's this cold. He was completely captivated when he saw all the toys. One of them was a stuffed squirrel that we haven't seen since.

I want to quickly tell you about the time before January 1st. The children are all so excited that they don't want to sleep anymore. He no longer naps after lunch and gives me quite a reproachful look when I put him to bed. He cries sometimes, too, but then I go to my room and he talks to himself for half an hour and then falls asleep. He talks about wanting to stay outside for a long time, but then his feet get cold after all, which was a lesson for him. He talks about all sorts of things, and when he recites the whole family from memory, then everyone is gone, and in his language, we are "all good," and he said that the whole time.

A few days ago, the temperature dropped noticeably. Suddenly it became night in the middle of broad daylight, because everything was freezing, everything turning into little ice balls. Usually, it comes within a few hours after an Almarok, and so it did this time. André rolls around in the deep snow; he crouches on all fours as if he were an animal. He crawls on all fours through the powder snow and his face is usually unprotected from the cold, so tears often run down his cheeks. We want to grill a fat catch soon. He often laughs, gets angry, and scolds. André says: "André, big brother." He and Christopher are very close. Christopher called out and tattles when André is by the water, or when Christopher helps with drying the dishes, he watches his favorite show. They build together. When he's building, he's so skilled that he makes fantastic structures, dreams of pirates that he folds over the animals. At every meal, he later gives them something to drink, sleeps in bed on his stomach.

The raccoons often scratch at the floor of our house. Andri now thinks it's Tobby the bear, who is snowed in under the house until spring. He's already looking forward to spring when the snow melts, because then he can play with Tobby. Father has often seen Rolly and Polly, the raccoons; he never wants to feed them, because I don't know, then they'll become too pushy. From the Christmas trees out there that are for the animals to eat in the spring. They are hanging on a nail in the living room. Andri has suddenly become very interested in letters; he sits in front of the typewriter and names the letters he already knows. The day after tomorrow is his third birthday, but I want to wait a week longer since I have nothing at all in the house to bake a cake. We want to invite the children from the lodge, a five-year-old boy and a three-year-old girl, who, apart from the Indians, are the only children for miles around.

Christopher eats everything you give him, from sauerkraut with blood sausage to raw celery and onions. He is so different from Andri in his manner; he laughs more, he cries more, he cuddles more and he boxes more, very uncomplicated. He drools all day long, and only very rarely does he sit still and look at a book. He makes funny attempts to build something; Andri was building his dreams at his age. He is now slowly starting to talk. He follows me at my heels, sticks his fingers into everything until he gets a smack. So it goes on and on forever; I have to try to shovel the windows a bit clear tomorrow.

Warmest regards,
Your Dorli.
\end{document}
