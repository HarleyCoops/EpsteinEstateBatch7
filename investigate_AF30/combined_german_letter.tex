```latex
\documentclass{article}
\usepackage[utf8]{inputenc}
\usepackage[T1]{fontenc}
\usepackage{geometry}
\geometry{a4paper, margin=1in}

\begin{document}

\subsection*{Quelle: AF30FE3E-86DB-4D4E-B9A9-BD84926C0F6F\_1\_105\_c.jpg}
Toronto, den 3. 1. 1956

Lieber Onkel Max!

Gelt, das ist lange her, seit ich Dir zum letzten Mal geschrieben habe. Aber Du hast sicher von der Grossmame gehoert, was ich hier so treibe, und dass es mir gutgeht.
Ich will gleich zur Sache kommen, ohne mich bei der Vorrede zu lange aufzuhalten. Mutter hat mir geschrieben, dass Du jemanden suchst, der Dir einen Teil der Arbeit abnimmt und Dich entlastet. Ich weiss nicht, welche Ansprueche Du stellst und ob ich denen gewachsen sein wuerde, aber die Arbeit wuerde mich reizen, Deine Korrespondenz zu erledigen und zu helfen, wo es ginge.
Ich habe hier einen Steno- und Schreibmaschinenkurs mitgemacht, und werde im Fruehjahr sowieso eine Stelle in einem Buero suchen. Ich hatte zwar auch jetzt die zwei Jahre als Diaetassistentin viel Bueroarbeit zu erledigen, aber ich moechte mich doch lieber zur Sekretaerin empor arbeiten, das waere ein Beruf, der mich, glaube ich, befriedigen und ausfuellen wuerde.
Wie waere es mit einer Probezeit? Dann haette ich eben eine Zeitlang Urlaub gemacht von Canada, und koennte immer wieder zurueck.
Lieber Onkel Max, glaube nicht, dass ich mir jetzt gleich Luftschloesser baue und todungluecklich bin, wenn nichts daraus wird. Das war eben ein Versuch, ein Stellengesuch, und dass man da nicht immer unmittelbaren Erfolg hat, habe ich hier gelernt. Da schaut man eben weiter. Aber ich moechte Dir nur noch sagen, dass ich mich gefreut haette, es waere eine Aufgabe.

Herzlichen Gruss,

Dein Dorle

\end{document}
```