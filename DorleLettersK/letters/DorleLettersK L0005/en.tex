\documentclass[12pt]{article}
\usepackage[utf8]{inputenc}
\usepackage[T1]{fontenc}
\usepackage[english]{babel}
\usepackage{geometry}
\geometry{margin=1in}
\usepackage{parskip}
\begin{document}
\section*{DorleLettersK L0005 (English)}
\noindent
St. Catharines, Feb. 27 / 53

Dear Mother,

Your letter arrived the day before yesterday, and I was very happy to get it. On the same day, Bärbel's Christmas letter came, which over time had grown to be 14 pages long. She writes quite good letters, aside from a few awful mistakes. She also told me about your drive in the snowstorm, and about Carnival.

But now to get straight to the plans: When I promised you back then, before I left, that I would take over the business later, I really did intend to. At the Betzels', I had also developed a great fondness for fruit farming. But it's really true that one slowly becomes a stranger to this field. Even though I was on the farm in the summer, it was only as one worker among many, so I couldn't really get a proper feel for it. And then it's not really fun, either; I would rather work late into the night. But it's just different, and I can tell that I've already been looking around at other things. So I don't know for sure if I will go to the farm again this summer, but would rather try to either move on in the winter or take a course. Yes, that is what I, too, would like to have clarity on, so that I could align my plans with it a little. Until now, I had really always firmly counted on going back, if only because I know that you can't do it in the long run and are counting on it; but you know, I've only been here a year now, what will it be like after 5 or 6 years? Then it will probably be a blank slate, and then the adjustment will of course be much, much harder. And what if I get married? I mean, in these 5 years, that could certainly happen. Then you will have worked yourself to the bone, and in the end, you'll have to sell or lease it out anyway. Wouldn't it really be better to lease it out? Why should you always have to struggle with the property, quite apart from the fact that it's physically too hard for you. In case Christel really did have Girth later, but she couldn't do it alone either. And if, but I believe, then you would have to move away from Flethen, so that you don't see what the other person is doing with the business. Secondly, what would you do then? You could sell the Opel and the U I, and buy yourself a small house and you should be able to get by on the rent that you I. But could you just sit there and do nothing? You're too young for that. One can do that for half a year.

It must be a difficult decision for you to part with the business, where you both worked so hard to build it all up. But it's also not very gratifying when you the work that should be done, and can't be done. I understand so well that you wish for one of us to carry it on, but I'm afraid that I would then very often think, "how much freer and brighter and more comfortable you could have had it if you had stayed." Also the wish one then gets used to things again over time.

You know, many people don't particularly like it here at all, but they stay because they earn well, but for me it's not just about the money, I can so easily imagine how one can make a decent, comfortable life for oneself, a cozy house, a car to drive out on Sundays, there is such beautiful countryside there. And above all, if one has children someday, they have a good future.

I've just written this as it came to mind, a bit jumbled, isn't it?

As I already wrote to Uschi, I would love to be a nurse. There is a course for nurse's aide, 8 months long, with a small wage, \$60 a month, I believe. I would very much like to take it.

I would be happy to send the money for one of the little ones' education, if you have leased out the farm. And in 2-3 years Uschi and Liesel will be earning too, so we should be able to pull that much together.

Enough for both.

Warm regards,
Doch.

For my birthday, I would like the drawing templates of the horses and the Alps (a colored one).
\end{document}
