\documentclass[12pt]{article}
\usepackage[utf8]{inputenc}
\usepackage[T1]{fontenc}
\usepackage[english]{babel}
\usepackage{geometry}
\geometry{margin=1in}
\usepackage{parskip}
\begin{document}
\section*{DorleLettersH L0002 (English)}
\noindent
1955-? Toronto, Dec. 27

Dear Mother,

That was a wonderful Christmas. Thank you so much for the great shoes, they're just what I wanted. If I put an insole in them, they'll fit well. And the sofa blanket is just what I can best use, and above all, it matches the tablecloth and the pillow so well. I was very happy about it. Now that Christmas is over, I'll also have more time for drawing, so I'll get started on the heads. Thanks again for everything. Two days before Christmas, I also received 5 books from Ackermann, which were from the months of June to October. Do you know how many books I was supposed to have received? I had received them regularly for 12 weeks and thought that was all, so when I moved, I didn't send my new address to Ackermann. Could Bärbel perhaps go to Ackermann and give them my new address? By the way, I really enjoyed the books. Medri said that you have the books too. I'll list what I have, I don't even know which volumes I'm missing. "Maja", "Der Favourite der Königin", "Haifische", "Der Süßhaftige", "das einfache Leben", "Minna eine Nomine", "Die gute Erde", "Strom des Schicksal", "der erfolgreiche", "Am dunklen Fluß", "Königin Viktoria", "der zauber faden". And now I've received: "ein Mann allein", "zwischenfall in Schwindel", "Petroleum für die Lampen Chinas", "der fremde Gott", "Cimarron". Which one was the prize-winning work?

Mecki has surely told you in great detail what Christmas was like at our place. It was really very nice this time, it gets better every year. When I think of my first or second Christmas, it was certainly different. Or one time on New Year's, I played until 2 o'clock in the morning and the next day I was at the cinema at 12 o'clock out of boredom. This time there will be 12 of us together, that will surely be nice. At first we wanted to do it at Inge and Hilde's, but now it will probably be at the home of a newly married couple from the group. We also want to do lead-pouring and play games.

The record evening was very nice. By the time I had converted the recipe, Mechi had already mixed everything together. She thought it was a bit simpler, doing it all without a scale. And then the butter was too salty for her, so she poured in more sugar, which turned the butter cookies into sugar cookies. Mechi was green with rage, and I thought they tasted good. For Christmas, I got a pair of first-class ice skates from Inge and Hilde, you know, the kind where the boots are attached. We'll be skating fast on those. Inge used to be a very good skater as a child. Mechi gave me yarn for a scarf, hat, and gloves, all white, perfect for ice skating. Grandma's package also arrived today, she's sending me a book by Uncle Hans, “Eskimo Künstler.” That seems very interesting. I'm very much looking forward to reading it.

Yesterday we went to see “The Messiah.” We were only able to get bad seats, but we all enjoyed it, the choir is wonderful, the soloists too, especially the soprano. The singer is small and crippled, but truly captivating. The choir is made up of men and women who do it for the love and joy of it, without pay. Last year I heard the same choir perform The Messiah and Beethoven's Ninth Symphony.

Yes, Mechi will have written to you about the photos. That was, of course, a terrible blow for Mechi, she was more attached to her camera than I was, since I hadn't taken a single picture yet. Not to mention the money, of course. Oh well, Vali, who is coming for a week in the spring, will gladly bring a camera over for us, which will be much cheaper than if we buy one here. Then Mechi won't have to pay me back the travel money yet, maybe she'll have the money together by then. — To be honest, I wouldn't have written home about it at all if Mechi weren't here, you have enough worries as it is. But sometimes I have my doubts about whether I shouldn't have sometimes written home about my unpleasant experiences too, that way you must think from my letters that everything was rosy for me and not at all for Mechi. Not the story about things in Canada, I've never embellished that but have always described it as I saw it at the time. But did I tell you how I sprained my knee, lay in bed for two weeks and had to deal with it for half a year? Or the sprained ankle that I had bandaged for 5 months? Or how, with a month to go, I ate nothing but burnt pudding with a devil-may-care attitude and walked to work because there was no money left for honey, bread, and gas? Or how I crawled into bed at 5 o'clock after work because the room was unheated? But that was at the beginning and it's over now, and it really didn't bother me at all, since I did it all voluntarily (except for the sprained joints!).

But now I wish you all the best for the New Year and not so much work and lots of joy and health.

Best wishes,
Dorle.
\end{document}
