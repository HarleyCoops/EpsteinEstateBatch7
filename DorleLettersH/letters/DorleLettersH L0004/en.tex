\documentclass[12pt]{article}
\usepackage[utf8]{inputenc}
\usepackage[T1]{fontenc}
\usepackage[english]{babel}
\usepackage{geometry}
\geometry{margin=1in}
\usepackage{parskip}
\begin{document}
\section*{DorleLettersH L0004 (English)}
\noindent
What made me happiest was that I found a job all by myself, without a friend helping me, although one could hardly... but that's only the second thing, that I know, or the sister asks me if I have ever worked in a hospital before. I wasn't entirely comfortable admitting it, but now I'm glad I did. I don't work in the kitchen, but on the 2nd floor with the patients. In the mornings, I work together with a 20-year-old Finnish woman, we first have to fill the tea kettles with tea water, put water in the glasses, and with ice and water. Then we tidy up the washstands, the windowsills, and so on, then very slowly we get to enjoy talking with the patients a bit. In nine rooms, it's not uncommon to have women who enjoy life anyway when you speak kindly with them, and so while they are combing their hair, they try to tell me as much as possible.
A woman wants to get into the water, or which one of us she wants to drink from each cup, etc. Yes, but of course I remember very precisely who doesn't. It makes me happy, like when you really love the flowers and put them in the vase after giving each one fresh water. When we've done all that, it's a quarter to 11, and then we eat until noon. After half an hour it's downstairs, they clean the living areas and the toilets and afterwards the laundry that has come up from the laundry room below. Then it's about 1 o'clock, and depending on how many admissions come, we also have to make the beds with fresh linens. Sometimes only 1 person comes, or as many as 8. When we are finished with our work, we see if we can help the nurses, otherwise we just do nothing, that is, we chat with the patients. For a few days now, it's been getting more and more complicated day by day to powder! It's a world of its own, how much fuss over nothing more than just having to wait for quitting time. Every wage in a German factory - always with me so I can act as an interpreter. But it's very convenient that I speak very little English. So the company takes me... pains, stiffness in the abdomen, so I need that, because recently I had an appendectomy. Of course I don't know all the words that didn't come to mind daily. I can't... if no one is talking over you, you still can't make them up. And learn, and above all it's the best opportunity to learn English. I start tomorrow at 8 o'clock, finally until 4 o'clock. So I have the day and Sunday off. I leave in the morning at 7 o'clock, ride with her to her hospital, then walk another 20 minutes. I eat breakfast at the hospital, in the evening I eat supper at 4 o'clock then walk after. It is, however, about a 3/4 to 1 hour walk, but it still saves 6-7 dollars a month in fare, and with that I can buy stamps, soap, toothpaste, etc. The patients are very nice, most of them also often give us grapes, oranges, samples or some other sweets. Meanwhile, in the... is that there is no permanent chauffeur, but rather... with him, so it happens that in the... get religiously honored. Many also send... There are also Black doctors and Japanese doctors, the Japanese are... I have to learn again, otherwise I speak too much German. The stupid thing is that your friends don't make fun of you, no matter how much nonsense you talk. In the evenings, I sing Christmas carols and hiking songs with the other Germans, when we're not singing together in a club. - By the way, Grandma wrote to me yesterday, and she is a bit horrified that I was cleaning in a household. Is that so bad? I think it's better in any case than sitting around and living off other people's money. And that's also the nice thing, that nobody is looked at differently because of the work they do.
This afternoon I had off, and I was at Mrs. Lean's to do the ironing. That's the Mrs. I was with the most, and who is very nice. She was offended at first because I went to the hospital, but now she's eternally grateful when I help out on my day off.
We've already had about -18°C here, today it snowed, in the city it's a terrible mess, but at Mrs. Lean's in the north it looks quite Christmassy. On University Street there is a weather station. On top of the building is a pillar of light that indicates the weather. It works like this: If the pillar is white at the top, there will be snow, if it's red it's cloudy, and if it's green it's clear. The pillar of light also flashes on and off, from top to bottom when the temperatures are falling, and the other way around when it's getting warmer. We drive past the thing every morning, it's nice.
The evening before last I said "Good night" loudly to the ceiling, and yesterday she said it first, well, it'll come along.
Is Irmgard coming to visit you?
Please tell everyone like Mrs. Stähli, Lothi, Klärli, Jessica, etc., that I... and didn't wipe the bathtub dry to the last drop, she said we were dirty girls. I think she either doesn't like Germans, or she's envious of our youth. In the future I will try very hard not to splash another drop of water, in the bathroom that is. But when she, for example, goes to the toilet, she makes a point of not closing the door. Oh well, one shouldn't take it so tragically. It's funny, Darla, Inge and Hilde are very much for this American smile, when you can put yourself on display, they kind of hold it against me that I don't like it. (Apart from that, I am by no means unfriendly, on the contrary I often feel as childish as Darla.) The thing is, this American friendliness is perhaps... but towards us, it's the exact opposite. It's always, go, go, quick, quick, we really have to go now, and that's when we're just getting up in the morning. The other day she had a theater ticket for "Intrigue and Love," and I wanted to quickly take a bath, since I had just come from cleaning, but that took too long for her again, and so I couldn't go, I would have liked to. She is cold, as she always was before, often very unpunctual and monosyllabic, but afterwards she always feels sorry. That's what I mean, I'm getting to know a lot here, that's certainly worth a lot too, and maybe I'll get used to it. But I haven't talked as little as I do here in a long time. We barely say "Good morning" or "Goodbye" to each other, I only ever hear a kind of mumble.
That just fits my judgment of the country of Canada and its people perfectly: everything is contradictions, whatever you encounter: one morning it's -18°C, in the evening it's warm, the markets are...

made up, but they live with runs in their stockings and wear underwear they would throw away, in the bathroom the tiles and the sink are polished, but the ceiling is black. I could list many more such examples for you. By the way, when I was talking with the ski instructor last Sunday, he had the same opinion. One cannot form any other judgment about Canada than that it is the land of contradictions. - By the way, I want to tell you again that I'm not telling you the less pleasant things so that you feel sorry for me or worry, but you should know how one feels in the beginning, what one misses and what one likes. To most other people I write that I like it and that I've settled in quite nicely, but you know me, and I think this way you can imagine a little bit what it's like here. I know that I often judge too quickly, but it's the first impression and that is important; of course it will change again, and I will write to you then. I've gotten somewhat used to the smearing, at first it bothered me terribly, because at home I never washed less than up to my mouth. Fingernails are actually painted less here than with us in Germany. And I will get used to many other things, especially the fashion, which by the way is fabulous, I haven't seen anyone in 3/4-length pants yet, it could be that that also depends on the season. The coats here are very nice, and there are very beautiful, simple evening dresses here. A very nice one costs about 20 dollars, I'd like to buy myself something like that sometime, because on Sundays it's all the same how you're dressed, at least I still like my sweaters just as much as I did back in Ravensburg.
\end{document}
