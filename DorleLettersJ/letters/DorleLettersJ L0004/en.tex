\documentclass[12pt]{article}
\usepackage[utf8]{inputenc}
\usepackage[T1]{fontenc}
\usepackage[english]{babel}
\usepackage{geometry}
\geometry{margin=1in}
\usepackage{parskip}
\begin{document}
\section*{DorleLettersJ L0004 (English)}
\noindent
1954 aug 14
Woodview, Aug. 14th 54

Dear Mutti,
I know I haven't written to you in a long time, but I always think that you have so little time to write letters, and the letters to Uschi are for everyone anyway. In Toronto it was also so outrageously hot that it made you lose all motivation. But right now I have time, because I'm on vacation. Yes, on Saturday, Inge and I stuffed the car full of provisions and drove here. We had already been here twice with the youth group in the spring, and now we've rented a small cottage for a week. It's really only one room, but it's divided by walls into a bedroom, kitchen, and living room. Outside is a beautiful veranda, right on the water. We feel like we're in paradise here. We have two rowboats and a motorboat for ourselves, so we row and have races on the lake. It's wonderful for swimming here; the lake isn't deep, but it's very clean. Yesterday we went on a boat tour with the owner in her boat. We traveled over 40 km through a chain of lakes, through locks and canals; it was truly wonderful. We read a lot and lie in the sun, play the accordion or the flute, or we play cards, and most of all we lie in the sun and do nothing—a lazy life, isn't it? And you have to work so hard. But still, this is the first vacation in 3 years where I'm not working, and lazing around for one week every three years isn't exactly fair, is it?
In the mornings and at night it's icy cold here in the north. We put the heat on, though, and wear everything we have, but during the day it gets really hot in the sun, wonderful swimming weather. Unfortunately, we can't go to the little island in the lake, because first, it has an awful lot of snakes there, which you can see when you row past, and second, something called "poison ivy" grows there, which is a plant that causes a terrible rash that you have to deal with for days and weeks. Everything around us here is primeval forest and lakes and then more forest. I have to pause now. Inge is bringing breakfast—hot chocolate and buttered toast.

Now it's noon, after lunch. Today we cooked on the fire pit by the lake, which tastes better than on the electric hotplate in the cabin. We even got mail, from our youth group. On Saturday, one of our members was supposed to give an introductory lecture on photography, but it was canceled due to low attendance. Another group went up north to Algonquin Park for the weekend. They left Friday evening, got there at 1 a.m., slept in the car, and had a wonderful time. In this park, there is only one road, and it only runs along the very edge. It's a huge area of lakes and primeval forest with no trails leading through it; the only way to get around is by canoe, mostly through the lakes and rivers. Sometimes you have to carry the boat, and it goes on like that for days. Hikes are only allowed with a special permit or with a guide, because too many people get lost. By the way, there are wild buffalo and bears there, moose and deer.
\end{document}
