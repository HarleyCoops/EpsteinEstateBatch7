```latex
\documentclass{article}
\usepackage[utf8]{inputenc}
\usepackage[T1]{fontenc}
\usepackage{geometry}
\geometry{a4paper, margin=1in}

\begin{document}

\subsection*{Quelle: 0FCFEB7A-1BF4-44DF-BDF4-CA4C296F0DE8\_1\_105\_c.jpg}
da auch Peter Bartl dort, obwohl sie doch
nicht gut englisch spricht. Die Kurzschrift
fällt mir nicht schwer, es geht nur
viel zu langsam in der Schule.
Am Montag haben wir wieder eine Auffüh-
rung in der Turnhalle anlässlich der
"National Health Week". Wir haben wieder
neues Leben eingeübt, das ist heissen Tanz
Seilchen und Reifen. Das macht wirklich
großen Spass. Ich dachte ja schon
Anfang, dass die Turney, glaub ich, das
einzige ist, was mir hier gefällt. Es ist das
beste an der ganzen Schule. Trampolin
springen, Seilchen, Basketball und Boden-
turnen. Wir spielen beinah jeden Tag.
Nächste Woche ist wieder ein Ballspiel-
turnier. Contests to ??? Beethoven
??? ??? der Kaffee hier schmeckt ???
nicht ??? leicht, dass sie nicht
nachtragen sind. ???
dort bleiben fest und können
anschließen. ???
(der zurückgeschickt wurde) Sie
hat 13\$ gekostet, das sind
82 DM. ??? ein paar ???
??? ??? ???
dabei allerdings keine ???
??? ??? ???
es vielleicht so machen, dass ???
Geld von mich ???

\subsection*{Quelle: EED86E5E-FB5B-406B-BD3B-6BF870AB0BC3\_1\_105\_c.jpg}
nicht mit. Ich war eben bei Sigismund und
habe ihm das Heft gezeigt. Er hatte Bedenken
ob wir die Aussprache hin kriegen, da ja
außer Rein hold wirklich alles Deutsche
sind. Aber ich finde, wir probieren es auf
jeden Fall. Ich will sehen, daß wir einen
ganzen Abend ausfüllen können mit dem
Spiel und mit Chören und Musikvorträgen.
Arbeit macht es schon, wenn man sich so
um alles kümmern muß, zumal ich wieder
überbesetzt bin mit der Abendschule.
Ich habe wieder Töpfern (wo ich jetzt an der
elektrischen Scheibe bin), dann Turnen
(mechi geht jetzt auch einmal die Woche) und
dann Steno und Maschinenschreiben.
In Steno und Schreiben muß man auch
immer Hausaufgaben machen. Übrigens
was das Ernst mit der Sekretärin beim
Onkel Max? Das wäre allerdings eine Stelle.
Da hätte ich schon Lust dazu.
Gestern, am Sonntag fuhren wir mit drei
Autos los morgens um 4 Uhr, da wir näm-
lich eine Fußwanderung machen wollten.
Nach Collingwood an der Georgian Bay fuhren
wir und kletterten dort in Höhlen und
Felsspalten herum, und machten Streifzüge
durch den Wald. Es war so schön sonnig und

\subsection*{Quelle: 888E3F1D-5DBF-4E46-B0F1-28A8CCAFFED1\_1\_105\_c.jpg}
Vancouver, 31. Juli / 55

Liebe Dorli und Mucki!

Viele Grüsse und mehr! Soeben schritt zur Heimreise - westwärts. Trotz eines furchtbaren Wetters, um nicht zu sagen - Mistwetters, habe ich nach herrlichen Tagen vorletzte nur zu kurz, da man schliesslich für alles Zeit braucht und auch die Sonne wollte als Thema auch beachtet werden.
Herzliche Grüsse und ein Bussel
Eure Grette

\subsection*{Quelle: AF30FE3E-86DB-4D4E-B9A9-BD84926C0F6F\_1\_105\_c.jpg}
Toronto, den 3. 1. 1956

Lieber Onkel Max!

Gelt, das ist lange her, seit ich Dir zum letzten Mal geschrieben habe. Aber Du hast sicher von der Grossmama gehoert, was ich hier so treibe, und dass es mir gutgeht.
Ich will gleich zur Sache kommen, ohne mich bei der Vorrede zu lange aufzuhalten. Mutter hat mir geschrieben, dass Du jemanden suchst, der Dir einen Teil der Arbeit abnimmt und Dich entlastet. Ich weiss nicht, welche Ansprueche Du stellst und ob ich denen gewachsen sein wuerde, aber die Arbeit wuerde mich reizen, Deine Korrespondenz zu erledigen und zu helfen, wo es ginge.
Ich habe hier einen Steno- und Schreibmaschinenkurs mitgemacht, und werde im Fruehjahr sowieso eine Stelle in einem Buero suchen. Ich hatte zwar auch jetzt die zwei Jahre als Diaetassistentin viel Bueroarbeit zu erledigen, aber ich moechte mich doch lieber zur Sekretaerin empor arbeiten, das waere ein Beruf, der mich, glaube ich, befriedigen und ausfuellen wuerde.
Wie waere es mit einer Probezeit? Dann haette ich eben eine Zeitlang Urlaub gemacht von Canada, und koennte immer wieder zurueck.
Lieber Onkel Max, glaube nicht, dass ich mir jetzt gleich Luftschloesser baue und todungluecklich bin, wenn nichts daraus wird. Das war eben ein Versuch, ein Stellengesuch, und dass man da nicht immer unmittelbaren Erfolg hat, habe ich hier gelernt. Da schaut man eben weiter. Aber ich moechte Dir nur noch sagen, dass ich mich gefreut haette, es waere eine Aufgabe.

Herzlichen Gruss,

Dein Dorle

\subsection*{Quelle: D3B4CE53-5E9E-4ED1-B75D-54B317CECCB9\_1\_105\_c.jpg}
hat man nicht eben, wenn man jeden Sonntag
fängt mit Geis Gartonen, das strengt auch
an. Der einzige Küster sind, muß jeden
Stunden einen 40 Pfund schweren Kloßsack tragen,
durch Sand, ohne Gabel. Außerdem wohnen wir am
auf dem eines Berges, der ist immer auch eine
Kletterwanderung, um man grünt oft werden die
Leut und schliene Kletter köhnen kaum sehen also
über 3000 Metern.
In der Bank macht es viel Spaß. Aber auch viel
bei beiden von Orten an in Vancouver. Da ist auch dumpf
die Polizei andauernd am in der Bank und schimpf
jetzt darum. Gefällt das alles, viel
sagen, man kann keinen, natürlich
bekommt man viel mit und muß viel
und bin jetzt Ratti, und viel
nase für gefälschter oder ungedeckter
und viel gefälscht. Eben auch ein
in Canada. Die Banken haben nicht genug
Banken, so stellen viel Betriebe ihre
nicht voll, die Arbeiter kaum nicht
kriegen, und so geht das rund.
So, jetzt muß ich zur Arbeit.
bin und unter eigentlich wie es
mich immer obendrauf und halte
daß die jetzt viel zu schnell vergeht.
Herzliche Grüße von Euch.

\subsection*{Quelle: 637BEC8E-B339-40D4-B5A5-EDFF30FCB1D5\_1\_105\_c.jpg}
daß es ihr hier gebe, Aber wenn Du für
50 DM noch keine schöne kriegst, nimm
ruhig mehr, dann ist es noch Osterge-
schenk oder nachträgliches Hochzeits-
jubiläumsgeschenk!
Der Winter ist hier eigentlich kälter als
die letzten Jahre, ziemlich viel Schnee
und jeden Tag Sonne. graue Wintertage
gibt es hier ganz, ganz selten, entweder
es schneit oder es scheint die Sonne,
ohne daß gleich das wasser läuft, es
ist trotzdem alles bockelhart gefroren.
Nebel gibt es hier vielleicht 3 mal im
Jahr. In Calgary hat es 2,70 meter Schnee
Da kann ich ja nächstes Jahr seh froh sein,
daß Christel der Rock zu eng war, hätte ich
nicht gedacht, er war 5 Nummern
größer als meiner. Aber das macht hier
scheinbar die andere Ernährung und
Luft, daß man dünner wird, Mechi hat
auch abgenommen. Dafür bekommt man
aber schlechte Zähne. Die Zahnärzte hier
sagen, das beinahe bei allen Neukana-
diern so sei. Mechi hat auch den Mund voller
Löcher, sie wartet aber mit Ausbessern
und Zahnweh auf die Krankenkasse
in Deutschland. Hier zahlt ja bekannt-

\subsection*{Quelle: F2FEC3D0-858C-4E82-AD20-BEC1FB81011D\_1\_105\_c.jpg}
waren, und alle hatten diese Ausfahrt
wieder so richtig genossen. Abends mußten
wir allerdings um 8 Uhr zurück sein
für den Club, wo wir einen hochinteressanten
Vortrag hörten über den Einfluß der Welt-
geschichte auf die Kirche. Ein guter Redner
kann auch das langweiligste Thema
interessant machen.
Inge und Klaus sind jetzt wieder öfters
im Club, Klaus gefällt es gut bei uns.
Ich noch öfters dort, natürlich nicht mehr
soviel wie früher. Inge geht wieder mit
zum Turnen, aber sie hat natürlich
jetzt mehr mit dem Haushalt zu tun.
Klaus ist Elektroingenieur und hat eine
Stelle als technischer Zeichner. Denke dir,
8 Tage vor der Hochzeit ist um Glückten seine
Schwester und sein Vetter tödlich in Deutsch-
land. deshalb machten wir die Hochzeit so
klein. Morgens um 10³⁰ Uhr war die Trau-
ung, die wirklich sehr nett war, dr. Gögginger
macht das sehr nett. Manfred und ich
waren Trauzeugen. Sonst war außer Frau
Pastor niemand in der Kirche, da die Jugend-
gruppe ja auf Fahrt war. Dann hatten wir
das Hochzeitsessen, Kartoffelsalat und Schnitzel

\subsection*{Quelle: C362477B-1DBB-47B4-AED8-F4A5FD0189BB\_1\_105\_c.jpg}
Monti, 17.12.55

Liebe Tante!
Zu Weihnachten wünsche ich Euch alles Gute
und recht frohe, gemütliche Feiertage. Hat es
Schnee bei Euch? Bei uns schneit es beinahe jeden
Tag ein bißchen, und kalt ist es immer, manchmal
sogar bis zu -15° Celsius. Am See hinüber
wird es jetzt auch schwer fahren.
Kältere Advent. Na, vielleicht klappt es am
Weihnachten, aber da arbeiten wir bereits
abends nach der Kirche; die eine um
6 Uhr, wir zur 10. oder 12. Bei Inge und
Klaus grillt sehr gut. Inge mit
den armen Buben backen und Kartoffel-
salat machen. Von der Gruppe aus machen wir nichts
als wie ja am Sonntag Abend Gruppenabend haben.
Um zur Weihnachtsaufführung: Denkt Euch, was soll
kommen? Ein Schattenspiel am Samstagabend.
Für alles andere am Sonntag vielleicht
zu Saal was gestopft voll, wie am letzten
Stühle aufbauen. In der ersten Reihe
Grafen gesessen, dahinter auch
schon Zivilisten. Nur beim
sie nicht beklagen, weil das Kostüm
nicht so schwierig war. Ich hatte obiges
Dedl an und unter eine grüne
nicht so richtig vorstellen, aber die
die gute haben, alle.

\subsection*{Quelle: 14ED5AD3-F4EF-4090-BFDB-0A454E494C60\_1\_105\_c.jpg}
Es hatte doch Jeder eine andere Auffassung, wie
eine Rolle gespielt werden soll. So ließ ich
sie erst' mal ihre Rollen spielen, ohne was
zu sagen, und wo es gut war,
anders war, als ich es in Erinnerung hatte,
ließ ich es dann so, ich dachte doch
besser, wenn sie es natürlich spielen als wenn
man ihnen was anderes einübt, sie
Herodes Szene z.B. ist etwas anders wie als
wir sie gespielt haben, der Herodes fängt nicht so an =
auf dem Thron, aber ein ausgeprägtes
Minenspiel, und die Herodes zene läuft auch probe
in jedem seiner Schritte.
es schadet, was daran zu ändern, die ist auch
so gedacht, der Heinz ist auch sehr gut als Walther,
und gerhart als Mohr, macht die Maria
auch sehr gut, und der Josef ist auch allerliebst,
es ist eben so ein berliner Josef, aber
Josef. Der Herodes ist auch Soldat,
so als wird das schon viel klappen, jetzt
schon muß man seine Kleider anprobieren
zu spielen nachher am Entmittags
seine Frau spielt es gut, d. h. die
eben fast nicht zu spielen eben
Stoffel ist gut. Ich glaube schon,
Es ist am 3 adventt abends, und
schon Einladungen weggeschickt.

\subsection*{Quelle: 0CF69970-07F4-4F17-B009-DFEAD2DC3E29\_1\_105\_c.jpg}
gesagt, es schien sehr gut ausgesehen; man hat
hat es auch sehr gut aufgesagt. Herr Grummel
begrüßt die Euch erst, denn
von Amsterdam. Weihnachtslieder
umgezogen war. Die Sternsinger
Urkant, Ellen, Dagr und An
weißgewanden an und einen Stern für die
das Goldreif im Haar und einen
den Herold. Gleich anschließend send
Garten und wie der Vorhang zu
nach, und ich "Ehre sei Gott in
jeden scenes mußten wir ja binnen
och ansehen, so daß wir jedes mal
tanz hatten. Dann kam die Ver
Mutter und Sybille als Gabriel. Ich
Mutter eine Folie gemacht und
amelchen Kindel-Gewand und einen
amen Stern auf dem gewand unter
ne. Die Wirtsscene: Joe als Wirt an
Apfel mitge, Kartoffeln und meine
Vera als böse Wirtin und Sy
der Josef kletter ein grünes Gewand
auf dem Boden, mit einem grauen
und schlüpfte gewickelt, Bart und P
Stab, der die Krippenscene sahen w
lieber weglassen müssen, da das
Proben immer schlecht ging. Dann
eine gefüttert. Bei den Hirten, seine
Fell auf der Bühne, ansonsten mit

\subsection*{Quelle: 7F6DD828-69E2-401C-B7C7-76ECEDDD901F\_1\_105\_c.jpg}
Mont Laurier, Juli 12, 55

Liebe Luti!
jetzt ist schon die Hälfte vom Urlaub um.
Es war einfach prima. jeden Tag haben wir
2-3 Stunden im Wasser gelegen, das heisst, auf
der Luftmatratze. Kochen tun wir nicht, wir essen
Salatgurken, Büchsen erbsen mit Essig + Öl, und Brot
Grapefruit und Oranges und getrocknete Aprikosen.
Schlafen im Auto geht prima, wir machen es
so, daß wir uns gegen Abend an einem See
waschen und umziehen, und wenn es dunkel
wird, schlüpft Inge schon in den Schlafsack hinten
und wir fahren zur nächsten Ortschaft, wo wir
dann entweder bei den ersten Häusern
oder auf einer Tankstelle schlafen. Morgens
früh um 5 Uhr krieche ich dann aus
meinem Schlafsack und fahre weg, ehe
die Leute kommen, Inge schläft meistens noch
hinten. Am nächsten See wird dann gewaschen
und gefrühstückt und dann fahren wir weiter
ein Stück, bis wir mittags zum baden auf-
gelegt. Wir haben unsere Fahrt gar nicht fest-
gelegt, das ist gerade das schönste. Nachdem
wir aus Montreal waren ging's am River
entlang bis Quebec. Quebec ist wunderbar, eine
amerikanische Stadt mit Stadtmauern. Von
Quebec aus ging's nach dem Norden zu, immer
noch am St. Lawrence River entlang, teilweise
ging aber die Strasse ins gebirge, d.h. es ist

\subsection*{Quelle: 2D7B2CAE-846A-446A-8965-AD6BB7610F7B\_1\_105\_c.jpg}
Toronto, den 19.11.55

Liebe Mutti und Geschwister!
Wie geht es Euch allen? Jetzt
abends auch wieder mehr Zeit
hoffentlich ist es mir, jetzt es immer
wegen der Schule. Ich bin wieder die ganze
Woche abends in der Schule. Töpfern, Turnen
Schreibmaschine und Kurzschrift. Und Sonntag
ist Jugendabend und dann bleibt wenig Zeit übrig.
Schriftführerin, das macht auch Arbeit.
Aber im Sommer habe ich dann
ein schlechtes Gewissen gehabt, wenn ich von
Eurer Arbeit für Fiten geschrieben habt und ich
fuhr um 3 Uhr nach der Arbeit zum Baden.
Zwischen letzten Brief und dem heutigen
aber ich schreiben mit dem Feder-
einen Federhalter mit dem ich noch
Jahr stammt, die glaube ich noch
kasten gefunden. Der Federhalter
wie einen den Bartel kriegt.
mit x Federn vorne, wo man
kriegt. Und all das kriegt
so teuer einen Kugel schreiber kriegt
unbedingt, mal wieder anfing
bist, der einzige die Namen der
so einen keinen Birn geschenkt, kriegt
Töpfern, Kuchen formen von
wißt Ihr, was ich heute gemacht habe?

\subsection*{Quelle: FE3885B3-6200-428A-AFFE-54BCC38E87C3\_1\_105\_c.jpg}
in Trenklamm mit, nimmt noch Hilfe stunden
und versucht, aufzurolen. Wenn nicht, kann
man dann das Abitur in diesem Fach nicht
machen, falls es ein Wahlfach ist. So ist man
dann vielleicht in Mathematk in der VII
und sonst in der OII.

Wir fangen schon an, unsere Wohnung zu schmieden.
Wir feiern am Neujahr bei uns
da, da wir meinen wir eine anständige Portion
Italiensalat und Würstel, denn
allen Julchen und Herrige machen
denn mit Essig Zwiebeln, kein
Alkohol werden sie oder trinken
machen wir einen heißen Punsch, und Backes.
das Auto fahren wir einen Braten und essen den
am nächsten Tag kalt, und gut für
8 Uhr !? morgens zum Schifahren
Schande Onkel Max einmal schicken
wir sagen kann er ja nicht. Wo
Am besten ist, ich schicke ihn an
nicht? Da ist es doch am ehesten.
Onkel Hans zurück? Weißt Du, wenn
Onkel Max nur auf Probe wäre
außer den Reise natürlich, falls
schon schön für eine Weile. Da
können wir bei uns den oder
was. Aber ich bin mir da ganz
sind, und ich möchte, fahr es ist und
Herzlichen Gruß und alles
wünscht Dir
Dein
B.C. (Namen?)
Onkel (?)

\subsection*{Quelle: D05BFCB8-1383-4C2C-AA52-05B492BC6549\_1\_105\_c.jpg}
Bisher aber trieben vor der ganzen Stadt
Vitaminen sonder, aber in
ist das wirklich so werden
auch alle Mehl und mit
andren Nährmittel hinzu
Vitaminen bereichert, so daß
Water unser Brot tatsächlich
Vitaminel, wir in vollkomm=
besten Willen nicht vor Zucker zu
täuschen. Wir benötigen ihn
ein Saccharine (Süßstoff), aber
künstlichen sechs Briefe giebt es jetzt
portal für unser Jochen Brande
die Ohrenchen lassen. Er ist auch
billig, aber Jochen kostet ein
auch nichts, 10¢ das Pfund.
haben wir viel geschlachtet
sie Rollen, seit ich weg bin \& 12
Jungpflanzen jungen gemacht,
Pachtland. Herr Getzel macht
auf Type 9 oder jetzt Type
Eier, so der gerüst. Da reicht so
dann ist den Boden ausgenützt
und die Pflanzen abfällig.
machen, beinaht aber es ist ja
machen. Er hat aber Erfolg damit.
Herzlichen Gruß. Jrek.

\subsection*{Quelle: A8FE6959-3E4E-490B-974C-70AE6A11EE7E\_1\_105\_c.jpg}
dießen übertrieben vor, der ganze
Vitaminen zauber, aber in der Stadt
ist das wirklich wichtig, so werden
auch alle mehle und milch und
andere nährmittel künstlich mit
Vitaminen bereichert, so daß im hiesigen
Watte weißbrot tatsächlich die gleichen
Vitamine sind wie im Vollkorn =
brot. Übrigens, wo ich gerade von
künstlichem rede, hier gibt es jetzt
ein Saccarine (Süßstoff), der ist beim
besten Willen nicht von zucker zu un-
terscheiden. Wir benützen ihn im Hos-
pital für unsere zuckerkranken und
die Abnehmkuren. Er ist auch sehr
billig, aber zucker kostet hier ja
auch nichts, 10¢ das Pfund. Betzels
haben mir auch wieder geschrieben
sie haben, seit ich weg bin, 812 morgen
Jungpflanzungen gemacht, meist auf
Pachtland. Herr betzel macht ja alles
auf Type 9 oder jetzt Type 4 mit
Eisenrohr gerüst. da reicht 20 jahre Pacht
denn ist der Boden ausgenützt und die
Pflanzung abfällig. Das ist
modern, beinahe wie es die Farmer hier
machen. Er hat aber erfolg damit.
Herzlichen gruß. Dorle.

\subsection*{Quelle: D138AC8E-5BD5-42FD-924A-30627CC75A88\_1\_105\_c.jpg}
D. Hofmann
k. 243 Franklin Ave
Toronto / Ont
Canada

\subsection*{Quelle: 2C59B68D-F839-4D92-BD04-8CF060A9D2B4\_1\_105\_c.jpg}
und dann lauter so kleine "Appetizers"
Ich hatte eine schöne Butterkremtorte
gebacken, mit Rosen und Stiefmütterchen
drauf gespritzt. Die Jugendgruppe hatte
Inge ein schönes Teeservice geschenkt,
das wurde gleich eingeweiht. Dann fuhren
wir los nach Woodview, wo die beiden
Bliebtreu, und Manfred und ich fuhren
weiter. Das war ein ganz nettes Stück zu
fahren, über 300 meilen (480 km); nachts
um 1/2 12 kamen wir zu den andern nach
Baysville.
Inges Schwester Hilde geht es gut in Deutsch-
land, sie arbeitet noch nicht, wird aber bald
in einem amerikanischen Büro in Wiesba-
den anfangen. Sie war den ganzen Sommer
mit ihrem Vater auf Reisen, in Holland und
in Belgien und in Italien.
Unsere neue Wohnung ist sehr nett, wir
haben jetzt doch mehr Platz, wir haben aber
auch entsprechend mehr Besuch. Manchmal
ist die halbe Gruppe hier. Du fragst nach
Weihnachtswünschen. Ich wünsche mir sehr
einen Anorak zum Schifahren mit Kaputze,
am liebsten blau. Oder aber einen guten
Fahrtenkocher. An kleineren Sachen wünsche ich
mir, Etui für Filter und Sonnenblende, um
an den Riemen zu machen, oder Waschbeutel.

[Text in left margin, written vertically]
Oder eine schöne Decke für mein Zimmer.
Ich möchte Dir noch eine Decke schicken, aber ich
denke, daß es vielleicht besser ist, wenn Du Dir
selbst eine kaufst. Herzliche Grüße von uns allen.

\subsection*{Quelle: CCB5AF43-EE21-4DE8-B4D3-A23945667023\_1\_105\_c.jpg}
auch gebacken, Butter backen, Käsebroten
und Kuntzelbrot. Unser Christbaum
wesentlich jünger und voller
Fülle als im letzten Jahr, er ist ein Prachtstück.
wieder hat gestern Bilder gemacht da wandert Ihr
Ja Bilder will ich Dir schicken und ein
dem Heiligabend sind wir ja nach der Kirche alle
alle zur Inge gegangen, Manfred, Siegfried,
Heinz, Joe, Edland Schwestern u. ich.
Da wir alle sehr hungrig waren, hat nur 16
gegessen; Kartoffelsalat und Würstchen, jeder so viel
er wollte. Joe, unser alles premier
\& Würstchen verkasimiriat. Dann zündeten wir
wir den Christbaum an, d.h. steckten
die Steckdose und wir sangen dazu Weihnachtslieder
und wir sangen dazu Weihnachtslieder
wir dann Gesellschaftsspiele gemacht
getrunken, Schallplatten gehört, viel
lacht. Nur essen es iller
nächsten Morgen.
Am Neujahr werden wir die Speisen
uns kochen, 16 - 18 Mann, mal
kochen, Bilder sehen. Das können
Ellger, Hilde kochen dürfen hat
und geschrieben, wie sie sich in einen
U.S. Luftwaffe angestellt, und
dort auch einen Bordstein Club dort
Tage dort ein Programm einer Projekt
schenkt bekommen, jetzt möcht Er
auf Weihnachten
schenkt auch in dem

\subsection*{Quelle: 0B8C252D-B36E-4501-907F-5B51FF04A75F\_1\_105\_c.jpg}
man registriert, Zeit und Datum, und
am Ausgang wird man untersucht, ob man
nicht gejagt hat. Das war eine großartige
Tour, dann ging es quer durch die
Laurentians bis Shawinigan Falls und St. Jerome
Von dort fuhren wir die High way hinauf in die
Hoch-Laurentians, wieder über 1000 Meter hoch,
aber nicht so wild, eher wie Allgäu, und
so richtige Allgäuer Häuschen. Es wohnen näm-
lich da viele Deutsche und Schweizer. Und
da kamen wir zufällig am "Santa Claus Village"
vorbei. Das ist so was Nettes und geschmackvolles.
4 Montrealer Geschäftsleute haben es gebaut.
Natürlich verdienen sie daran auch, aber die
Anlage hat sie 200 000 Dollar gekostet. Es ist
ein großer Hügel mit ungefähr 7-8 bunten
Hexenhäuschen, in jedem ist was anderes.
In einem werden indianische Handarbeiten ver-
kauft, einem mit Postkarten, einem mit Waren
oder Andenken. Im Kapellchen ist die Weihnacht-
krippe und die Glocke kann man läuten und
sich was wünschen. Und in einem Häuschen
ist der Weihnachtsmann mit seinem Schlitten
und er spricht mit den Kindern und gibt ihnen
Zutaten. Er sprach deutsch mit uns, und hat
uns alles erzählt über die Anlage. Im Garten
und in den Häusern laufen Ziegen und Lamas
und junge Bärchen herum, alle ganz zahm.

[Text in left margin, written vertically]
dies kann man nicht oder nur unter größten Schwierigkeiten mit der Schreibmaschine schreiben

\subsection*{Quelle: EA93F742-3BBE-4588-8D0F-D2B687E803C9\_1\_105\_c.jpg}
Toronto, den 10.10.55

Liebe Mutter!
Vorgestern kam das "gotteskind" an. Vielen
Dank dafür. das tut mir leid, daß du solche
Scherereien damit hattest. Aber wir sind heute
auch gleich aus Rollen abschreiben gegangen,
ich habe mir 3 jungens geholt, die mit ge-
holfen haben. Die Rollen haben wir vorläufig
so verteilt. (Gottvater: André); (Engel gabriel:
Erika); (Josef: Sigismund,); (Maria: Michi); (Guldin-
sach: Uli ;); (Wirtin: Vera); (Herodes: Manfred)
(Tod: Werner); (Teufel: Wolf); (Rabant: Horst Wer-
mescher); (Michel: Horst Mielde); (Stoffel: Siegfried)
(Cyrich: Werner Kjar); (Melchior: Hans Dieter)
(Walthauser: Heinz); (Caspar: Reinhold). mit den
Sternsingern sind wir noch nicht klar. Ich spiele
nicht mit, da ich doch die Leitung übernehmen
muß und sowie so mehr Mädeln da sind
als Rollen, höchstens den ersten Sternsinger.
Nächsten Sonntag ist erste Probe. Inge will
die Kostüme machen. Musikanten haben
wir allerdings gute in der gruppe. Manfred
spielt gut klavier, ebenso Hans-Dieter, Klaus
(Inges Mann) spielt sehr gut geige, auch Ellen,
Werner spielt mandoline und Reinhold neben
Accordeon Blockflöte. André spielt auch sehr
gut klavier. Uli spielt querflöte, hat sie aber

[Text in left margin, written vertically]
1955
10.10.

\subsection*{Quelle: 2202DE8A-E2B2-4E4D-B6B0-415638B24E31\_1\_105\_c.jpg}
Die Johren Tante Jette hatte Joe in seiner Fabrik gemacht
odrusen es jedenfalls, der gedreht und Blut daran
Das hat vielleicht einen Krach gemacht als er
dem Karolus das Schwert von der Füße warf,
und so kindisch, daß der Karolus
Frieden wünschte, wenn er nicht
leben wollte. Der Trabant war
Eckhard von Schwern. Der Jockel war
Ein trüber, das Gesicht gepudert weiß in
dunkel umrandet. Das war Werner Schmedter.
er sieht privat so ein bischen so aus wie ein Jockel,
und hat auch eine ganz dunkle Stimme. Der Teufel
hatte schon enge hosen an, schwarzen Bellosen
engen roten Umhang und schwarze Kappe mit
kleinen Hörnern. Das Gesicht mit schwarzem
er hatte eine gabel und sprang herum, das felt war
es war sofort nach dem es vorkam, hell war
Die letzten Scene war auch gut, nachdem ich
mit Werner auf der Bühne aufgezeichnet hatte
wo die Heilige stehen sollten, die sonst im
an die anderen gestanden waren.
Jochen machte ihre Sache sehr gut,
sangen wir alle zusammen an
still, still! (Das andere kannten wir nicht
hatten keinen Text mehr zum einüben) Nachdem
kamen die Persönlichkeiten und
uns, mit uns sagten es war das
was seit dem Bestehen der Gemeinde
erreicht. Das hat dann auch Spaß gemacht.

\subsection*{Quelle: FF4B809C-7C0C-48C2-858F-61E8957C319F\_1\_105\_c.jpg}
Ja nach nacht arbeit kann sie nicht
vergrößern, und Inge wollte ihren Vatern einen
Kalender schenken. Da habe ich den Vergrößerungs
pack mit zu Inge genommen und habe ihr
3 Sandkapseln gehalten die vergrößert
alle prima geworden, und sie sind
so ein Bißchen schief eingefegt, aber da war
natürlich der dämliche Rahmen
schuld! Meine Negative waren
gut, da war es ja nicht schlimm
kein einziges versaut, und ich
habe vor der Zeit verspätet, und
da mein Job ja immer schon das
Inge hat mein Stativ, und mein
Fotoapparat macht mir viel Spaß, ich ziehe
am liebsten allen es, früh morgens, oder so,
da kann man am besten knipsen
Brandt man doch sprit. Will mir
ein großes Apparat um kaufen, wenn ich die
kann fällst. Ich glaube, 'Krauss' heißt
sie wieder zu hause ist, die hat in
zu Hause weh, und sie blättet oft
Listen hin um. Vielleicht ist es bei
real, so kein bißchen namen trocken
voll och so, ich weiß nicht genau.
Vergrüßen sind wir ja nicht so
sammen, da sie öfters weg geht oder
jetzt arbeitet sie für einen Monat

\subsection*{Quelle: 20FAB697-E70B-43B3-8CE3-F393651858E4\_1\_105\_c.jpg}
ungefähr 1000 meter hoch und ähnelt
dem Schwarzwald, nur daß die einzige
Straße, die durch oh geht, nur ein Weg ist,
wie sie vielleicht die alte Straße nach Meran
burg. Da ging es andauernd auf und ab,
im 1. Gang rauf, im 2. Gang runter,
so steil war es. Und dann alle 2 Meilen
ein herrlicher See. In St. Simeon übernachteten
wir, da war es eisig kalt. Das
River ist dort schon salzig. Da ist uns auch
was nettes passiert. Nachts schlossen wir immer
das Auto ab. Beim Volkswagen kann man den
von innen aufmachen, aber nicht von außen.
Morgens wischen wir uns und bauten die
Türen zu, die Schlüssel steckten am Armaturen-
brett. Zufällig war der "Hood" vorne offen,
wo das Werkzeug drin ist. So dachte ich, ich
kann bestimmt auch was die Photoklauer
in Toronto konnten. So habe ich einen
Schraubenzieher krumm gebogen und in
1/4 Minute war das Fenster aufgebrochen.
Das war unser Glück, denn wir waren nur
im Badeanzug in ziemlicher Kälte morgens
um 5 Uhr, und Autowerkstatt gibt es natürlich
da oben keine mehr. Dann kam ein wunder-
bares Stück, Berge und Seen, alles ungefähr
1000 meter hoch, Wald und ganz einzelne Häuser

\subsection*{Quelle: 802B2E3C-BB43-47D0-8348-225DF619EC52\_1\_105\_c.jpg}
Wohl Dich in Dich werden wir auch in Kitchener
St. Catherines oder Hamilton spielen und den
Erlös werden wir dem Kirchen Bund fond den
Kirchen geben.
Ich habe mir schon was nettes ausgedacht, was
wir in der Gruppe machen können nach Wahl:
Ich werde eine Art Lotto Quiz mit ihnen machen,
ihnen 10 Fragen stellen, die sie auf einem
Zettel beantworten. B. lautet die Frage: Hauptstädt
hätte zum jeder nicht beantworteten Frage bekommt
man einen Punkt, für denjenigen der zuerst
50 Punkte hat, ein Buch. Derjenige der am einen
Abend die meisten richtigen Punkte hat, darf
für den nächsten Abend das Thema wählen.
z.B. über Napoleon oder Goethe oder Schiller, oder
oder Geometrie oder Physik, oder Karl den Großen,
irgendwas, aber so, daß jeder in seinem Lexikon
sehen oder irgendwo darüber lesen kann.
Stellen am darauffolgenden Abend mal die Fragen
wenn man doch vielleicht öfter in ein Lexikon
man sich auf den nächsten Abend vorbereitet.
Das ganze geht in nur 10 Minuten.
wieso Mitglied der deutschen Kirchgemeinschaft,
da bekommen wir für 1.85\$ sehr gut Bücher oder so.
Für heute will ich aufhören.
Viele herzliche Grüße
Dork

\subsection*{Quelle: 827C8FD5-EC38-4557-93F0-742016ACEF38\_1\_105\_c.jpg}
Monto, Sept. 19, 55

Liebe Mutter!
Heute nur ganz kurz, mecki macht eben
Bilder, eine Bestellung von 450 Bildern, da
muß ich helfen fixieren.
Aber nun schnell, weswegen ich schreibe,
mecki und ich haben uns ausgedacht, daß
wir doch mit der jugendgruppe das gottes-
kind aufführen könnten. Wäre es möglich,
daß du es uns per Luftpost schickst? Das
Porto geht auf unsere Kosten, wir haben sowie-
so 5 Alben von Photopost bestellt, sodaß
alles auf eine Rechnung geht. wir haben uns schon
die Rollen ausgedacht, da müssen wir aber
so bald wie möglich mit Proben anfangen.
Unsere Fahrt war prima, die Hochzeit von
... er und ... ich schreibe eben während
der Arbeit, deshalb mehr
in ein paar Tagen.
Herzlichen Gruß
Dork

P.S. Wenn bärbel es sofort abschickt,
schicke ich postwendend ein Reel für
den Viewmaster.

\subsection*{Quelle: 14126989-478F-4FA7-A253-9DCE700BD615\_1\_105\_c.jpg}
keine Krankenkasse, und auch nicht die
teuerste Privatkasse, den Dentist,
vermutlich weil sie sonst Bankrott
machen würde. Dasselbe mit medi-
kamenten, wenn man nicht im Hos-
pital ist. Jetzt, wo die Gesund-
woche ist, wird wieder sehr darauf
geachtet, daß jeder Bescheid weiß
über Vitamine und Calorien, ich
muß sagen, da wird hier viel mehr
darauf geachtet, und frische Äpfel
und Orangen werden sehr viel gegessen
nur, auch viel mehr frisches gemüse
als in Deutschland, das zu billigen
Preisen zu kaufen ist. Es wird viel
mehr Salatt gegessen, vor allem Kraut-
salat und Kopfsalat, dann rohe
Karotten (nur so ganz) und rohen Celery
rohen Spinat (sehr gut) und Radieschen
gurkenscheiben (ungemacht) und Paprika.
Der Salad wird hier überhaupt meistens
mit der Hand gegessen, zu jeder Mahl-
zeit irgendetwas frisches. So werden ver-
hältnismäßig wenig Bananen gekauft,
obwohl sie so billig sind, aber da sind
nicht viele Vitamine drin und machen
so dick. Das kommt euch vielleicht ein

\subsection*{Quelle: 8DF74E2B-A1C5-4465-81A4-07112F27C684\_1\_105\_c.jpg}
oder ich nie über Haupt soviel an
Chie Woche, d.h. schlafend schreibe ich sie jeden
Morgen
Hilde Kochendörfer hat mich wieder
geholt in Deutschland. Aber wie war
nicht, du Type für bin, viel zu sehr
empfindlich und genau, außerdem
drüben, im Sarg an dem der bei
hat letzte Woche angefangen zu
So wie der Onkelchen aus
sicher nett. Inge und Klaus machen
mit im Gruppenleben, das ist für
morgen an als Elektroingenieur, bis
als technischer Zeichner gearbeitet.
und Heute haben wir mit Sigmund
die Kostüme besprochen, die Frage
ich glaube, die werden ganz schön.
ungefähr 30 yards Stoff kaufen,
25 cents, das sind 7.50 f und den
was also ungefähr 10 f das können
kann tragen. Wir proben heute, ein
die Wolle, zuerst sah es so aus,
ins Wasser fiel, aber da Sigmund
Hans einmal war fort auf unserer
antlitten wir alle Killek, aber Ihre
ihren Mann Paul, am Sarg waren
er ist jetzt mit unserer Männer der
unser Günther der Teufel, Joe's Bruder

\subsection*{Quelle: 9C378423-7A50-4EC1-B993-7B50CC6975B9\_1\_105\_c.jpg}
einer Gattin; Die Hirten waren
als Wickel, alt- und rostig, Bart, Hut, Jocken =
Umhang und Schaftstiefel als Stoffel.
Mittelalter, kappe, gewickelte Füße
und kurzer Umhang und Hirtenjacke
oder als Gyniak, ein Korb, keine Kopf-
bedeckung, gewickelte Strümpfe und Pelz weste.
Gedacht beim Gyniak und Wickel
die beiden an Cäcilia gerecht hat.
Der Stallzimmer ganz prächtig in der Wirkung
Gestupft, daß sie bestimmt wurde.
Der Engel, den sollte wie Gabriel werden,
gekleidet, das sah gut aus, bevor der
gen wir, vom Himmel hoch da komm
und summten es während der Engel

waren Heinzi, Reinkold und Gerold.
als Wolken zu war wie ein europäer
Teppich und Krone. Reinkold als Wickel
ein Wollener kleiner, nur weiße Gewand
saß. Der nicht war bunt und
hoch und = Himmel und Schäfchen. Dann
eine, unsere beste Szene. Herodes.
Oder kommt sehr nicht. Er war wie
det, weißes gewickeltes Gewand und
Schärpe oder Umhang) darüber, goldene
und am freien Arm Goldspangen.
Der Palast war wie ein römischer
Grüner Waffen Remel, Helm und gewickelte Beine

\subsection*{Quelle: 986F5036-072A-4697-A89A-6AB607A2AE66\_1\_105\_c.jpg}
Toronto, 26.12.55
Dec. 26
1955

Liebe Mutti!

Ich arbeite die ganze Woche "split",
d.h. ich habe frei am 1. - 4. Das ist praktisch zum
Briefpackschreiben. Also zuerst vielen Dank für das
Packet. Der Anorak passt prima, (nur die Ärmel sind ein klein bischen zu lang)
wir sehr gut. Ich finde es jetzt
Jungs. Der Gürtel ist vielleicht ein bischen
am Tag ist vielleicht schick, da er recht
Die Bündchen können wir schon
ist nichts schönes ein, und die Jacke
Sehr beutel ist auch so zum Koch.
meinen d.h. was prima. Jedes denn bin hat
und auch zum Sommer und
Kann oder jetzt kann ich
Schifahren gehen, von Inge habe ich
bekommen, fehlen mir noch die Schier
wollen, welche Erich noch hatte nicht her
ich auch alles so gut brauchen kann.
Grossmama hat ein Packet über
schenkt, von einer Cousine her geschenkt
hinüberbefreien. Tante Elfi hat
zusammen ein Buch von den Alpen mit
geschrieben, von Fredi die Photos.
Ja gesehen, die mit ja ganz toll
und Klaus haben mir eine Armbanduhr
meine ist ja seit 5 Jahren kaputt.
wieder ganz komisch, eine Uhr zu
Dank auf für die Gutsel, wir haben

\subsection*{Quelle: 6A6307A9-091E-4886-8D62-656C40CF256F\_1\_105\_c.jpg}
Dondo i. Lindi, d. 27. Juli 66.

Liebe Mutter,
heute kam Dein Brief mit dem von
Onkel Max, vielen Dank dafür, sowie
für Deinen andern Brief. ney, das
wollte ich nicht, da so 4. oder 5. Tage
vergangen sein. Die Karten sind auch
dabei gewesen. Theklutd bekommt
ihr Sekretärinnen sehr oft Geld
oder auch Kleider. John hat ja
das alles übrig, passt auch
noch ganz. Abgesehen von seiner Lehre
bekommt er 45-50 £ pro Woche, das sind
schon 800 DM im Monat. gut spricht auch
übrigen Bezirkskrankenhäuser sind
über 1000 P.M. weißt Du, Angel hier
ist es selten, daß sich ein Mädchen wirk-
lich interessiert und die Verant-
wortung übernimmt, das merke ich ja an
mir im Hospital. meine vorgesetzte
ist seit 6 Jahren im Hospital- (3½ J. hier
und 6 Jahren im anderen Hospital)
fragt mich oft abends, ob sie aufnimmt
gehen kann, ich mache Rich alles fertig
oder oft, wenn es besonders viel ist
bleibe ich abends länger oder
fange deshalb früher an. Das ist
nicht selbstverständlich, so sagt
sie. Gegenüber der Bank, die

\subsection*{Quelle: 73543016-6D46-4258-AA8D-1BC579353AD0\_1\_105\_c.jpg}
7
Weißt Du, letztes Jahr hatten die
ein Krippenspiel aufgeführt, das war mit allen
Kanonen. Es war auch eine gute Reklame für
unsere Gruppe, und es kamen gleich 3 nach dem
Spiel gefragt, ob sie nicht kommen dürften.
Morgen findet die Sonntagschule auch ein Krippen-
spiel auf, da müssen wir ja hin und es uns
ansehen.
Heute ist letzter Tag um die Karten und Briefe
abzusenden, auch für Toronto, und ich muß
noch 45 Karten schreiben.
Liebe Mutti, ich hoffe, daß Du den Kaffeekocher
brauchen kannst. Ich habe gedacht, wenn ihr
morgens auf den Markt fahrt, da kannst Du alles
hinstellen, und bis Du im Bett gewaschen
hast, oder fertig angezogen, oder gefrühstückt, ist der
Kaffee fertig, und er kocht nie über.
Das Ding für den Kaffee heraus und das
Fülle kalles Wasser ein, je nach dem wieviel Du willst,
4, 6 oder 8 Tassen, aber nicht weniger als 4 Tassen.
(Da sind Zeichen.) Dann tust Du die Röhre wieder
mich den Kaffee behälter, und fülle
behälter Kaffee, da's umgibt Du ausprobieren, wie stark
Du ihn willst. Zege unten rein einen von den Filter blät-
tern, wenn Du willst, ist aber nicht nötig, es wird eine
kuller Tasse kuhl, warmer, wenn Du den Kaffee
wird, mach den Deckel drauf und stecke die Schnur

\subsection*{Quelle: 8068D011-0E32-4577-87CF-8B85D5A8595F\_1\_105\_c.jpg}
Jetzt gekommen ist Paul Kmosch, der Trabant
Joe ist da, Guldinsach, Vera Ohs weiter, eine
neue Sybill, der Engel Gabriel, Sternsinger, Grimm
Sternsinger, Grimm einmal, Siegfried, Horst möch-
te aber die Hirten, Heinz, kein Gold und Julaid
die Könige. Da Manfred nach dem
zurück kam, gaben wir ihm den Zettel,
aber er steckt immer noch
bei keinen Proben nicht dabei. Wenn ich ihm
dann richtig grob ist, meine ich
es im Moment nicht, eingeschnappt und an
Erich läßt er mich zu einem
Der Manfred ist mit einem Grund ein
parat stellen. Er macht es nicht, aber
fing das Spiel vor lauter, das sei
zum Spielen nicht für Kinder,
der Retter er war eine Wut, weil
den Kainzledel mit dem Auto nicht
Später sah er das, daß das Spiel nicht
schon mal in Hornburg im Theater gespielt
worden war, da war es auf einmal
Es ist nämlich Hamburger was
Josef Gott vater abgespielt hat bei
soll nicht recht. Ich habe schon
dags keine Kritiken Bilder gemacht
worden hinter einem Künstler der
mal dann öfters mehr Wert auf die

\subsection*{Quelle: C18F0088-7034-4C79-B371-DC5F90238988\_1\_105\_c.jpg}
Einmal lief uns ein großer brauner Bär
beinahe über den Weg. Er saß im gras am
Wege und als wir vor bei fuhren, machte er
sofort kehrt und trabte zum Wald. Schade,
wir konnten ihn nicht photographieren.
In Chicoutimi gingen wir sogar ins Kino, wir
sahen den gleichen Film wie an meinem
Geburtstag "Aschenbrödel". Dort ist überall
viel Holzindustrie, die ganzen Flüsse sind
voller Stämme, so viele, das habe ich noch gar
nie so gesehen. Wir haben viele Bilder gemacht
da werde ich davon mal schicken. Am nächsten
Tag ging es durch den Laurentide Park,
das ist ein "provincial park" von 4000 square
miles und mit 1600 Seen. Da darf nicht
gejagt werden, aber das wird gefischt, bis
2 Meter große Lachse und eine Art von Forellen.
Aber man kann dort schlecht lang bleiben
und wohnen, weil alles links und
rechts der Straße solcher Urwald ist, daß man
nicht zwei meter weit rein kann. Aber geba-
det haben wir trotzdem. Der Weg ist einfach
rein gehauen in den Wald, links und rechts
liegen noch die Bäume. Einmal mußten
wir stoppen, da liefen zwei Elchkühe über
den Weg. Und einmal sahen wir zwei kleine
Bärchen, die da spielten. Vor dem Park wird

\subsection*{Quelle: A2527E89-0520-4702-8DC2-C3F254E80448\_1\_105\_c.jpg}
sein. Das ist alles. Der Koffer Das
an nähen den Koffer zu binden
und hört von selbst auf, wenn der
ist. Dann bleibt soviel Hitze im
nie kalt wird, aber auch nicht
also angst, daß Du Koffer kochst
Wenn Du den Kaffer auf warmen
nächsten Tage oder so mußt Du in
mit den Kaffer behälter, ein tun, aber
ohne und sehen, daß der Kaffer nicht
ist, wenn Du fühlst, die Kaffer klar
Wasser außen, wegen den Herzen, wenn
handelt, böse lasse den Deckel weg.
ein kann.
Säbels und alles Pulloren habe ich
gedacht es.
schreibt es. Für Grecel ist der Kleider-
wenn keinen Schrank hat, mal die
Kleider bürsten usw. Für Christel ist es
bel pßt es.

Nun noch viel herzliche Weihnachts-
grüße und ein frohes neues Jahr
wünscht Euch allen Drk

\subsection*{Quelle: 78DA7650-517E-4149-9A40-A76063732C52\_1\_105\_c.jpg}
kaufen, daß sie Farbstiften machen kann. Ich habe
Inge eine Farbage com (Hochsmeltweisen? oder wiel?) einen
geschenkt, so einen, wo man mit dem Kopf auf einen
Knopf drückt und der Deckel aufgeht, und für
kleine Kinder eine Rille backen.
Zimmer einen Jugendherbergenkalender geschenkt.
Am selben wie Ihr ich habe
Am 4.1. fängt die Abendschule wieder an, wir
hatten 2 Wochen Ferien. Jetzt es sind
es eh, vor allem große Ferien im
10 wochen lang. Da gehen dann die Hälfte
aller Schulkinder arbeiten, in
Fabrik, in Hospitals als Kellnerin
deshalb gehen die Fabriken, Hospitals
auch sehr runter und andere
da oben. Ja und die ist da in
Highschool. Ja und eine Universität, man
sucht sich selber seine Fächer aus besonders nach
dem ersten Jahre.
und Geschlecht ist Pflichtfach da
noch eine bestimmte Anzahl Fächer dazu
Kinder z.B. mit Wahlfach, Physik
Schreibmaschiene, Buchführung, Haushalt-
lehre, für Jungens auch praktische
Klassen, Dreharbeit, in Jeder werden
sonst wie Arbeiten. Wenn man in
durchkommt bleibt man aber wegen
andern macht diesen Fach noch
nochmal mit all

\end{document}
```