\documentclass[12pt]{article}
\usepackage[utf8]{inputenc}
\usepackage[T1]{fontenc}
\usepackage[english]{babel}
\usepackage{geometry}
\geometry{margin=1in}
\usepackage{parskip}
\begin{document}
\section*{L0004 (English)}
\noindent
Monti, 17.12.55

Dear Aunt!
For Christmas, I wish you all the best and very merry, cozy holidays. Do you have snow? Here it snows a little almost every day and it's always cold, sometimes -15° Celsius. The year is almost over and on Christmas Eve, it will get even colder. Yes, Advent goes by so quickly. We'll be working on Christmas. On the 24th, in the evening after church, which is at 6 o'clock, we will be there until 10 or 12. Inge and Klaus play the violin very well and we will bake potatoes and make potato salad. We aren't doing anything with the group, since we have our group night on Sunday evening anyway.

As for the Christmas performance, guess what's happening: a Swabian nativity play, maybe on Sunday. We have to do everything that needs to be done, for example, setting up the chairs in the hall. In the first row, there are 8 benches where the counts sat, including the shy Zittau. They couldn't complain about the costume, because it was intended for Feldt. I had the above-mentioned top and a green middle part. You can't really picture it, but the people… You know, last year the Anthroposophists put on a nativity play that was really something. It was also good publicity for our group, and right after the play, 3 people asked if they could come. Tomorrow morning is Sunday school and a nativity play will be put on, so we have to go and watch it.

Today is the last day to send letters and cards, even to Toronto. I still have to write 45 cards.
Dear Mutti, I hope you can use the coffee maker. I thought that when you go to the market in the morning, you can turn everything on, and by the time you've made the bed or gotten your mail, the coffee will be ready.
(You'll have to learn how to use it. You pour in the roasted coffee and water, take out the thing for the coffee grounds, and fill it with cold water, depending on whether you want 4, 6, or 8 cups, but not less than 4 cups. Then, put the coffee container back in, and fill the container with coffee. You'll have to experiment to see how strong you want it. Put one of the filter papers in the bottom if you want, but it's not necessary. The coffee will be brewed. Put the lid on and plug it in. That's all. The coffee starts to brew after 10 seconds and stirs itself. When the coffee is ready, it stays hot enough that it never gets cold, but it doesn't boil either, so you don't have to worry about that. If you want to warm up the coffee the next day or so, you have to put the inner tube with the coffee container in, but without coffee on top. Make sure that the water isn't in the coffee when you fill it. Never get water on the outside because of the heating element. When you're not using it, leave the lid off so fresh air can get in.)
I washed the sabels and pullovers myself; it's so easy. If you ever have the chance, write. For Grosel, there's the garment bag; that's good if you don't have a closet, and the thing for shoes, clothes brushes, etc. For Christel, there's the cooking spoon—I hope it's suitable.

Now, many heartfelt Christmas greetings and a Happy New Year to you all from your
\end{document}
