\documentclass[12pt]{article}
\usepackage[utf8]{inputenc}
\usepackage[T1]{fontenc}
\usepackage[english]{babel}
\usepackage{geometry}
\geometry{margin=1in}
\usepackage{parskip}
\begin{document}
\section*{L0004 (English)}
\noindent
Monti, December 17, 1955

Dear Aunt,
For Christmas, I wish you all the best and a very merry and cozy holiday season. Do you have snow? Here it snows a little almost every day and it's always cold, sometimes even down to -15° Celsius. With the call-up this year, it will certainly seem harder now. In terms of the cold. Well, maybe it will work out during Advent. On Christmas, we'll already be working in the evening after the church service, which is at 6 o'clock; on the 24th in the evening after church we'll be working until 10 or 12. Inge and Klaus are very good at grilling. Inge and Klaus will make potato salad for the poor lads. We're not doing anything as a group, since we have our group night on Sunday evening anyway. On the 24th at midnight is the performance. Guess what's supposed to happen? A little chat on Sunday, perhaps. The hall was jam-packed, like the last time we set up chairs. In the first row sat 8 boys, among them one of ours, from the German pupils. Only for God the Father they weren't allowed to join in, because the costume was so similar to mine. I had a golden-brown blanket on top and a green one underneath. We can't quite imagine it, but the people all— Everyone had a different idea of how they should play their roles at first. So I let them act it out first without saying anything, and where it was good, even though it was different from how I remembered it, I left it that way, because it's better if they play it naturally than if you drill something else into them. The Herod scene, for example, is a bit different from how we played it; Herod no longer sits on the throne, but is a real glutton, and the Herod scene is rehearsed while drinking.

There's no harm in changing something about it. It's excellent; Heinz is also very good as the guard and Gerhart as the Moor, still, still and very good, and Josef is also— well, he's just a Berlin-style Joseph, but great. Gerhart is also a soldier. Joe as the innkeeper has already learned a lot; you already have to curb his talkativeness. Later, at the beginning, he plays well opposite his wife, that is, she's just harsh, not to be curbed. Stoffel is good. I think so. It's on the evening of the 3rd of Advent; invitations have already been sent out.

Now Kunocha has arrived, the Trabant. Joe is here, Gulbinsack, Vera who is next, a new Sybill, the angel Gabriel, Ellen the Star Singer, Sigismund, but Horst wants to be Siegfried, Horst means the kings. When Manfred came back, we gave him the note, but he's always stuffing socks and is never at any rehearsal. When I then tell him off really bluntly, he's properly intimidated for a moment and in the end, he just gives me a gentle laugh. Manfred is— for one reason, you can tell he doesn't do it, but the play started out of sheer— that it was the cold that made me so angry, because I later saw that the play had already been performed in a theater in Hamburg, and then suddenly it was— You see, it's a Hamburg thing, what was performed with Josef as God the Father is not supposed to be right. I have already made pictures— no tears were shed behind an artist who then placed more value on the—

Said they looked very good; it was also recited very well. Grandma first greeted the people; then Mr. von Amsterdam played Christmas carols. Grandma had changed her clothes. The Star Singers sang before the arrival; Ellen, Dagr, and Angelika; they wore white robes and a star for the one who carried Gabriel. Right at the end, the herald with hair and a staff in the garden, and as the curtain closed, they sang again and I "Glory to God in the highest"; after every scene we had to change in a flash, so that we had on different clothes each time. Then came the Annunciation; Mother and Sybille as Gabriel. I had made a foil for Mother and a little baby's outfit and a poor star on it. And the robe underneath. The inn scene: Joe as the innkeeper and Vera as the other innkeeper and Sybille. Josef had a green robe on the floor, with a gray one and the flute wrapped, beard and staff. In the nativity scene, we unfortunately forgot to change, since that always went badly in rehearsals. Then someone played the flute. With the shepherds, his fur on the left, otherwise his with the belt. With the one donkey has quite at the and his wife; The shepherds were as Mickel, old and mature, beard, hat, jacket-like cape and shepherd's boots; as Stoffel, middle-aged, cap, wrapped feet and short cape and shepherd's crook; or as Cyrnak, a boy, no head covering, lantern, wrapped stockings and fur vest. I thought of both uncles with Cyrnak and the shepherd with the crook in a just calculation, he has them literally in the style of the room quite forcefully stuffed, so that they definitely— the angel, the same as Gabriel, looked, that looked good, before the "From Heaven Above to Earth I Come" came, and they hummed it while the angel— as world emperor; Reinkobol and Gerescholt Jepter and crown. Reinkobol as the shepherd with the crook, a woolen clown, only white robes sat. The one who wasn't was colorful with a turban head and shirt and sash. Then one, our best scene: Herod. Or doesn't come much. He was dressed, white wrapped robe and sash or cape) underneath, golden and gold bangles on his bare arm. The palace was like a Roman hall. Green weapons, sky, helmet and checkered legs.

The years Aunt Jette made in his factory, the cuckoo once had a little spat there because he wanted to see the shriek from the horn from the car or class the dog or his feet in between if he didn't slide well on the bone. The Trabant was Eckhard von Schwern. The Jockel was a fat one, his face powdered and his eyes painted. That was Werner Schmechel. He privately acts a little bit like a Jockel, and also has a very deep voice. The devil already had on such tight pants, a black pullover and a black cap with little horns. The face with a black fork and Franz Krumm, the tent was— it was right after it was over. The last scene was also good, after I had marked out on stage with Werner where the critics should stand, who otherwise had stood with the others, and they did their part very well. Ellen as the buddy was also good, and then all together at the manger "Still, still, still." (We hardly knew the other one anymore.) After that, the important people came to us and congratulated us; it was the most that had been achieved since the congregation was founded. That was fun then, too.

Perhaps we will also perform in Kitchener, St. Catharines or Hamilton and we will give the proceeds to the churches for the church fund.

I've already thought of something nice we can do in the group, by choice: either a kind of lottery quiz, I will ask them 10 questions which they answer on a piece of paper. For example, the first question is: who did not answer the capital. Everyone gets one point, and the first person to get 50 points gets a book. The person who has the most correct points in one evening gets to choose the topic for the next evening. For example, about Napoleon, or Beethoven, or geometry or physics, or Charlemagne, anything, but in a way that everyone can look it up in their encyclopedia or read about it somewhere. Then on the following evening, you might actually look in an encyclopedia more often when you're preparing for the next evening. The whole thing only takes 10 minutes. Since it's known why a member of the German church choir— we can get very good books for \$1.85.

I'll stop for today.
Many warm greetings,
Dork
\end{document}
