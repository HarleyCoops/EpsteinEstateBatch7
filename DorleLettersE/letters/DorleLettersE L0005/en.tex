\documentclass[12pt]{article}
\usepackage[utf8]{inputenc}
\usepackage[T1]{fontenc}
\usepackage[english]{babel}
\usepackage{geometry}
\geometry{margin=1in}
\usepackage{parskip}
\begin{document}
\section*{DorleLettersE L0005 (English)}
\noindent
I give it to her here. But if you can't get a nice one for 50 DM, feel free to spend more, and it can be an Easter gift or a belated wedding anniversary present as well!
The winter here is actually colder than in recent years, with quite a lot of snow and sun every day. We very, very rarely have gray winter days here; either it's snowing or the sun is shining, without everything immediately turning to slush—it's still frozen solid as a rock. We might get fog here 3 times a year. In Calgary, they have 2.70 meters of snow. So I can be very glad next year that Christel's skirt was too tight for her. I wouldn't have thought it; it was 5 sizes larger than mine. But apparently, it's the different diet and the air here that makes you slimmer. Mechi has lost weight too. On the other hand, you get bad teeth. The dentists here say that's the case for almost all new Canadians. Mechi also has a fever and everything, but she is waiting with perseverance and a toothache for the health insurance in Germany. As you know, no health insurance here pays for that, not even the most expensive private insurance, because it would probably go bankrupt otherwise. The same goes for medication if you're not in the hospital. Now that it's health week, there's a lot of emphasis again on making sure everyone knows about vitamins and calories. I have to say, much more attention is paid to that here, and fresh apples and oranges are eaten a lot, as well as much more fresh vegetables than in Germany, which can be bought at cheap prices. A lot more salad is eaten, especially coleslaw and lettuce, then raw carrots (just whole like that) and raw celery, raw spinach (very good), and radishes, cucumber slices (plain), and bell peppers. Salad here is mostly eaten by hand anyway, with something fresh at every meal. Consequently, relatively few bananas are bought, even though they are so cheap, but they don't have many vitamins and are so fattening. This might seem to you like one of those awful fads, the whole vitamin craze, but in the city, it's really important. That's why all the flours and milk and other foods are artificially fortified with vitamins, so that the local cotton-like white bread actually has the same vitamins as whole-grain bread. By the way, speaking of artificial things, there's a saccharine (sweetener) here now that, with the best will in the world, is indistinguishable from sugar. We use it in the hospital for our diabetics and for weight-loss diets. It's also very cheap, but sugar costs nothing here either, 10¢ a pound. The Betzels have written to me again too. Since I've been gone, they've planted 12 acres of young plants, mostly on leased land. Mr. Betzel does everything on Type 9 or now Type 4 rootstock with iron pipe trellises. A 20-year lease is enough for that; by then the soil is exhausted and the planting is declining. That's the modern way, almost like the farmers do it here. But he's having success with it.
Best wishes, Dorle.
\end{document}
