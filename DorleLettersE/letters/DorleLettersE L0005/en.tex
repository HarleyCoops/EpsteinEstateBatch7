\documentclass[12pt]{article}
\usepackage[utf8]{inputenc}
\usepackage[T1]{fontenc}
\usepackage[english]{babel}
\usepackage{geometry}
\geometry{margin=1in}
\usepackage{parskip}
\begin{document}
\section*{DorleLettersE L0005 (English)}
\noindent
Monti, December 17, 1955

Dear Aunt!
For Christmas, I wish you all the best and very merry, cozy holidays. Do you have snow? Here it snows a little almost every day and it's always cold, sometimes -15° Celsius.

On Christmas Eve, we will be baking cookies and making potato salad at Inge and Klaus's place after church, which is at 6 o'clock. On the 24th, church is at 10 or 12 in the evening. We aren't doing anything as a group, since we have our group night on Sunday evening anyway.

Now, about the Christmas performance. Guess what we're putting on: a Swabian nativity play. The hall was jam-packed, just like last time. We had to set up extra chairs. The dignitaries sat in the first row, including Count von Zech. I was terribly nervous, but I think the audience couldn't tell because the costume wasn't quite right. I had a green blanket wrapped around me, which you can't really picture, but they said it looked very good; and that my lines were delivered very well. The pastor greeted everyone first. Then, Christmas carols were sung by angels in front of the manger; they wore gowns of innocence, a beautiful wreath in their hair, and a star-topped staff in their hand. Right at the end, Frede and I sang "Glory to God in the highest." After every scene, we had to change within seconds, so we were dressed differently each time. Then came the Annunciation, with Sybille as Gabriel. We had a rehearsal on Friday and another on Saturday morning. For the nativity scene, we had lanterns and a star on our robes and a halo around our heads. With the shepherds and their flock on stage, we sang a song with them. The shepherds were Mickel, old and huge, with a beard, hat, cape, and shaft boots; Stoffel, with a medieval cap and wrapped feet; and Cyriak, a boy with no cap, wrapped stockings, and a fur vest. There was a lot of laughter at Cyriak and how he dealt with the other two—when Verrückel came in, he literally stormed the stable and shoved them so hard that they definitely wobbled. The angel, the same one as Gabriel, was dressed in a way that looked good before the curtain went up, singing "From Heaven Above to Earth I Come," and hummed it while the angel appeared, and also during the scene with the kings.

The soirée at Tante Jote's factory was put on by Joe, who might have seen some blood on it once; anyway, the table was set, and he might have made a racket when he threw the sword from the door, thus announcing that the dog wants peace if it doesn't want to live. The attendant was Eckhard von Schwern. Jockel had his face powdered completely white with his eyes outlined in dark makeup. That was Werner Schmerder. In real life, he looks a bit like a Jockel and also has a very deep voice. The devil wore such tight pants, black trousers, and a black cap with little horns. His face was made up with a black beard and bangs. He had a pitchfork, and you could hear it as soon as he appeared. The last scene was also good, after Werner and I had marked on the stage where the critics should stand, since they had always stood on the other side before. Ellen did a very good job as Kunigunde, and then we all sat together at the manger singing "Still, still, still." (Nobody knew the other song and we didn't have time to practice it). Afterwards, the important people came up to us and congratulated us. They said it was the best thing that had been offered since the parish was founded. The main credit for that goes to the boys.

Everyone had a different idea of how a role should be played. So at first, I let them play their roles without saying anything, and where it was good, even if it was different from how I remembered it, I left it that way, believing it's better if they play it naturally than if you drill something else into them. The Herod scene, for example, is a bit different from when we played it; Herod now relies more on his authority than on exaggerated theatrics, and Herodias doesn't scheme so much; with every step she takes, she considers whether it will harm her, what can be changed, and it's excellent. Heinz is also very good as Balthasar and Gerhart as the Moor, very, very good, and Josef is also good—he's a Berlin-style Josef, but great. Herod is also a soldier, so it seems you've come a long way. He has to patch his clothes at the beginning, and afterwards, as his wife's ward, he plays his part well, which is to say, the clothes are almost beyond repair. Stoffel is also good. I think it will be on the evening of the 3rd Sunday of Advent, and invitations have already been sent out.

The cast is now: Kuno as the attendant, Joe is there, Gulda is in, Vera without a part, a new Sybille as the angel Gabriel, Ellen, Erich as a caroler, Grimm as me, Josef is Kraus, Siegfried, Horst Möchle, Horst Weune are the shepherds. Heinz, Reinhold, and Gerhard are the kings. Since Manfred came back late from Munich, we gave him the part of God the Father, but he's still complaining. He wasn't at any of the rehearsals. When I tell him off really bluntly, he's properly intimidated for a moment, and in the end, he just gives me a faint grin. Manfred is fundamentally a peaceful person, but he makes it quite difficult for you. He was furious with me because I didn't pick him up in the car. Later, he saw that the play was already putting me under stress. He's from Hamburg, which is why Josef setting God the Father on fire isn't right.

I see them maybe once a week, that is, I see them sleeping every morning.

Hilde Kocherdörfer has brought me back to Germany. But we weren't the right type for each other, because she is so sensitive and precise. Besides, she seemed to have started working for the person who was last living with her, and that with seven... Inge and Klaus are involved in the group life, that is for tomorrow as an electrical engineer, until then he worked as a technical draftsman. Now Klaus brings a significant... And tonight we discussed the costumes with Sigmund, whether they are finished. I think they will be quite beautiful. We have to buy about 50 yards of fabric, at 25 cents, that's \$7.50, and the rest, so about \$10. We can afford that. We rehearse once a week. At first, it looked like it would fall through, but since Sigmund and Hay were insistent on our parts, we wanted to cut the roles. Her husband Fred, on Saturday he was... he is now with our men, your Günther is the devil, Joe's brother.

We will also perform in Kitchener, St. Catharines, or Hamilton, and we will give the proceeds to the church building fund.

I've already thought of something nice we can do in the group after Christmas: a kind of quiz. I will ask them 10 questions, which they will answer on a piece of paper. For example, the first question is: Whoever doesn't answer it gets one point. The first person to get 50 points wins a book. The person with the most correct points in one evening gets to choose the topic for the next evening. For example, about Napoleon, or Beethoven, or Goethe, or geometry, or physics, or Charlemagne, anything, but in a way that everyone can look it up in their encyclopedia or even read about it beforehand. The questions are asked the following evening. This way, one might look into an encyclopedia more often and prepare for the next evening. The whole thing only takes 10 minutes.

For today I will stop.
Many warm greetings,
Dork

You know, last year the Anthroposophists put on a nativity play that was beneath all contempt. It was also good advertising for our group, and right after the play, 3 people came and asked if they could join.
Tomorrow the Sunday school is also putting on a nativity play, so we have to go and watch it.
Today is the last day to send cards and letters, even to Toronto. I still have to write 45 cards.

Dear Mom, I hope you can use the coffee maker. I thought, when you go to the market in the morning, you can turn everything on, and by the time you've washed up or made tea, the coffee will be ready.
Then, put the necessary amount of coffee beans in, not less than 4 cups. Fill the coffee container with the coffee beans. Pour water into the bottom, if you want, but it's not necessary, pour a cup of cold water over the coffee, put the lid on and set the machine.
And it brews.
I want to write you briefly, the number and the height and depending on how much you want.
That's all. The coffee... stir the coffee... and it stops by itself when the coffee is ready. Then it stays hot enough so it doesn't get cold. So don't worry about brewing coffee. If you want to keep the coffee warm for the next day or so, you have to put it in and the coffee container in as well, but without coffee. If you feel that the coffee... water outside, because of the hearts, if you don't handle it, leave the lid off.
I thought of the sabers and all the powder myself. It's so simple. If you ever write, for Grosel it's the clothes... if you don't have a closet, the clothes brushes, etc. For Christel, the donkey cleans it.

Now, many heartfelt Christmas greetings and a Happy New Year to you all from Drk.
\end{document}
