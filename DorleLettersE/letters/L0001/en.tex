\documentclass[12pt]{article}
\usepackage[utf8]{inputenc}
\usepackage[T1]{fontenc}
\usepackage[english]{babel}
\usepackage{geometry}
\geometry{margin=1in}
\usepackage{parskip}
\begin{document}
\section*{L0001 (English)}
\noindent
Mont Laurier, July 12, 55

Dear Luti,
Half of our vacation is already over, and
it has been simply wonderful. Every day we have
spent 2-3 hours in the water, that is, on
the air mattress. We don't do any cooking; we eat
cucumbers, canned food with vinegar and oil, and bread
with it, and oranges and dried apricots.
Sleeping in the car is working out great. What we do
is, toward evening, we wash up and change
at a lake, and when it gets dark,
Inge slips into her sleeping bag in the back
and we drive to the next town, where we
then sleep either by the first houses
or at a gas station. Early in the morning
at 5 o'clock, I crawl out of
my sleeping bag and drive off before
people show up; Inge is usually still sleeping
in the back. At the next lake, we wash up
and have breakfast, and then we drive again
for a bit, until we're ready for a swim
at midday, which is the best part. After
we left Montreal, we went along the river
to Quebec. Quebec is a wonderful, unique
American city with city walls. From
Quebec, we headed north, still
along the St. Lawrence River. At times,
though, the road went into the mountains, which means it's about 1000 meters high and resembles
the Black Forest, except that the only
road that goes through it is just a track,
like maybe the old road to Meinburg.
It was constantly up and down,
uphill in 1st gear, downhill in 2nd gear,
that's how steep it was. And then every 2 miles,
a magnificent lake. We spent the night in St. Simeon;
it was freezing cold there. The
river is already salty there. Something
nice happened to us there, too. At night, we always locked
the car. With the Volkswagen, you can open it
from the inside, but not from the outside.
In the morning, we wiped ourselves down and the
doors were closed, with the keys in the dashboard.
By chance, the "hood" in the front was open,
where the tools are. So I thought, I
can surely do what those photo thieves
in Toronto did. So I bent a
screwdriver and in
1/4 of a minute, the window was broken open.
That was our good fortune, because we were only
in our swimsuits in the considerable cold at 5
in the morning, and of course there are no
car repair shops up there anymore. Then came a wonderful
stretch, mountains and lakes, everything about
1000 meters high, forest and very isolated houses.
Once, a large brown bear nearly
ran across our path. He was sitting in the grass by the
road and as we drove past, he
immediately turned around and trotted to the woods. A shame,
we couldn't photograph him.
In Chicoutimi, we even went to the movies; we
saw the same film as on my
birthday, "Cinderella." There is a lot
of logging industry everywhere there; the
rivers are
full of logs, so many, I've never
seen anything like it. We took a lot of pictures;
I'll send some of them sometime. The next
day, we went through Laurentide Park,
which is a "provincial park" of 4000 square
miles and with 1600 lakes. Hunting is not
allowed there, but there is fishing, for salmon
up to 2 meters long and a type of trout.
But it's hard to stay there for long
and live there, because everything to the left and
right of the road is such a primeval forest that you
can't go two meters into it. But we
went swimming anyway. The road is simply
carved into the forest; to the left and right,
the trees are still lying there. Once we had
to stop because two moose cows were running across
the road. And another time we saw two little
bear cubs playing there. Before the park, you
are registered, time and date, and
at the exit, you are checked to see if you
have been hunting. That was a magnificent
tour. Then we went across the
Laurentians to Shawinigan Falls and St. Jerome.
From there, we drove up the highway into the
High Laurentians, again over 1000 meters high,
but not as wild, more like the Allgäu, and
with real Allgäu-style houses. That's because
many Germans and Swiss live there. And
there, we happened to pass by "Santa Claus Village."
It is so nice and tasteful.
4 Montreal businessmen built it.
Of course, they make money from it, but the
complex cost them 200,000 dollars. It is
a large hill with about 7-8 colorful
witch's cottages, each with something different inside.
In one, Native American handicrafts are
sold, one with postcards, others with goods
or souvenirs. In the little chapel is the Christmas
nativity scene, and you can ring the bell and
make a wish. And in one cottage
is Santa Claus with his sleigh,
and he talks to the children and gives them
lollipops. He spoke German with us and told
us all about the complex. In the garden
and in the houses, goats and lambs
and young bear cubs run around, all very tame.

you can't wash them, or only with the greatest difficulty.
\end{document}
