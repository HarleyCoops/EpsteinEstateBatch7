\documentclass[12pt]{article}
\usepackage[utf8]{inputenc}
\usepackage[T1]{fontenc}
\usepackage[english]{babel}
\usepackage{geometry}
\geometry{margin=1in}
\usepackage{parskip}
\begin{document}
\section*{DorleLettersE L0001 (English)}
\noindent
Mont Laurier, July 12, '55

Dear Luti,
Half the vacation is already over.
It was just wonderful. Every day we spent
2-3 hours in the water, that is, on
the air mattress. We don't do any cooking; we eat
cucumbers, canned food with vinegar \& oil, and bread
with it, and oranges and dried apricots.
Sleeping in the car works great. We do it
like this: towards evening, we wash up
and change at a lake, and when it gets dark,
Inge slips into the sleeping bag in the back
and we drive to the next town, where we
then sleep either by the first houses
or at a gas station. Early in the morning
at 5 o'clock, I crawl out of
my sleeping bag and drive off before
the people arrive; Inge is usually still sleeping
in the back. At the next lake, we wash up
and have breakfast, and then we drive again
for a bit until we're ready for a swim at noon.
We haven't planned our route at all,
which is the best part. After we left
Montreal, we went along the river
to Quebec. Quebec is a wonderful, unique
American city with city walls. From
Quebec, we headed north, still
along the St. Lawrence River, but sometimes
the road went into the mountains, meaning it's about 1000 meters high and resembles
the Black Forest, except that the only
road that goes through it is just a path,
like maybe the old road to Meinburg.
It was constantly up and down,
uphill in 1st gear, downhill in 2nd gear,
it was that steep. And then every 2 miles,
a magnificent lake. We spent the night in St. Simeon;
it was freezing cold there. The
river is already salty there. Something
nice happened to us there, too. At night, we always locked
the car. With the Volkswagen, you can open it
from the inside, but not from the outside.
In the morning, we locked ourselves out and couldn't get
the doors open; the keys were in the dashboard.
By chance, the "hood" in the front was open,
where the tools are. So I thought I
could probably do what the photo thieves
in Toronto did. I bent a
screwdriver and in
1/4 of a minute, the window was broken open.
That was our good fortune, because we were only
in our swimsuits in the considerable cold at
5 in the morning, and of course there are no
car repair shops up there anymore. Then came a wonderful
stretch, mountains and lakes, all about
1000 meters high, forest and very isolated houses.
Once, a big brown bear
nearly crossed our path. He was sitting in the grass by
the roadside, and as we drove by, he
immediately turned around and trotted to the forest. A pity
we couldn't photograph him.
In Chicoutimi, we even went to the movies; we
saw the same film as on my
birthday, "Cinderella." There's
a lot of timber industry there; the
rivers are full of logs, so many, I have
never seen anything like it. We took a lot of
pictures, I'll send some of them sometime. The next
day, we went through Laurentide Park,
which is a provincial park of 4000 square
miles and with 1600 lakes. You're not allowed
to hunt there, but people fish for
salmon up to 2 meters long and a type of trout.
But it's hard to stay there for long
and live, because everything to the left and
right of the road is such a primeval forest that you
can't go two meters into it. But we went
swimming anyway. The road is simply
carved into the forest; to the left and right,
the trees are still lying there. Once we had
to stop when two cow moose ran across
the road. And once we saw two little
bear cubs playing. Before the park, you
are registered, time and date, and
at the exit, you are checked to see if you
have been hunting. That was a magnificent
tour. Then we went across the
Laurentians to Shawinigan Falls and St. Jerome.
From there, we drove up the highway into the
High Laurentians, again over 1000 meters high,
but not as wild, more like the Allgäu, and
with real Allgäu-style houses, as
many Germans and Swiss live there. And
there we happened to pass by "Santa Claus Village."
It is so nice and tasteful.
Four Montreal businessmen built it.
Of course, they make money from it, but the
complex cost them 200,000 dollars. It is
a large hill with about 7-8 colorful
witch's cottages, each with something different inside.
In one, Native American handicrafts are sold,
in one only postcards, in another
you can—in a chapel is the Christmas
nativity scene and you can ring the bell and
make a wish. And in one little house
is Santa Claus with his sleigh
and he talks to the children and gives them
lollipops. He spoke German with us and told
us all about the place. In the garden
and in the houses, goats and llamas
and young bear cubs run around, all very tame.

which are hardly around anymore, but are beautiful to look at.
The chapels are lined with glass bottles
that are stuck in the walls.
\end{document}
