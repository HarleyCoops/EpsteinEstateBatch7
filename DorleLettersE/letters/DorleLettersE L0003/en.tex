\documentclass[12pt]{article}
\usepackage[utf8]{inputenc}
\usepackage[T1]{fontenc}
\usepackage[english]{babel}
\usepackage{geometry}
\geometry{margin=1in}
\usepackage{parskip}
\begin{document}
\section*{DorleLettersE L0003 (English)}
\noindent
Toronto, 10/10/55

Dear Mother,
The "gotteskind" arrived the day before yesterday. Thank you so much for it. I'm sorry you had such a hassle with it. But we went right out today to copy the parts from the rolls; I got three boys to help me. For now, we've cast the roles like this: (God the Father: André); (Angel Gabriel: Erika); (Joseph: Sigismund); (Mary: Nicki); Guldinsach: Uli; (Innkeeper's Wife: Vera); (Herod: Manfred); (Death: Werner); (Devil: Wolf); (Attendant: Horst Wermescher); (Michel: Horst Miele); (Stoffel: Siegfried); (Cyrich: Werner Kjarr); (Melchior: Hans Dieter); (Walthauser: Heinz); (Caspar: Reinhold). We're still not sure about the star singers. I'm not acting in it, since I have to take on the directing and anyway there are more girls than roles, maybe the first star singer at most. The first rehearsal is next Sunday. Inge wants to make the costumes. We do, however, have good musicians in the group. Manfred plays piano well, as does Hans-Dieter; Klaus (Inge's husband) plays the violin very well, Ellen too, Werner plays mandolin and Reinhold plays the recorder in addition to the accordion. André also plays piano very well. Uli plays the transverse flute, but she didn't bring it with her. I was just at Sigismund's and showed him the script. He was concerned about whether we'll get the pronunciation right, since apart from Reinhold, literally everyone is German. But I think we should give it a try in any case. I want to see if we can fill a whole evening with the play and with choirs and musical performances. It is a lot of work, having to take care of everything, especially since I'm overbooked again with evening school. I have pottery again (where I'm now on the electric wheel), then gymnastics (Mechi goes once a week now too), and then shorthand and typing. You always have to do homework for shorthand and typing, too. By the way, were you serious about the secretary position with Uncle Max? That would really be something. I'd certainly be interested in that.

Yesterday, on Sunday, we set off in three cars at 4 o'clock in the morning because we wanted to go on a hike. We drove to Collingwood on Georgian Bay and climbed around in caves and rock crevices there, and took excursions through the woods. It was so beautifully sunny and warm, and everyone really enjoyed the outing again. In the evening, though, we had to be back by 8 o'clock for the club, where we heard a highly interesting lecture on the influence of world history on the church. A good speaker can make even the most boring topic interesting.

Inge and Klaus are at the club more often again now; Klaus likes it with us. I'm still there often, though of course not as much as before. Inge is coming to gymnastics again, but naturally she has more to do with the household now. Klaus is an electrical engineer and has a job as a technical draftsman. Imagine, 8 days before the wedding, his sister and his cousin were tragically killed in Germany. That's why we kept the wedding so small. The ceremony was at 10:30 in the morning, and it was really lovely; Dr. Görgigen does it very nicely. Manfred and I were the witnesses. Otherwise, besides the pastor's wife, no one was in the church, since the youth group was away on a trip. Then we had the wedding meal, potato salad and schnitzel, and then all sorts of little "appetizers." I had baked a beautiful buttercream torte, with roses and pansies piped on top. The youth group had given Inge a lovely tea service, which was put to use right away. Then we set off for Woodview, where the two of them stayed, and Manfred and I drove on. That was quite a nice drive, over 300 miles (480 km); we got to the others in Baysville at 11:30 at night.

Inge's sister Hilde is doing well in Germany. She isn't working yet, but will soon start at an American office in Wiesbaden. She spent the whole summer traveling with her father, in Holland and Belgium and in Italy.

Our new apartment is very nice; we have more space now, but we also have a corresponding number of visitors. Sometimes half the group is here. You asked about Christmas wishes. I would really love an anorak for skiing with a hood, preferably blue. Or else a good camping stove. For smaller things, I'd like a case for filters and a lens hood to attach to the strap, or a wash bag.

* I still haven't answered your last letter, but I wanted to at least write you a few lines. Your mother wrote to me as well. Warmest greetings from all of us.
\end{document}
