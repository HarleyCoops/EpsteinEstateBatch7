\documentclass[12pt]{article}
\usepackage[utf8]{inputenc}
\usepackage[T1]{fontenc}
\usepackage[english]{babel}
\usepackage{geometry}
\geometry{margin=1in}
\usepackage{parskip}
\begin{document}
\section*{L0005 (English)}
\noindent
Toronto, June 2nd
55

Dear Mother, brothers and sisters!

I was very happy to receive the post-birthday letter and thank you so much for it. The jug was unfortunately still in the kiln to be fired, but I have sent it off now. You see, the firing takes quite a long time, then we glazed them on the last day of school and then our teacher wasn't there anymore. I also have the record with Hungarian Dances, 16 dances on one record, played by Jascha Heifetz. Did you also not know Jascha Heifetz and Yehudi Menuhin? I already heard both of them here last year; they are the two best violinists in the world. Mechi is still asking me about it. I have already renewed my subscription for the symphony concerts for next fall and winter. Mechi thinks she will fall asleep. The program will be very good. 5 of the best conductors in the world are coming, Heifetz and Rubinstein (violin and piano). I'm already looking forward to it again. From the Metropolitan Opera I saw "Madame Butterfly" and "Carmen." The hall is not very suitable for the purpose, it's actually a sports arena for hockey, wrestling, boxing, etc. But the voices and the scenery were uniquely beautiful, you can really tell that it's the best of the best. I liked Carmen better, but maybe only because I was too tired during Madame Butterfly, because two days before that I had slept on the stove, driven 300 km by car, and then went straight to the opera. That was on the Monday of our long weekend at Deer Lake. We went with 5 cars and had 4 cabins. But there were only 5 beds in the girls' cabin, so I slept on the floor in a sleeping bag, but I didn't pay anything either. Each cabin cooked for itself, so we girls of course made out well and cheaply. We had to leave the cars at the beach, the cabins were all on islands, there was no electricity, just an oil lamp for light. One evening we had a campfire on an island, it was wonderful, there were plenty of dry trees, the islands are mostly not large and so numerous and close together that one could easily get lost, even during the day.

Last weekend was Queen Victoria's birthday on Monday, so all the office and factory workers had three days off, they call it a "long weekend" here. On weekends like that, the youth group goes on its outings. 18 from the club drove to Deer Lake, 150 miles (225 km). Since I still worked on Saturday and Siegfried did too until 12 o'clock, we and two other boys didn't leave until 3 o'clock in the afternoon. We drove mostly in the dark, because all the others had already been gone since 7 in the morning. We arrived at 7 in the evening. We had to row in boats to the islands where the cottages were, in which we were cooked for by the girls and the others by the boys. Frau Kockendörfer, Inge, Hilde, Jule, Dagmar and I were the girls. I immediately had a proper supper, potato salad, sausages and bread. In the morning at seven, since we got the overnight stay paid for, we cooked for ourselves, which is why it was much cheaper this time. Afterwards I went with Horst. Then I went fishing with Heinz, although I don't care for it at all, but I wanted to see if he would catch anything. He caught 3 fish, but then he was so bloody in the face, in his hair, and on his arms and legs, often from the mosquito bites, that we couldn't stand it anymore. The next morning Siegfried, Heinz, Dagmar and I went out in the rowboat at 6 o'clock sharp to take pictures. The lake is very large and full of countless islands with wild forest, rocks, and brush on them. Often only a rock sticks out of the lake or wild trees. In the many pictures that Heinz, Dodo and I took, you will see everything clearly. At 9 o'clock we drove back to the cabins, had breakfast, swam and played in the lake from the boat, and fished. In between, Heinz, Siegfried and I went out again. He had seen a wonderful island in the morning, where we wanted to sunbathe. But it didn't come to that, because there were so many from the club on the lake who kept following us. When they wanted to tow us with their motorboat, Siegfried sprayed them with water. That was the nasty declaration of war for the water fight that followed. But first we left our clothes and the cameras and watches on an island. Then we attacked them live in the bay. That's how we fell out of the boat, soaking wet. Jo and Horst were in one boat, Varna, Dagmar and Herold in the other. The last ones sailed to our island, hoisted our flag, got it wet, and quickly took off. But we didn't mind, because we were the only ones, on the way home we first went for a swim from the boat, me in shorts and a T-shirt, because my swimsuit is still in the store, as usual. The water was wonderful, as we were drying off, we sat our flag hung up on an island with the trees. We are proud of it, since we managed until the rest in the evening, when it started to rain. These lines are a nice report etc. of the whole battle. In the evening we all sat in one cabin where Blomquist had heated it. And when everyone was asleep we made a campfire on an island and sang, we came home at 11:30 at night. The mosquitoes were acting wild, if you didn't smear yourself with some stuff that stank worse than the fly repellent for the cows in Stitten. My whole head was full of blisters, it was bad. On Monday morning, the bed was no longer to be seen at first, but later the sun came out again. The three of us laid ourselves out on an island and painted. Darli, Inge, Hilde, Frau Kockendörfer, Horst, as well as Hans-Dieter already went back to Toronto at 1 o'clock, the ones with the Kockendörfers went to the opera in the evening. The next two cars drove home at 3 o'clock, so that then only Heinz, Siegfried, Uconfried and I were together in one cabin. The weather wasn't very nice, so we went out again for a swim before on the way home we got into a terrific, lukewarm thunderstorm, and we arrived at the cabin soaking wet. I fried bacon, cooked pea soup and coffee, it was uncomfortable. The coffee water just wouldn't boil, because the stove was almost out, so in the meantime we went out fishing, but nobody can find anything with Heinz, but we had so much fun doing it, because the fishing line got caught on the many roots in the lake every five minutes, and we didn't catch a single fish. In the evening at 9:30 we left and in the morning at 3:30 I was in bed. I wasn't freshly rested in the morning, but still standing at the quay, and everywhere laundry lines from judges and a beautiful memory of the 3 days. We drove so late because you can hardly get through between 6 and 9 in the evening, the traffic is so heavy after a weekend like that, and 1 o'clock in the afternoon was too early for us. The last day was almost the most beautiful.

This last week I then spent every evening, yes, faithfully developing films and enlarging pictures, that's why I didn't write sooner. - At the hospital I rubbed myself with salt water, ether, with all possible remedies, so that it looked bad. Now it's pretty much all over, just one spot is a little red, thank God there was no infection, that happens quickly too. - By the way, I'm on health insurance starting July 15th, it's a better feeling.
This Sunday Heinz, Siegfried, Jo, Enke, Dahl and I were at Lake Minnetonka, it was beautiful, Jo, Heinz, Siegfried and I went swimming, the water is chilly but nice. You know, I haven't had such lazy Sundays for as long as I can remember, and I wish with all my heart for all of you to be able to join in, but you don't get anything out of that. I want to see as much as possible this one year.
Imagine, last week the latest film with O. W. Fischer was playing here, Siegfried went to see it. I go to the cinema very, very seldom.

Some of it has nothing to do with reality, but some of it is true.
\end{document}
