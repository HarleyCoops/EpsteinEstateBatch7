\documentclass[12pt]{article}
\usepackage[utf8]{inputenc}
\usepackage[T1]{fontenc}
\usepackage[english]{babel}
\usepackage{geometry}
\geometry{margin=1in}
\usepackage{parskip}
\begin{document}
\section*{L0004 (English)}
\noindent
Toronto, May 30th, 55.

Dear Mother!

I just read your letter, and I wanted to answer you right away, since it's been a long time since I've written. I wasn't upset, I was just wondering if I might have offended you somehow, and that's why you hadn't written. You can always write to Dorle first, she shouldn't think that you're neglecting her. She has it so easy, since she is very much alone. For the last few weeks Inge and her parents were here, and they had no time at all for Dorle, which I didn't think was very nice. Dorle had lent them the car for 3 weeks, and when they wanted it one Sunday, they couldn't get it. It's possible, though, that she didn't dare ask for her own car back. She's so good-natured with Inge, and Inge takes advantage of it. I talked it over with Manfred once, and he feels the same way. You remember, when I first got here, I wrote complaining that Dorle and Inge were driving out West. When it came time to ask for vacation time at work, Inge suddenly didn't want to go anymore, and good-natured Dorle didn't either, even though I would have gone with them. So they both moved their vacation to the first half of July and wanted to drive to Halifax with the car to see the waves. They asked if I wanted to come, but honestly, that wouldn't be a vacation for me. The trip out West is now definite for the Laubs, they might even go in July, and that's why we haven't talked about it much lately. The other day I asked Dorle if her vacation trip was 100\% on, and she said that Inge's fiancé is coming on July 1st and so of course they aren't going. Isn't that something, coming from Inge? Is that the kind of friend you can make plans like that with? But I've always said, that is obviously no friendship at all. I'd rather have no friend than one like that. I'm sure that once the fiancé is here, Inge won't want anything to do with Dorle anymore, because they won't need the car anymore either, as they're buying one for themselves. I've been wondering if I should take her with me to B.C., but that won't work out; besides, the car will probably be full, and I don't know if she wants to take that much vacation time. - She's also often alone in the evenings, even when I'm home, but she doesn't know what to do with me. When Siegfried comes by to go out for the evening, she'll go with him. It's not easy, believe me, it's not very pleasant. - The other day, Uli gave a talk for the club evening. Manfred always reads from the Bible, which doesn't interest me at all. But Uli did something really great. He's also quite a pacifist, hardly ever goes to church, like me, but is certainly a good Christian who always contributes so nicely to the group. The point he was making was that young people, above all, must help wherever they can. I think it touched everyone's heart, mine too. By the way, Uli had some bad luck. He tore up his arm at work and was on crutches for a few weeks, which makes you feel bad yourself.

This Friday I was at a graduation ceremony at a private—Catholic—high school. You see, Miss O'Connor, who is now the Mother Superior, had a sister there, and she asked me to take some pictures at the ceremony. Unfortunately, her brother had an accident that very day, so she couldn't come herself. I left the hospital at 2:30 and met her parents and her sisters at the school. (One of the sisters is also a nun at our hospital and looks just like Ruth Leuwerik.) I went up to the third row where all the teachers, priests, and nuns were sitting and filmed. The priests were so nice, they let me sit on a chair so I could be a bit higher up. The girls wear long white dresses and all get a crown, the whole thing is much more festive than back home. But otherwise it's the same, one of the students, called a graduate, gives a speech, as does the principal, prizes are awarded, choirs sing, and one girl played the Taylorike sewing machine, a very lovely child. And then everyone meets in the school's inner courtyard, the graduates get presents from their friends and acquaintances, they congratulate each other, and from a wonderfully set table, everyone can take as many sandwiches, cakes, and punch as they want. And most of all, there's picture-taking, nothing but picture-taking. Not only were the kids happy with my 8mm film, but I was too, which made me very happy. By the way, what the Canadians notice most is how well they are enlarged. They're not familiar with that here. - After the ceremony, I went home with the family, where my Miss O'Connor also arrived. We chatted and later played games with a few strangers from the exhibition. There were quite a few Germans around, all adults, two of them as well. The father is a farmer and doesn't live in Toronto; the two sisters have a wonderful apartment here. There are 12 children, imagine that, one of the daughters is a nurse, a Red Cross Sister in the Air Force in Korea, that's a fulfilling profession too. - I got home at 9 in the evening. I'm always happy when I get to be in Canadian company, especially with Miss O'Connor, she's a truly fabulous woman. I'd guess she's about 33 years old, loves flowers terribly, and she also says she often misses the country, since she grew up on the farm. I think she's the main reason I like being at the hospital so much. She recently told the Sister who is in charge of all the nurses that I would like to work as a nurse, and imagine, it's just possible I might be allowed to go into training on our floor, that would be my dream. But it's not a sure thing yet.

Whenever I have any flowers to spare, I always arrange a few nicely for her desk; she's always so happy about it. It really is the best and most important thing when you get along so well with your boss. I can take my days off whenever I want, and even if I want to leave early sometime, it's never a problem. Last Saturday, she and I both had errands to run, I told her about mine, and she gave me the afternoon off, which I actually wasn't supposed to have until next week. But I would never lie to her, that would never occur to me, even though it's done so much here. Dorle also said I should just say I had to go to the dentist, but that would be shameless, I just don't do things like that. She has already told me about Uster, as she was there last year too. She wants to go to Europe next year, so she'll have to visit me then, that would be lovely. - By the way, you'll of course get some of the pictures from the ceremony, I also want to send a few to Miss Lehmann, she might be interested. I think it's wonderful that I was able to see that, as usually no one ever... It will be harder for her to get used to the family again and to adapt to them.

June 19. I just saw that the letter hasn't been mailed.
The last time we were at the boys' place was before the big holidays.
On Saturday there's a big birthday party for Joe and Manfred and Ellen in Udora.
We'll stay the night there and on Sunday morning there's a swimming competition in Lake Simcoe. Inge and I bought ourselves air mattresses, it's fun to be able to lie on the water for hours like that. Erdmute wrote to us recently, she's back in Wuppertal now and is also teaching cooking classes, which she has to laugh about herself. We'd like to visit Christoph's cousin here sometime, he's 10 miles away from here, and his name is Klaus Hofmann.
Warm greetings from Dirk.
\end{document}
