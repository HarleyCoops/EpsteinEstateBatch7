\documentclass[12pt]{article}
\usepackage[utf8]{inputenc}
\usepackage[T1]{fontenc}
\usepackage[english]{babel}
\usepackage{geometry}
\geometry{margin=1in}
\usepackage{parskip}
\begin{document}
\section*{L0002 (English)}
\noindent
Toronto, Feb. 24, 55

Dear Mutti,
I want to apologize for my letter; I'm sorry, I was really thinking too selfishly. I certainly won't do that again. It may well be that I was naive and didn't think everything through back then. One just shouldn't send off bad or unpleasant things right away, but rather wait until everything has cooled down a bit. I didn't know that my Christmas letter, which I sent before Michi's, didn't arrive, so please, forgive me again. Please don't dwell on it, if it was colder without the Eber boys. And when I suggested back then whether you all might want to come, it was because it would have been easier for me, not because I had a guilty conscience about it, since I had actually promised to come back. I had been firmly resolved to do that back in Germany; after all, I was happy at the Sitzels' even without working outside, and by the way, I still am today. What can I do if I like it so much here? In the beginning, I fought against these thoughts often enough \& we still haven't seriously discussed whether it's advisable for all of you to come over, other than calculating how quickly we could have a house. I'm not trying to persuade Mochi either, simply because she knows you all better. I don't think she has ever seriously considered it. I believe she is already quite mixed up. She is really adapting tremendously, almost too quickly. 1 1/2 years is almost too long if one plans to go back and continue living over there. I believe, or have the feeling, that for her, 1/2 a year would be the absolute limit. She bravely tells herself that she would never stay here and will definitely be in Germany next spring. Just don't think I want to get rid of her; I just mean that it will be very hard for her

to readjust again, also to German circumstances, and an adjustment to more Rhenish circumstances is much more difficult. But let's hope for the best. We enjoyed our ski trip very much. Me too. It was great. We had snow and sun, unfortunately only 4 pairs of skis, so I almost didn't get to ski at all. And when a pair of boards was available, my ski boots were in use, since several of the girls didn't have boots. In the evening we had a costume ball, which unfortunately turned out to be pretty lousy, as most people had sore muscles and were tired from all the fresh air. But the next day at noon, before we left, we had another nice dance. The ski instructor and his wife were also there, which was nice when we sat around the fireplace in the evenings and sang while he played the guitar; that was a really great atmosphere. We sang a lot of songs like that with a chorus, and where one person sings a solo; "Stumpfsinn, der mein Vergnügen," and "Schön ist ein Zylinderhut," etc. But also beautiful folk songs. The ski instructor is going to build himself a cabin up there in the summer. We took some nice pictures; Mechi has already developed some of them. Starting today, Mechi is working overtime at the hospital, so she'll be earning really good money, which is great; she's now working from 4:15 - 7. She also earns quite a bit from the pictures. I'll also be happy when she has paid off her debts, not because I need the money, but it is getting to be a bit much. She has paid off her enlarger; I think she has already paid Ehr off herself. She certainly won't regret the purchase. What I find a bit more unnecessary is that she now wants to buy skis for \$18, but that's her own business. I don't think our vacation to B.C. is going to happen, which is a shame for Mechi, although it might have been almost financially impossible for her. With the loss of work, it would have come to about \$150. So we'll probably go to Nova Scotia and I'll go up north, which will surely be very nice too. Inge will also come along; Mechth isn't very enthusiastic about that, but that does happen more often, that one gets along better with his wife than with his sisters, especially since we did a lot together last summer. Next week we're going to the opera, "La Traviata." On March 3rd we're invited to Uli's birthday party, 12 from our group; we want to give him an air mattress or a sleeping bag, which is very useful on camping trips. Inge and Hilde are already over the moon; their mother is coming by plane on Good Friday and staying for 7 weeks, and will then travel back to Germany with their father. So today they bought a couch and built a tea table; they got the money for it back from overpaid taxes. That's certainly more pleasant than my situation last year, when I had to pay an additional \$42.

I recently saw "Desirée" here with Marlon Brando, which was good. But "On the Waterfront" (Marlon Brando) was even better; in Germany I think it's called "Die Faust im Nacken." The movies here are just as good, or even better, at least as far as the production is concerned. "Romeo and Juliet," "Julius Caesar," "The Living Desert" and "The Vanishing Prairie," the two animal films by Walt Disney, "The Kidnappers," a children's film, "Doctor in the House," a film about medical students, "Adventures of Robinson Crusoe" was very nice. I haven't seen a real medical film yet, though. But there are very nice historical films, "The Young Bess," for example, exactly like the book I have. Or "Lady Hamilton" about Lord Nelson and the naval battle of Trafalgar. Then "Cleopatra" from the assassination of Caesar. Subscrip... phony concerts is good, they always perform very good pieces, will play on one of the evenings next. On March 18th, the Berlin Philharmonic Orchestra is playing here at Massey Hall. The tickets for it were already sold out in January. They are playing Handel, Brahms, Till Eulenspiegel, and the overture to Tannhäuser. The Vienna Boys' Choir will be performing in April; I would really like to go to that.

We started doing pottery a while ago, since the carving fell by the wayside due to a lack of time. We've already finished a few jars and jugs and vases, that is, some of them haven't been fired yet. Last time we glazed a few, and now they will be fired again. Because when the glazes are wet, unfired, they look red, and then they turn light green when fired. It's like with reversal film in photography. We still have a lot of fun with sports; Mechi now goes twice a week too. Afterwards, we always go swimming. It was the first time Mechi had ever been to an indoor swimming pool. When I had the late shift and didn't get home until 8 o'clock, Mechi always used to cook for me, but now she's getting better and better. By the way, Jenny wrote to me that Mechi looks so angry in the picture I sent once. That was taken at the Santa Claus Parade. Mechi was snapping picture after picture, and then she wanted to change the film and found that both rolls in both cameras had been torn out, so she was angry, she wasn't just looking that way.

Love, Dorli.
\end{document}
