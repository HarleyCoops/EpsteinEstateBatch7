\documentclass[12pt]{article}
\usepackage[utf8]{inputenc}
\usepackage[T1]{fontenc}
\usepackage[english]{babel}
\usepackage{geometry}
\geometry{margin=1in}
\usepackage{parskip}
\begin{document}
\section*{L0003 (English)}
\noindent
Toronto, May 2, 55

Dear Mother!
I wish you all the best for your birthday.
Thank you very much for your letter, I was
very happy to receive it. And then the
sleeping bag for my birthday. It's just
fantastic. I've already tried it on ten times,
it's absolutely wonderful. We had always wanted
to sew some ourselves, i.e., buy the comforter
and then sew the zipper in;
but of course we wouldn't have had
such a long zipper, and it wouldn't have
been so beautifully lightweight either. With the cover,
I can also always keep it in the front of the car.
Thanks again. From Mech I got a
nice wallet and a
ticket for the Metropolitan Opera New
York for "Carmen". Inge and Hilde and
their parents gave me a wonderful pearl
necklace, a beautiful yellow fabric for a dress,
and also an opera ticket for "Madame Butter-
fly". I plan to sew the dress next week.
I already have the pattern. On my birthday
we went to the cinema to see "The Glass Slipper"
with Leslie Caron; it's the fairy tale of
Cinderella, modernized, truly wonderful.
Mech really liked it too. On my
birthday morning, I wanted to get up an hour early and
open my presents, which is something I've
done 6 times now
on my birthday.
But Mechi said
that wasn't happening, and
so we argued in bed
in the morning about whether I
was allowed to open them or not.
Then I looked at the
clock, and it was 10 minutes
to 7, and at 7 o'clock
I was supposed to be at
work. So I left the presents
be and rushed off
like crazy.
Right now, Inge and Hilde's
parents are here. The
father is unfortunately only here for
2 weeks; he's on a
business trip through the
States. He has a few large department
stores in Germany, so he's mainly
interested in the big
department stores here. He
has, I think, already shot 20 rolls of film.
He is really
very nice, 57 years old,
while the mother is only
29. They just got married
last year, but they had
already known each other for 10 years

and just waited
until the girls were older.
She immediately addressed me with "du,"
which was nice.
On Saturday, we went
with two cars
to the Falls; Mr.
Kochendörfer's travel companion
and his wife were with us too.
It was very nice, for once
from a completely
different point of view,
that of the rich world.
We were visiting, you see, so the
dollars just flew \&
of course we were all
treated. But for that, we got to see something and ate well
on the 12th floor of one of more than 1 million
hotels with a view of the Falls. We
rode all the elevators and railways
that go under the Falls, only the
two ships that used to travel below the Falls
weren't there, because they
burned down completely, lock, stock, and barrel,
two weeks before the season started. That was a
much-talked-about event. The ships had been
built down there over 50 years ago,
since you couldn't get them in there
fully assembled. It's a huge loss of business
if they're no longer running. Imagine, Hilde
is going back to Germany with her mother
on June 30th. It came about rather
suddenly, but her father wants her back,
since she has lost so much weight and is so
nervous, so it's for the best. She has
a more coveted life over there anyway.
Later she is supposed to go to the French-
speaking part of Switzerland for a year to learn French.
But she's leaving all her things here, since she
might come back if she
doesn't like it over there anymore. I'm
curious to see how she settles back in
over there, but as I said, she's going into very good circumstances there, so she'll certainly
be able to eat bananas and oranges there too.
But I think it's nice and smart of the
father to let the girls be on their own
for a while and basically fend for themselves
for 2 years. They didn't save
much, but still.
Inge is expecting her fiancé in July
or August. We'll see what comes of that.
Last Sunday we were at Lake Erie
on a peninsula that was basically
just sand dunes, and the sun
was really blazing.
I also received a long
birthday letter from the Betzels. They're also having a hard time
finding workers; for the household, though,
they've had a deaf-mute girl for
2 years, which seems to be working out quite well.
But otherwise they're making progress; in recent years
they've done a lot of new planting, for the winter
they bought a large dump trailer,
and the barn is being rebuilt. That makes
the whole thing fun.
Well then, once again, very warm birthday
greetings and all the best.
Doris
\end{document}
